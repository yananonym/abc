\section{Vector Space Structure and Constraint Polytopes}

The integrality constraints of ratio-based multiplicative bases define a profound geometric structure in exponent space. This section develops the geometry and algebra of these constraints.

\subsection{The Integrality Constraint System}

For the canonical epimoric basis, an exponent vector $[b_1, b_2, \ldots, b_m]$ produces an integer if and only if for every prime $q$:

\begin{equation}
v_q\left(\prod_{k=1}^{m} p_k^{b_k}\right) \geq v_q\left(\prod_{k=1}^{m} (p_k - 1)^{b_k}\right)
\end{equation}

Expanding:

\begin{equation}
\delta_{q,p_k} b_k \geq \sum_{k=1}^{m} b_k \cdot v_q(p_k - 1)
\end{equation}

where $\delta_{q,p_k}$ is 1 if $q = p_k$ and 0 otherwise.

\subsection{The Constraint Polytope}

The set of valid exponent vectors (allowing negative integers) forms a convex polytope in $\mathbb{R}^m$ defined by these linear inequalities. For non-negative integers only, we restrict to the cone $\mathbb{R}_{\geq 0}^m$.

\subsubsection{Fundamental Properties}

\begin{enumerate}
\item The polytope is \textbf{non-empty}: the vector $[0, 0, \ldots, 0]$ (corresponding to $n=1$) satisfies all constraints
\item The polytope is \textbf{closed under integer scaling}: if $[b_1, \ldots, b_m]$ is valid and $\lambda \in \mathbb{Z}$, then $[\lambda b_1, \ldots, \lambda b_m]$ is valid
\item The lattice points of the polytope correspond exactly to representable integers
\end{enumerate}

\subsection{Recursive Structure of Constraints}

A key property specific to the canonical epimoric system: the denominators $(p_k - 1)$ involve only primes with index $< k$:

\begin{equation}
(p_k - 1) = \prod_{j < k} p_j^{v_{p_j}(p_k - 1)}
\end{equation}

This creates a \textbf{recursive obstruction structure}. The constraint for prime $p_k$ depends on exponents $b_j$ with $j \geq k$, but not on the previously determined $b_j$ with $j < k$ (except through their cumulative contribution to numerator valuation).

\subsubsection{Consequence}

We can validate exponent vectors using a backward-pass algorithm:

\begin{algorithm}
\caption{Validate Exponent Vector for Integrality}
\begin{algorithmic}
\STATE Input: $[b_1, \ldots, b_m]$
\STATE Initialize: deficit array $D[1 \ldots m] \leftarrow 0$
\FOR{$k = m$ down to $1$}
\STATE Compute $v_{p_k}(p_k - 1) = 0$ (always 0 by definition)
\FOR{$j = k+1$ to $m$}
\STATE $D[k] \leftarrow D[k] + b_j \cdot v_{p_k}(p_j - 1)$
\ENDFOR
\IF{$b_k < D[k]$}
\STATE Return FALSE (integrality violation)
\ENDIF
\ENDFOR
\STATE Return TRUE
\end{algorithmic}
\end{algorithm}

\subsection{Counting Valid Vectors}

For vectors with a fixed exponent sum $S = \sum_{k} b_k$, we ask: how many valid vectors are there?

Let $V_{\text{valid}}(S, m)$ denote the count of valid vectors with sum $S$ using exponents $[b_1, \ldots, b_m]$.

\subsubsection{Growth Estimates}

Empirically, for increasing $S$:

\begin{equation}
V_{\text{valid}}(S, m) \ll \binom{S + m - 1}{S}
\end{equation}

(the binomial count with no constraints).

The density of valid vectors:

\begin{equation}
\rho(S, m) = \frac{V_{\text{valid}}(S, m)}{\binom{S + m - 1}{S}} \to 0 \text{ as } S \to \infty
\end{equation}

This vanishing density is one source of the semi-regularity observed in $\Omega_E(n)$: as integers grow, they become increasingly sparse in the space of all possible exponent vectors, because larger exponents require increasingly tight satisfaction of the constraint polytope.

\subsection{Forbidden Regions and Prime Gaps}

\subsubsection{Definition}

A region in the exponent space is \textbf{forbidden} if it contains no valid integers, i.e., no lattice point of the region can be expressed as an exponent vector for any natural number.

\subsubsection{Correlation with Prime Gaps}

Large prime gaps create corresponding regions of ``sparsity'' in the constraint polytope. For example:

\begin{itemize}
\item The gap between primes 23 and 29 (gap of 6) creates constraints that make certain exponent combinations impossible
\item Conversely, twin prime regions create tight clusters where the constraints are permissive, allowing dense packing of valid vectors
\end{itemize}

\subsection{The Valuation Matrix and Rank Analysis}

Define the \textbf{constraint matrix} $C$ where:

\begin{itemize}
\item Rows are indexed by primes $q$
\item Columns are indexed by basis elements (ratio indices $k$)
\item Entry $C_{q,k} = v_q(p_k - 1)$ (the exponent of $q$ in the factorization of $p_k - 1$)
\end{itemize}

For example, the first few rows:

\begin{center}
\begin{tabular}{c|ccccc}
 & $k=1$ & $k=2$ & $k=3$ & $k=4$ & $\cdots$ \\
$q=2$ & $0$ & $1$ & $2$ & $1$ & $\cdots$ \\
$q=3$ & $0$ & $0$ & $0$ & $1$ & $\cdots$ \\
$q=5$ & $0$ & $0$ & $0$ & $0$ & $\cdots$ \\
$\vdots$ & $\vdots$ & $\vdots$ & $\vdots$ & $\vdots$ & $\ddots$
\end{tabular}

where the entry $C_{2,3} = 2$ because $p_3 - 1 = 5 - 1 = 4 = 2^2$.
\end{center}

The rank of this matrix (over $\mathbb{Z}$ or over finite fields) governs the dimension of the constraint space.

\subsubsection{Rank Properties}

For the canonical epimoric system:

\begin{itemize}
\item The matrix is lower-triangular (once properly ordered): entry $C_{q,k}$ can be nonzero only if $q < p_k$
\item The rank grows with the number of primes considered
\item Rank deficiency occurs at finite truncation; the infinite matrix has full rank
\end{itemize}

\subsection{Geometric Visualization in Low Dimensions}

For small exponent dimensions (e.g., $m = 2$ or $m = 3$), the constraint polytope has geometric structure:

\subsubsection{Example: The 2-Dimensional Case}

Using only the first two basis elements $(2/1)$ and $(3/2)$, the constraint is:

\begin{equation}
b_1 \geq b_2
\end{equation}

(since the denominator of $(3/2)$ is $2$, which must appear in the numerator from $(2/1)^{b_1}$).

The valid lattice points form a triangular region: $\{(b_1, b_2) : b_1 \geq b_2, b_1 \geq 0, b_2 \geq 0\}$.

The integers representable with only these two ratios are: $1, 2, 3, 4, 6, 8, 9, 12, 16, 18, \ldots$ (numbers of the form $2^a \cdot 3^b$).

\subsubsection{Example: The 3-Dimensional Case}

Adding the third ratio $(5/4)$, the constraints become:

\begin{equation}
\begin{aligned}
b_1 &\geq b_2 + 2 b_3 \\
b_2 &\geq 0\\
b_3 &\geq 0
\end{aligned}
\end{equation}

The constraint polytope becomes more intricate, reflecting the contribution of the denominator $4 = 2^2$ from $(5/4)$.

\subsection{Topological Features and Persistent Homology}

\subsubsection{Homological Analysis}

The constraint polytope, viewed as a simplicial complex or cell complex, admits topological analysis via persistent homology. Features include:

\begin{itemize}
\item Connected components (for non-negative integer exponents)
\item Boundary structure (faces defined by active constraints)
\item Persistent cycles and cavities at various scales
\end{itemize}

\subsubsection{Correlation with Prime Distribution}

The constraint polytope exhibits structural properties correlated with prime distribution:

\begin{enumerate}
\item Large prime gaps correspond to discontinuities or singularities in the polytope.
\item Twin primes create regular, repeating features in the polytope structure.
\item The homological complexity grows in correlation with prime gap distribution.
\end{enumerate}

\subsection{Integer Programming Formulation}

The problem of finding all valid exponent vectors with a given sum can be posed as an integer program:

\begin{equation}
\begin{aligned}
\text{maximize} & \quad \sum_{k=1}^{m} b_k \\
\text{subject to} & \quad v_q\left(\prod p_k^{b_k}\right) \geq v_q\left(\prod (p_k-1)^{b_k}\right) \quad \forall q\\
& \quad \sum_{k=1}^{m} b_k = S\\
& \quad b_k \in \mathbb{Z}_{\geq 0}
\end{aligned}
\end{equation}

Modern integer programming solvers (branch-and-cut, cutting planes) can enumerate solutions and extract structural information about the feasible region.

\subsection{Open Questions}

\begin{enumerate}
\item Does the constraint polytope have a closed-form description in terms of prime gap sequences?
\item Can we compute the homological invariants of the polytope explicitly?
\item Is there a generating function for $V_{\text{valid}}(S)$ that relates to analytic number theory?
\item Can forbidden regions be characterized algorithmically to detect prime gaps?
\end{enumerate}

These questions suggest deep connections between the combinatorial geometry of exponent space and fundamental properties of the prime sequence.
