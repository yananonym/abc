\subsection{Fourier-Character Formulation: Abstract Exponent Spaces}

\subsubsection{Abstract Formulation: General Constraint Structures}

The obstruction polytope framework encodes multiplicative constraints through the Fundamental Theorem of Arithmetic. This subsection develops the general theory of abstract exponent spaces with arbitrary constraint structures. Given any valid constraint structure derived from a finite basis, we establish the corresponding algebraic and spectral properties. This formulation applies independently of whether the constraints arise from primes or from other sources, providing a unified mathematical framework.

\subsubsection{Abstract Exponent Monoids}

\paragraph{Definition (Exponent Monoid).}

Let $\mathcal{E}_m$ denote the free commutative monoid generated by $m$ symbols $\{b_1, b_2, \ldots, b_m\}$:
\begin{equation}
\mathcal{E}_m := \left\{ (b_1, \ldots, b_m) : b_i \in \mathbb{N}_0 \right\} \cong \mathbb{N}_0^m
\end{equation}

The operation is componentwise addition: $(b_1, \ldots, b_m) + (b'_1, \ldots, b'_m) = (b_1 + b'_1, \ldots, b_m + b'_m)$. The identity element is $\mathbf{0} = (0, \ldots, 0)$.

\textbf{Key point:} We impose \textit{no prime structure whatsoever} on $\mathcal{E}_m$. It is purely an algebraic monoid, with exponents as abstract symbols.

\paragraph{Definition (Constraint Valuations).}

Let $\mathcal{V}$ be a finite abstract set (to be interpreted as valuations, ultimately primes). For each $v \in \mathcal{V}$, assign:

\begin{equation}
\mu_v : \mathcal{E}_m \to \mathbb{Z}_{\geq 0}
\end{equation}

satisfying additivity: $\mu_v(\mathbf{b} + \mathbf{b}') = \mu_v(\mathbf{b}) + \mu_v(\mathbf{b}')$.

Additionally, for each pair $(v, k) \in \mathcal{V} \times \{1, \ldots, m\}$, specify a non-negative integer deficit:
\begin{equation}
\delta_{v,k} \in \mathbb{Z}_{\geq 0}
\end{equation}

\paragraph{Definition (Validity Condition).}

An exponent vector $\mathbf{b} \in \mathcal{E}_m$ is \textit{valid} with respect to the constraint structure $(\mathcal{V}, \{\mu_v\}, \{\delta_{v,k}\})$ if:
\begin{equation}
\label{eq:validity_abstract}
\mu_v(\mathbf{b}) \geq \sum_{k=1}^{m} b_k \cdot \delta_{v,k} \quad \forall v \in \mathcal{V}
\end{equation}

Denote the set of valid vectors as $\mathcal{V}_{\text{valid}} \subseteq \mathcal{E}_m$.

\textbf{Interpretation:} In the epimoric system, $\mu_v(\mathbf{b})$ encodes the numerator's divisibility by prime $q$ (where $v$ represents $q$), and $\sum_k b_k \delta_{v,k}$ encodes the denominator's divisibility. The constraint enforces that the numerator is sufficiently divisible.

\subsubsection{Character Theory on Exponent Monoids}

\paragraph{Definition (Character Group).}

A \textit{character} of $\mathcal{E}_m$ is a monoid homomorphism:
\begin{equation}
\chi : \mathcal{E}_m \to \mathbb{C}^{\times}
\end{equation}

satisfying $\chi(\mathbf{b} + \mathbf{b}') = \chi(\mathbf{b}) \cdot \chi(\mathbf{b}')$ and $\chi(\mathbf{0}) = 1$.

The character group $\widehat{\mathcal{E}}_m$ is the set of all characters:
\begin{equation}
\widehat{\mathcal{E}}_m := \text{Hom}_{\text{mon}}(\mathcal{E}_m, \mathbb{C}^{\times}) \cong (\mathbb{C}^{\times})^m
\end{equation}

Each character is uniquely determined by an $m$-tuple $(\chi_1, \ldots, \chi_m)$ with $\chi_i \in \mathbb{C}^{\times}$:
\begin{equation}
\chi(\mathbf{b}) = \prod_{i=1}^m \chi_i^{b_i}
\end{equation}

Using logarithmic form, write $\chi_i = e^{2\pi i s_i}$ for $s_i \in \mathbb{R}/\mathbb{Z}$:
\begin{equation}
\chi(\mathbf{b}) = \exp\left( 2\pi i \sum_{i=1}^{m} s_i b_i \right)
\end{equation}

\paragraph{Lemma (Fourier Inversion on Character Groups).}

For any function $f : \mathcal{E}_m \to \mathbb{C}$ with compact support (only finitely many $\mathbf{b}$ satisfy $f(\mathbf{b}) \neq 0$), the Fourier transform:
\begin{equation}
\hat{f}(\chi) := \sum_{\mathbf{b} \in \mathcal{E}_m} f(\mathbf{b}) \chi(\mathbf{b})^{-1}
\end{equation}

satisfies the inversion formula:
\begin{equation}
f(\mathbf{b}) = \frac{1}{\text{Vol}(\widehat{\mathcal{E}}_m)} \int_{\widehat{\mathcal{E}}_m} \hat{f}(\chi) \chi(\mathbf{b}) \, d\mu(\chi)
\end{equation}

where $d\mu(\chi)$ is the Haar measure on the character group (normalized so $\text{Vol}(\widehat{\mathcal{E}}_m) = 1$).

\subsubsection{Deficit Characters and Validity Conditions}

\paragraph{Definition (Deficit Character).}

For each constraint valuation $v \in \mathcal{V}$, define the \textit{deficit character}:
\begin{equation}
\phi_v(\mathbf{b}) := \exp\left( 2\pi i \cdot \frac{\mu_v(\mathbf{b}) - \sum_k b_k \delta_{v,k}}{N_v} \right)
\end{equation}

where $N_v$ is a normalization constant (e.g., $N_v = \text{lcm}_k \delta_{v,k}$ or $N_v = 1 + \max_k \delta_{v,k}$).

The deficit character $\phi_v(\mathbf{b})$ is a unit-norm complex number that measures the \textit{phase mismatch} between numerator divisibility $\mu_v(\mathbf{b})$ and denominator obligation $\sum_k b_k \delta_{v,k}$.

\paragraph{Theorem 1 (Validity as Character Orthogonality).}

An exponent vector $\mathbf{b}$ is valid (Eq. \ref{eq:validity_abstract}) if and only if:
\begin{equation}
\label{eq:char_orthogonality}
\prod_{v \in \mathcal{V}} \phi_v(\mathbf{b}) = 1 \quad \text{(complete phase alignment)}
\end{equation}

Equivalently, the total phase is an integer multiple of $2\pi$:
\begin{equation}
\sum_{v \in \mathcal{V}} \arg(\phi_v(\mathbf{b})) \equiv 0 \pmod{2\pi}
\end{equation}

\begin{proof}
Validity requires $\mu_v(\mathbf{b}) \geq \sum_k b_k \delta_{v,k}$ for all $v$. Define:
\begin{equation}
r_v(\mathbf{b}) := \mu_v(\mathbf{b}) - \sum_k b_k \delta_{v,k} \geq 0
\end{equation}

Then:
\begin{equation}
\phi_v(\mathbf{b}) = \exp\left( 2\pi i \cdot \frac{r_v(\mathbf{b})}{N_v} \right)
\end{equation}

Since $r_v(\mathbf{b})$ is a non-negative integer, $\phi_v(\mathbf{b})$ is a root of unity (approximately, depending on $N_v$). The product:
\begin{equation}
\prod_{v \in \mathcal{V}} \phi_v(\mathbf{b}) = \exp\left( 2\pi i \sum_v \frac{r_v(\mathbf{b})}{N_v} \right)
\end{equation}

equals 1 if and only if $\sum_v \frac{r_v(\mathbf{b})}{N_v} \in \mathbb{Z}$.

For the standard choice $N_v = 1$, this requires $\sum_v r_v(\mathbf{b}) \in \mathbb{Z}$, which is automatic since each $r_v(\mathbf{b})$ is an integer. Thus validity is equivalent to the character product being trivial.

For general $N_v$, choose normalization such that validity enforces the harmonic condition exactly.
\end{proof}

\subsubsection{Fourier Analysis of Validity Indicator}

\paragraph{Definition (Validity Indicator Function).}

Define:
\begin{equation}
\mathbf{1}_{\mathcal{V}_{\text{valid}}}(\mathbf{b}) := \begin{cases} 1 & \text{if } \mathbf{b} \in \mathcal{V}_{\text{valid}} \\ 0 & \text{otherwise} \end{cases}
\end{equation}

This function encodes which exponent vectors satisfy all constraints simultaneously.

\paragraph{Definition (Fourier Transform of Validity).}

The Fourier transform of the validity indicator is:
\begin{equation}
\label{eq:fourier_validity_indicator}
\hat{\mathbf{1}}(\chi) := \sum_{\mathbf{b} \in \mathcal{E}_m} \mathbf{1}_{\mathcal{V}_{\text{valid}}}(\mathbf{b}) \cdot \chi(\mathbf{b})^{-1} = \sum_{\mathbf{b} \in \mathcal{V}_{\text{valid}}} \prod_{i=1}^m \chi_i^{-b_i}
\end{equation}

This is a generating function over valid vectors.

\paragraph{Theorem 2 (Factorization of Validity Fourier Transform).}

If the constraint structure admits a \textit{cascade decomposition}, meaning validity factorizes as:
\begin{equation}
\mathbf{b} \in \mathcal{V}_{\text{valid}} \iff \forall k : \mathbf{b}_{\leq k} \in C_k
\end{equation}

where $C_k \subseteq \mathbb{N}_0^k$ are constraint sets and $\mathbf{b}_{\leq k} := (b_1, \ldots, b_k)$, then:
\begin{equation}
\hat{\mathbf{1}}(\chi_1, \ldots, \chi_m) = \prod_{k=1}^m F_k(\chi_1, \ldots, \chi_k)
\end{equation}

where each factor $F_k$ encodes constraints at stage $k$.

\begin{proof}
By definition:
\begin{equation}
\hat{\mathbf{1}} = \sum_{\mathbf{b} : \forall k, \mathbf{b}_{\leq k} \in C_k} \prod_i \chi_i^{-b_i}
\end{equation}

Rewrite this as a telescoping product. For each position $k$, the sum over $b_k$ restricted by $C_k$ factors independently (given $b_1, \ldots, b_{k-1}$):
\begin{equation}
\hat{\mathbf{1}} = \prod_{k=1}^m \left( \sum_{(b_1, \ldots, b_k) \in C_k} \chi_k^{-b_k} \cdot (\text{prefix term}) \right)
\end{equation}

Arranging the product properly gives $F_k(\chi_1, \ldots, \chi_k)$.
\end{proof}

\textbf{Critical implication:} The factorization structure of $\hat{\mathbf{1}}(\chi)$ directly encodes the constraint cascade. Singularities (poles and zeros) of $F_k$ reveal which valuations are active at stage $k$. This provides a pathway to recover constraint structure without assuming primes.

\subsubsection{Meromorphic Extension and Spectral Properties}

\paragraph{Definition (Generating Function for Valid Vectors).}

Consider:
\begin{equation}
\label{eq:generating_function_validity}
Z(s) := \sum_{\mathbf{b} \in \mathcal{V}_{\text{valid}}} \exp\left( -s \sum_i b_i \right) = \sum_{\mathbf{b} \in \mathcal{V}_{\text{valid}}} e^{-s \|\mathbf{b}\|_1}
\end{equation}

where $\|\mathbf{b}\|_1 = \sum_i b_i$ is the exponent sum (related to the \textit{weighted magnitude} of the vector).

For $\text{Re}(s)$ sufficiently large, this series converges. For smaller $\text{Re}(s)$, it may diverge, but admits analytic continuation to a meromorphic function.

\paragraph{Theorem 3 (Singularity Structure of the Generating Function).}

For a constraint structure with deficits $\{\delta_{v,k}\}$ and corresponding singularity set $\mathcal{S}$ determined by the structure, the generating function $Z(s)$ admits a meromorphic continuation to $\mathbb{C}$ with:

\begin{enumerate}
\item A pole at $s = 0$ of order at most $m$ (from unconstrained exponential growth).

\item Poles at locations determined by the spectral structure of the constraint deficits, with pole locations encoding the divisibility structure of the deficit parameters.

\item The residue structure at each pole encodes the multiplicity and divisibility properties of the constraint system.

\end{enumerate}

\noindent\textbf{Consequence (Primes)}: When the constraint structure arises from the multiplicative structure of integers with basis primes $\{p_1, \ldots, p_m\}$ and deficits $\delta_{v,k} = v_q(p_k - 1)$ for primes $q$, the pole locations include $s = \frac{2\pi i n}{\log q}$ for each such prime $q$.

\begin{proof}[Sketch]
The generating function factors over the cascade structure:
\begin{equation}
Z(s) = \prod_{k=1}^m Z_k(s)
\end{equation}

where $Z_k(s)$ accounts for valid assignments of $b_k$ given prior exponents.

For unconstrained $b_k$, we have $Z_k(s) = \sum_{b_k \geq 0} e^{-s b_k} = \frac{1}{1 - e^{-s}}$. This has a pole at $s = 0$ (and periodically at $s = 2\pi i n$).

When constraints involve prime $q$, the sum over $b_k$ is restricted, creating a modified denominator. Specifically, if the constraint depends on $v_q(\text{something})$, then $e^{-s}$ is replaced by $e^{-s / \log q}$ in the denominator, shifting the pole location to $s = 2\pi i n / \log q$.

The product structure yields the stated singularity pattern.
\end{proof}

\subsubsection{Recovering Primes from Spectral Data}

\paragraph{Theorem 4 (Prime Recovery via Pole Analysis).}

Let $\mathcal{B} \subseteq \mathcal{E}_m$ be a finite set of observed valid vectors (e.g., exponent vectors of known integers).

Compute the empirical generating function:
\begin{equation}
Z_{\text{emp}}(s) := \sum_{\mathbf{b} \in \mathcal{B}} \exp\left( -s \sum_i b_i \right)
\end{equation}

Via analytic continuation, identify the pole set $\mathcal{P}_{\text{poles}} = \{s_j : s_j \text{ is a pole of } Z_{\text{emp}}\}$.

Then the set of \textit{candidate primes} is:
\begin{equation}
\mathcal{P}_{\text{cand}} := \left\{ q \in \mathbb{N} : \exists s_j \in \mathcal{P}_{\text{poles}}, n \in \mathbb{Z} \text{ such that } s_j = \frac{2\pi i n}{\log q} \right\}
\end{equation}

A candidate prime $q$ is \textit{valid} if the hypothesis that $q$ divides some $p_k - 1$ is consistent with all observed valid vectors.

\begin{proof}
The empirical generating function $Z_{\text{emp}}(s)$ is a finite sum, hence an entire function with exponential growth. However, if $\mathcal{B}$ is a representative sample of all valid vectors with $\sum_i b_i \leq N$, then $Z_{\text{emp}}$ approximates $Z(s)$ for $\text{Re}(s) > \text{const}/N$.

In this region, the poles of $Z(s)$ are visible as singularities of $Z_{\text{emp}}$ or as locations where $Z_{\text{emp}}$ has rapid growth.

By identifying pole locations $s_j$ and solving $s_j = 2\pi i n / \log q$, we recover $q$.
\end{proof}

\paragraph{Remark (Consistency Check).}

To validate recovered primes, one must verify that the inferred constraint structure explains all observed valid vectors. This closes the loop: from observations, we deduce structure; from structure, we predict observations. Prediction matching observation validates the deduced primes.

This consistency check transforms the algorithm from a speculative tool into a rigorous inference method.

\subsubsection{Self-Consistency and Closure}

\paragraph{Definition (Closed Constraint Structure).}

A constraint structure $(\mathcal{V}, \{\mu_v\}, \{\delta_{v,k}\})$ is \textit{closed} if:

For all $\mathbf{b}, \mathbf{b}' \in \mathcal{V}_{\text{valid}}$, we have $\mathbf{b} + \mathbf{b}' \in \mathcal{V}_{\text{valid}}$.

Closure means the valid vectors form a sub-monoid of $\mathcal{E}_m$.

\paragraph{Theorem 5 (Closure and Additivity of Valuations).}

Closure holds automatically if:

\begin{enumerate}
\item Each $\mu_v$ is additive (given).
\item The deficit structure $\{\delta_{v,k}\}$ is intrinsic to the indices $k$ (doesn't vary with other exponents).
\end{enumerate}

\begin{proof}
If $\mathbf{b}, \mathbf{b}' \in \mathcal{V}_{\text{valid}}$, then for all $v$:
\begin{equation}
\mu_v(\mathbf{b}) \geq \sum_k b_k \delta_{v,k}, \quad \mu_v(\mathbf{b}') \geq \sum_k b'_k \delta_{v,k}
\end{equation}

By additivity of $\mu_v$:
\begin{equation}
\mu_v(\mathbf{b} + \mathbf{b}') = \mu_v(\mathbf{b}) + \mu_v(\mathbf{b}') \geq \sum_k b_k \delta_{v,k} + \sum_k b'_k \delta_{v,k} = \sum_k (b_k + b'_k) \delta_{v,k}
\end{equation}

Thus $\mathbf{b} + \mathbf{b}' \in \mathcal{V}_{\text{valid}}$.
\end{proof}

\paragraph{Significance.}

Closure ensures that if two integers are expressible in the epimoric basis, their product is also expressible. This is a non-trivial structural property and reflects the multiplicative closure of integers themselves.

\subsubsection{Connection to Polytope Geometry}

\paragraph{Embedding into Standard Polytope.}

The abstract formulation connects to the obstruction polytope $\mathcal{P}_m$ (from prior sections) via the map:
\begin{equation}
\iota : \mathcal{V}_{\text{valid}} \to \mathcal{P}_m \cap \mathbb{N}_0^m
\end{equation}

given by identity. The constraint structure $(\mathcal{V}, \{\mu_v\}, \{\delta_{v,k}\})$ defines the polytope:
\begin{equation}
\mathcal{P}_m = \left\{ \mathbf{b} \in \mathbb{R}_{\geq 0}^m : \mu_v(\mathbf{b}) \geq \sum_k b_k \delta_{v,k} \quad \forall v \in \mathcal{V} \right\}
\end{equation}

The lattice points of $\mathcal{P}_m$ with integer coordinates form exactly $\mathcal{V}_{\text{valid}}$.

\paragraph{Character Group as Polytope Dual.}

The character group $\widehat{\mathcal{E}}_m \cong (\mathbb{C}^{\times})^m$ is isomorphic to the space of linear functionals on $\mathcal{E}_m$. The Fourier transform $\hat{\mathbf{1}}(\chi)$ is the characteristic function of the valid polytope, as a function on the dual space.

This duality is the foundation for singularity analysis and pole detection: singularities in the Fourier domain (character group) correspond to geometric features of the primal domain (polytope geometry).
