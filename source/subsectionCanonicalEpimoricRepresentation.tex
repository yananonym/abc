\section{Canonical Epimoric Representation: The $p_k/(p_k - 1)$ Multiplicative Basis}

The \textbf{canonical epimoric representation} is the ratio-based multiplicative basis defined by the set of all ratios:

\begin{equation}
\varepsilon_k = \frac{p_k}{p_k - 1}
\end{equation}

where $p_k$ is the $k$-th prime. This representation is called \textit{canonical} because it represents fundamental musical intervals in Just Intonation (the octave, fifth, and major third) and satisfies the closure properties required for multiplicative encoding.

\subsection{Definition and Motivation}

The canonical epimoric basis is the multiplicative basis most closely ``aligned'' with the sequence of primes themselves. Each ratio has:
\begin{itemize}
\item Numerator: the prime $p_k$
\item Denominator: $p_k - 1$, which is always strictly less than $p_k$
\end{itemize}

Every natural number $n > 1$ admits a \textbf{unique canonical epimoric factorization}:

\begin{equation}
n = \prod_{k=1}^{\infty} \left(\frac{p_k}{p_k - 1}\right)^{b_k}
\end{equation}

where $b_k \in \mathbb{N}_0$ and only finitely many exponents are nonzero.

The exponent vector in canonical epimoric form:

\begin{equation}
\mathbf{v}_n^{\text{can-epim}} = [b_1, b_2, b_3, \ldots]
\end{equation}

\subsection{Musical Significance}

The canonical epimoric notation was developed explicitly for musical purposes as an alternative to Monzo notation by Jan Machalski in their procedural audiovisual composition \textit{Epimoric Music} (2021). In this system:

\begin{itemize}
\item $\varepsilon_1 = 2/1$ represents the octave (perfect doubling)
\item $\varepsilon_2 = 3/2$ represents the perfect fifth
\item $\varepsilon_3 = 5/4$ represents the major third
\item $\varepsilon_4 = 7/6$ represents the minor third
\end{itemize}

These are the fundamental harmonic intervals that form the basis of Just Intonation and have been used in music for centuries.

\subsection{Canonical Epimoric vs Prime Representation}

Consider $n = 60 = 2^2 \cdot 3 \cdot 5$ in prime factorization (monzo $[2, 1, 1]_\text{prime}$).

In canonical epimoric form, we have:

\begin{align}
60 &= \left(\frac{2}{1}\right)^5 \cdot \left(\frac{3}{2}\right)^1 \cdot \left(\frac{5}{4}\right)^1\\
&= [5, 1, 1]_{\text{can-epim}}
\end{align}

The difference in exponent vectors arises from the denominator contributions: the denominators $1, 2, 4$ must be ``paid for'' by accumulating exponents in the numerators of earlier ratios.

\subsection{Conversion Formula}

The conversion from canonical epimoric exponents $[b_1, b_2, \ldots]$ to standard prime exponents $[a_1, a_2, \ldots]$ is:

\begin{equation}
a_k = b_k - b_{k-1}
\end{equation}

with $b_0 := 0$. Conversely, to go from prime exponents to epimoric:

\begin{equation}
b_k = \sum_{j=1}^{k} a_j
\end{equation}

This cumulative sum structure reflects the \textit{entanglement} of the bases in the epimoric system, in contrast to the independence of the prime basis.

\subsection{Why ``Canonical''?}

This representation is canonical because:

\begin{enumerate}
\item It has minimal denominator variation (each denominator is one less than the numerator)
\item It aligns with harmonic series structures in acoustics and music
\item The numerators are exactly the primes, creating a direct correspondence
\item It is the epimoric system most closely related to standard prime factorization
\end{enumerate}

In contrast, more general epimeric systems (like $(p_k + 2)/p_k$ or $(p_k + 3)/p_k$) are called \textit{non-canonical epimeric systems} and show greater structural complexity and less musical relevance.
