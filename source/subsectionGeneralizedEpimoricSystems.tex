\section{Generalized Epimeric Factorization Systems}

The canonical epimoric system (with displacement $q=1$, giving ratios $p_k/(p_k - 1)$) is only one member of a family of multiplicative bases. We now explore the general class of \textbf{epimeric systems} indexed by a displacement parameter $q$.

\subsection{Formal Definition of Epimeric Systems}

For a fixed positive integer $q$, the $q$-\textbf{epimeric multiplicative basis} is defined by the set of ratios:

\begin{equation}
R_q = \left\{ \frac{p_k + q}{p_k} : k = 1, 2, 3, \ldots \right\}
\end{equation}

Every natural number $n$ may be represented (uniquely up to rational reconstruction) as:

\begin{equation}
n = \prod_{k=1}^{\infty} \left(\frac{p_k + q}{p_k}\right)^{c_k}
\end{equation}

where $c_k \in \mathbb{Z}$ and only finitely many exponents are nonzero. When all $c_k \geq 0$, the representation is canonical.

\subsection{The (p+2)/p System: Twin-Prime Driven Structure}

For $q = 2$, we obtain the ratios:

\begin{equation}
R_2 = \left\{ \frac{4}{2}, \frac{5}{3}, \frac{7}{5}, \frac{9}{7}, \frac{13}{11}, \ldots \right\}
\end{equation}

This system, which is sometimes denoted the \textbf{epimeric system of degree 2}, exhibits deep structure related to the twin prime conjecture.

\subsubsection{Mechanics of Cancellation in the (p+2)/p System}

Unlike the canonical epimoric system, where denominators appear in numerators of earlier ratios, the $(p+2)/p$ system exhibits a different cancellation pattern:

\begin{itemize}
\item When $(p_k, p_k+2)$ are twin primes (e.g., 3 and 5, or 5 and 7), the denominator $p_k$ of one ratio appears as part of the numerator $p_{k'}+2$ of another ratio, enabling \textit{twin prime cancellation}
\item When $p_k + 2$ is composite, the denominator $p_k$ is cancelled by a combination of other ratios whose numerator factorizations include $p_k$
\item When a prime $p_k$ lacks the cancellation mechanism (e.g., isolated primes with no twin), the system requires negative exponents (``prime debt'') to balance the equation
\end{itemize}

\subsubsection{Representation Table for (p+2)/p (q=2)}

Consider the representations for small integers. For $n = 1$ to $10$:

\begin{center}
\small
\begin{tabular}{|c|c|c|}
\hline
$n$ & Vector $[c_1, c_2, c_3, \ldots]_{(p+2)/p}$ & Product \\
\hline
1 & $[0]$ & $1$ \\
2 & $[1]$ & $4/2$ \\
3 & $[0, 1, 1, 1]$ & $(5/3)(7/5)(9/7) = 3$ (Twin Prime Cascade) \\
4 & $[2]$ & $(4/2)^2$ \\
5 & $[0, 2, 1, 1]$ & $(5/3)^2(7/5)(9/7) = 5$ \\
6 & $[1, 1, 1, 1]$ & $(4/2)(5/3)(7/5)(9/7) = 6$ \\
7 & $[0, 2, 2, 1]$ & $(5/3)^2(7/5)^2(9/7) = 7$ \\
8 & $[3]$ & $(4/2)^3$ \\
9 & $[0, 2, 2, 2]$ & $(5/3)^2(7/5)^2(9/7)^2 = 9$ \\
10 & $[1, 2, 1, 1]$ & $(4/2)(5/3)^2(7/5)(9/7) = 10$ \\
\hline
\end{tabular}
\end{center}

The central observation: to represent the prime $3$, we must traverse a cascade of three twin-prime adjacent pairs $(2,4)$, $(3,5)$, $(5,7)$ because $3$ is in the numerator of $5/3$, and the denominator $3$ must be created from the structure of the system. This creates an inherent complexity in representing $3$ in this system.

\subsection{The (p+3)/p System: Parity Collapse Phenomena}

For $q = 3$, we obtain:

\begin{equation}
R_3 = \left\{ \frac{5}{2}, \frac{6}{3}, \frac{8}{5}, \frac{10}{7}, \frac{14}{11}, \ldots \right\}
\end{equation}

This system exhibits a \textbf{parity collapse} phenomenon: for all $p > 2$, the numerator $p + 3$ is even (since odd + odd = even). Consequently, the prime $2$ is over-supplied in the numerators of this system, while odd primes appear more sporadically.

\subsubsection{Representation Table for (p+3)/p (q=3)}

For small integers:

\begin{center}
\small
\begin{tabular}{|c|c|c|}
\hline
$n$ & Vector $[d_1, d_2, d_3, \ldots]_{(p+3)/p}$ & Notes \\
\hline
1 & $[0]$ & Identity \\
2 & $[0, 1]$ & $(6/3) = 2$ \\
3 & $[-1, -2, -1, \ldots, 2, \ldots]$ (complex) & 3 is hardest to represent (prime debt) \\
4 & $[0, 2]$ & $(6/3)^2$ \\
5 & $[1, 1]$ & $(5/2)(6/3) = 5$ \\
6 & \text{(requires vector with many entries)} & Requires 2 and 3 factorizations \\
\hline
\end{tabular}
\end{center}

Notably, $3$ becomes the hardest number to represent in the $(p+3)/p$ system, requiring extensive prime debt, because $3$ is the only prime that occurs as a denominator in $6/3 = 2$.

\subsection{General Structure: The (p+q)/p Family}

For arbitrary displacement $q$, the properties of the epimeric system $R_q$ depend critically on:

\begin{enumerate}
\item The distribution of values $\{p_k + q : k \in \mathbb{N}\}$ in the integer lattice
\item How these values factorize into primes (which may or may not include the primes themselves)
\item The structure of prime gaps relative to the displacement $q$
\item Whether $q$ is even or odd (parity properties)
\end{enumerate}

\subsection{Multiplicative Basis Completeness for Epimeric Systems}

Despite structural variations, all epimeric systems share a fundamental property:

\begin{theorem}[Epimeric Multiplicative Basis Completeness]
For any fixed positive integer displacement $q$, the set of ratios $R_q = \{(p_k + q)/p_k : k \geq 1\}$ generates a multiplicative basis for $\mathbb{Q}^+$. Consequently, every natural number $n$ has a unique representation (allowing negative exponents) as a finite product of these ratios with integer exponents.
\end{theorem}

This theorem, which can be proven by basis transformation arguments and the invertibility of the valuation matrix, guarantees that no matter which displacement $q$ we choose, we obtain a valid multiplicative coordinate system for the rational numbers.

\subsection{Comparative Complexity Across Systems}

The \textit{difficulty} of representing a given integer varies dramatically across epimeric systems:

\begin{itemize}
\item In $R_1$ (canonical epimoric), small integers like $2, 3, 5$ have short, simple vector representations
\item In $R_2$ (twin-prime system), the complexity is redistributed: some small integers become harder, reflecting twin prime structure
\item In $R_3$ (parity collapse system), odd numbers become harder, even numbers easier
\end{itemize}

Each system $R_q$ illuminates different structural aspects of the integers and their relationship to prime distribution.

\subsection{Connection to Multiplicative Bases Theory}

In abstract multiplicative basis theory, the epimeric systems form a \textbf{family of basis transformations}. Each basis $R_q$ is obtained from the prime basis $P$ via a linear transformation that depends on $q$. The existence of multiple valid bases reflects the fundamental observation that the multiplicative structure of integers can be coordinatized in many different ways.

The canonical epimoric system is \textit{canonical} not because it is unique, but because it is the most \textit{parsimonious} with respect to the primes themselves, aligning most closely with the Fundamental Theorem of Arithmetic while revealing hidden harmonic structure.
