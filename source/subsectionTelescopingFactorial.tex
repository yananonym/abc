\subsection{A telescoping representation of the factorial and implications for Wilson's theorem}

Consider the identity
\begin{equation}\label{eq:factorial-telescope}
n!
=\exp\!\left(\sum_{k=1}^{n-1} (n-k)\,\ln\!\left(\frac{k+1}{k}\right)\right).
\end{equation}
This formula is an exact consequence of telescoping and discrete summation by parts; no new structure is being introduced.

Indeed, write
\[
\ln\!\left(\frac{k+1}{k}\right)=\ln(k+1)-\ln k.
\]
Then
\[
\sum_{k=1}^{n-1} (n-k)\bigl(\ln(k+1)-\ln k\bigr)
\]
is a standard Abel summation (discrete integration by parts). Expanding yields
\[
\sum_{m=2}^{n} \ln m \sum_{k=1}^{m-1} 1
= \sum_{m=2}^{n} (m-1)\ln m,
\]
and a further telescoping step shows
\(
\sum_{m=2}^{n} (m-1)\ln m = \sum_{j=1}^{n} \ln j = \ln n!
\).
Thus \eqref{eq:factorial-telescope} is simply a reindexing of the usual definition of $n!$ expressed via a telescoping sum of logarithmic increments.

\subsubsection{Interpretation via discrete summation by parts}

Formula \eqref{eq:factorial-telescope} exhibits $\ln n!$ as the convolution of the sequence $\ln\!\left(\frac{k+1}{k}\right)$ with the linear weight $(n-k)$. This is the canonical outcome of Abel summation applied to the partial sums of $\ln k$. No nonstandard operator is involved: the factorial is reconstructed from first differences of $\ln k$ with respect to the forward difference operator.

Equivalently, $n!$ is obtained by exponentiating a weighted telescoping sum of successive ratios $(k+1)/k$. This makes explicit that the factorial is determined entirely by local multiplicative increments.

\subsubsection{Reduction modulo a prime and Wilson's theorem}

Let $p$ be prime. The telescoping product underlying \eqref{eq:factorial-telescope} remains valid in any commutative ring where the relevant inverses exist. In particular, in $\mathbb{F}_p^\times$ one has
\[
(p-1)! = \prod_{k=1}^{p-2} \left(\frac{k+1}{k}\right)^{p-1-k},
\]
a purely multiplicative telescoping identity.

Wilson's theorem,
\(
(p-1)! \equiv -1 \pmod p,
\)
is then understood as a statement about the structure of this telescoping product in the cyclic group $\mathbb{F}_p^\times$. The involution $x\mapsto x^{-1}$ pairs all non-self-inverse elements of the product, leading to complete cancellation. The unique self-inverse element is $-1$, which accounts for the residual factor.

From this perspective, Wilson's theorem reflects the fact that the telescoping structure underlying the factorial is compatible with inversion symmetry in $\mathbb{F}_p^\times$, with the exceptional contribution arising solely from the fixed point of that symmetry. No appeal to analytic notions is required; the phenomenon is entirely algebraic and follows from standard properties of telescoping products in finite groups.
