\section{Introduction}

\subsection{Framework and Scope}

The Fundamental Theorem of Arithmetic establishes that every integer greater than one factors uniquely as a product of prime powers. Taking the Fundamental Theorem as given, the manuscript establishes that primes manifest simultaneously as singularities in three mathematically independent frameworks: algebraic-coherence theory (via character theory), spectral analysis (via transfer operators), and symbolic dynamics.

The cascade constraint structure, derived from Wilson's theorem and the multiplicative closure of integers, encodes the complete multiplicative structure of integers. Primes manifest as singularities in three distinct mathematical frameworks, each employing independent methodologies:

\begin{enumerate}

\item \textbf{Algebraic-Coherence Framework (via Character Theory)}: The exponent vectors corresponding to valid integers form a multiplicative structure over the character group $\hat{\mathbb{Z}}^m_{\text{exponents}}$. Primes correspond to exponent vectors exhibiting maximal coherence with unique character functionals.

\item \textbf{Spectral Framework (via Perron-Frobenius Theory)}: The transfer operator on valid exponent vectors admits spectral decomposition. Primes correspond to critical points (jump discontinuities in the derivative) of the Perron-Frobenius eigenvalue function $\lambda(s)$.

\item \textbf{Dynamical Framework (via Symbolic Dynamics)}: Valid exponent vectors form a subshift of finite type under the cascade constraints. Primes correspond to points where the topological entropy exhibits singularities (non-differentiability).

\end{enumerate}

These three frameworks are mathematically independent in the sense that each uses distinct mathematical machinery: group-theoretic character theory, spectral theory of non-negative operators, and symbolic dynamics. Yet they characterize the same set of objects (primes) via different properties. The central result establishes that these three characterizations are equivalent: an integer is prime if and only if all three properties hold simultaneously.

\subsection{Main Contributions}

The primary mathematical contributions are the multiplicative closure theorem and the three-fold equivalence. Multiplicative closure uniquely forces the normalization constants to be the primes minus one. Quantum coherence, spectral poles, and topological entropy discontinuities all characterize exactly the set of primes with equivalence rigorously established.

The cascade constraint structure derives from Wilson's theorem. Computational verification for integers one through one hundred validates the theoretical results.
