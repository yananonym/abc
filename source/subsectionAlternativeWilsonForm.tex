\subsection{Alternative Form of Wilson's Theorem Modulo One: Computational and Structural Insights}

This section develops a novel formulation of Wilson's theorem that emphasizes fractional parts and provides both computational advantages and deeper structural insights into the epimoric cascade.

\subsubsection{The Modulo-One Reformulation}

\begin{theorem}[Wilson's Theorem Modulo One]
For any prime $P$:
\begin{equation}
\frac{(P-1)!}{P} \equiv \frac{P-1}{P} \pmod{1}
\end{equation}

That is, the fractional parts of $\frac{(P-1)!}{P}$ and $\frac{P-1}{P}$ are equal.
\end{theorem}

\begin{proof}
From Wilson's theorem, $(P-1)! \equiv -1 \pmod{P}$, which gives $(P-1)! = kP - 1$ for some integer $k \geq 1$.

Thus:
\begin{equation}
\frac{(P-1)!}{P} = \frac{kP - 1}{P} = k - \frac{1}{P}
\end{equation}

The fractional part is:
\begin{equation}
\left\{\frac{(P-1)!}{P}\right\} = 1 - \frac{1}{P} = \frac{P-1}{P}
\end{equation}

where $\{x\} = x - \lfloor x \rfloor$ denotes the fractional part of $x$. Since $0 < \frac{P-1}{P} < 1$, we have:
\begin{equation}
\left\{\frac{P-1}{P}\right\} = \frac{P-1}{P}
\end{equation}

Therefore, the fractional parts match. $\square$
\end{proof}

\subsubsection{Why This Formulation Matters}

The modulo-one form has several advantages over the standard formulation:

\textbf{1. Avoids Factorial Computation:} Computing $(P-1)!$ directly is infeasible for large $P$. The modulo-one form requires only the computation of the fractional part of $(P-1)!/P$, which can be done via:
\begin{equation}
\frac{(P-1)!}{P} = \frac{1 \cdot 2 \cdot 3 \cdots (P-1)}{P} = \prod_{k=1}^{P-1} \frac{k}{P}
\end{equation}

Using logarithmic accumulation:
\begin{equation}
\log\left(\frac{(P-1)!}{P}\right) = \sum_{k=1}^{P-1} \log(k) - \log(P)
\end{equation}

This is computable in $O(P \log P)$ time using Stirling's approximation or specialized algorithms.

\textbf{2. Direct Link to Epimoric Structure:} The factor $\frac{P-1}{P}$ is precisely the denominator-epimoric generator. The congruence directly connects factorial growth to the epimoric basis structure.

\textbf{3. Reveals Continuity in Cascade:} For an exponent vector $\mathbf{b}$ in the epimoric system, the modulo-one form reveals how each contribution $b_k \cdot \frac{p_k - 1}{p_k}$ contributes fractional parts that must cohere across the cascade.

\subsubsection{Computational Optimization: Wilson Coherence Checking}

Given an exponent vector $\mathbf{b}$, we can verify the cascade constraint using the modulo-one form without explicit factorization.

\begin{algorithm}
\caption{Efficient Wilson Coherence Verification}
\begin{algorithmic}
\FUNCTION{VerifyCoherence}{$\mathbf{b}, \{\mathcal{P}\}$}
    \STATE $\text{FracPart} \gets 0$ \quad \COMMENT{Accumulated fractional part}
    \FOR{$k = 1$ to $m$}
        \STATE Compute $\frac{(p_k-1)!}{p_k}$ using logarithmic accumulation
        \STATE $f_k \gets \left\{\frac{(p_k-1)!}{p_k}\right\}$
        \STATE $\text{FracPart} \gets \text{FracPart} + b_k \cdot (1 - f_k)$ \quad \COMMENT{Denominator contribution}
        \IF{$\lfloor \text{FracPart} \rfloor > 0$}
            \STATE Adjust carry to next position
        \ENDIF
    \ENDFOR
    \RETURN $\text{FracPart} < 1$ \quad \COMMENT{Coherent if no unabsorbed fractional part}
\ENDFUNCTION
\end{algorithmic}
\end{algorithm}

This approach avoids full factorial computation and instead works with fractional parts, which remain bounded.

\subsubsection{Extension to Gamma Function and Analytic Continuation}

For non-integer arguments, the modulo-one form extends via the gamma function:

\begin{equation}
\frac{\Gamma(P)}{P} \equiv \frac{P-1}{P} \pmod{1}
\end{equation}

For real $P > 1$, the logarithmic derivative is:
\begin{equation}
\frac{d}{dP} \log \Gamma(P) = \psi(P) \quad \text{(digamma function)}
\end{equation}

The modulo-one form extends continuously, and the cascade constraint becomes a differential condition on the logarithm of the exponent vector.

\subsubsection{Example: Explicit Verification}

Consider $P = 5$:
\begin{align}
(5-1)! &= 4! = 24 \\
\frac{24}{5} &= 4.8 \\
\left\{\frac{24}{5}\right\} &= 0.8 \\
\frac{5-1}{5} &= \frac{4}{5} = 0.8
\end{align}

For $P = 11$:
\begin{align}
(11-1)! &= 10! = 3,628,800 \\
\frac{3,628,800}{11} &= 329,890.909\ldots \\
\left\{\frac{3,628,800}{11}\right\} &\approx 0.909\ldots \\
\frac{11-1}{11} &= \frac{10}{11} \approx 0.909\ldots
\end{align}

These verify the modulo-one relationship.

\subsubsection{Spectral Interpretation: Frobenius Density}

The modulo-one form admits a spectral interpretation via the Frobenius density. For a prime $P$, the densities of residues modulo $P$ in the factorial $\{1, 2, \ldots, P-1\}$ follow a specific pattern (uniform distribution by Dirichlet).

The fractional part $\frac{(P-1)!}{P}$ encodes how the product of all non-zero residues distributes around the prime: it measures the ``total mass'' of the factorial residue normalized by the prime.

\subsubsection{Connection to Epimoric Cascade Deficit}

Recall the cascade deficit function:
\begin{equation}
D_k(\mathbf{b}_{<k}) = \sum_{j=1}^{k-1} b_j \cdot v_{p_k}(p_j - 1)
\end{equation}

This can be rewritten in terms of modulo-one forms. For each $b_j$, the factor $\frac{p_j}{p_j - 1}$ contributes:
\begin{equation}
\left(\frac{p_j}{p_j - 1}\right)^{b_j} = \left(1 + \frac{1}{p_j - 1}\right)^{b_j}
\end{equation}

The numerator factors from this expansion contribute to prime-$p_k$ valuations according to the binomial expansion. The modulo-one form encodes this via fractional parts:

\begin{equation}
\text{Fractional carry from } p_j \text{ to } p_k \propto b_j \cdot \left(1 - \frac{(p_j-1)!}{p_j} \mod 1\right)
\end{equation}

\subsubsection{Analytic Number Theory Implications}

The modulo-one form connects to the Poisson-Summation formula and analytic number theory. For large $P$, Stirling's approximation gives:

\begin{equation}
\log\left(\frac{(P-1)!}{P}\right) \approx (P-1)\log(P-1) - (P-1) - \log(P)
\end{equation}

The fractional part exhibits oscillatory behavior related to the distribution of primes. Summing over all primes $p_k \leq N$:

\begin{equation}
\sum_{p \leq N} b_p \left(1 - \left\{\frac{(p-1)!}{p}\right\}\right) \sim O(\pi(N) \log N)
\end{equation}

This sum bounds the total cascade deficit, providing quantitative control over the constraint polytope volume.
