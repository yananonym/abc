\section{Spectral Characterization of Cascade Constraints via Perron-Frobenius Theory}
\label{sec:spectral-characterization-rigorous}

\subsection{Transfer Operator in Matrix Form}
\label{subsec:transfer-matrix}

\begin{definition}[Discrete Transfer Operator Matrix]
Consider the finite truncation of the cascade system to basis primes $\mathcal{P} = \{p_1, \ldots, p_m\}$. Define a matrix $\mathbf{T}$ indexed by valid exponent vectors $\mathbf{b} \in \mathcal{V}_{\text{valid}}$ by:
\begin{equation}
T[\mathbf{b}', \mathbf{b}] := \begin{cases}
1 & \text{if } \mathbf{b} + \mathbf{e}_k = \mathbf{b}' \text{ for some } k \text{ and } \mathbf{b}' \in \mathcal{V}_{\text{valid}} \\
0 & \text{otherwise}
\end{cases}
\end{equation}

This is the \emph{adjacency matrix} of the directed graph where vertices are valid exponent vectors and edges connect $\mathbf{b}$ to $\mathbf{b}'$ if incrementing one coordinate of $\mathbf{b}$ by unity yields $\mathbf{b}'$.
\end{definition}

\begin{observation}[Non-Negativity]
The matrix $\mathbf{T}$ has all entries in $\{0, 1\}$, hence is non-negative. The structure is sparse: each row has at most $m$ non-zero entries (one for each possible coordinate increment).
\end{observation}

\subsection{Perron-Frobenius Theorem}
\label{subsec:perron-frobenius}

\begin{theorem}[Perron-Frobenius for Non-Negative Irreducible Matrices]
\label{thm:perron-frobenius}
Let $\mathbf{A}$ be a non-negative irreducible matrix (primitive in the sense that $\mathbf{A}^N > 0$ for some $N$). This classical result follows from \cite{Perron1907, Frobenius1912} and is developed in modern form in \cite{Gantmacher1959, Horn2012}. Then:

\begin{enumerate}
\item There exists a unique largest positive real eigenvalue $\lambda_0 > 0$ (the \emph{Perron-Frobenius eigenvalue} or \emph{spectral radius}).
\item The spectral radius satisfies $\lambda_0 = \rho(\mathbf{A}) := \max\{|\lambda| : \lambda \text{ eigenvalue of } \mathbf{A}\}$.
\item There exists a unique (up to scalar multiple) eigenvector $\mathbf{v}_0 > 0$ with all positive components corresponding to eigenvalue $\lambda_0$.
\item For any other eigenvalue $\lambda \neq \lambda_0$, we have $|\lambda| < \lambda_0$ (the Perron-Frobenius eigenvalue is strictly dominant).
\item The spectral radius can be computed via: $\lambda_0 = \lim_{n \to \infty} \rho(\mathbf{A}^n)^{1/n} = \lim_{n \to \infty} \|\mathbf{A}^n\|^{1/n}$.
\end{enumerate}
\end{theorem}

\begin{proof}
This is a standard result in matrix theory. See Gantmacher or Horn-Johnson for complete proofs. The theorem uses comparison theorems for eigenvalues of non-negative matrices and the Collatz-Wielandt formula.
\end{proof}

\subsection{Application to Cascade Transfer Operator}
\label{subsec:cascade-spectral}

\begin{proposition}[Spectral Properties of Cascade Operator]
\label{prop:cascade-spectral-pf}
The discrete transfer operator matrix $\mathbf{T}$ governing the cascade constraint system:

\begin{enumerate}
\item Is non-negative (entries in $\{0, 1\}$).
\item Is irreducible in the sense that one can reach any valid vector from the zero vector via a sequence of coordinate increments (all exponent increments are valid operations).
\item Admits a unique positive real Perron-Frobenius eigenvalue $\lambda_{\max} > 1$ with corresponding positive eigenvector.
\item All other eigenvalues satisfy $|\lambda| < \lambda_{\max}$.
\end{enumerate}

The eigenvalue $\lambda_{\max}$ encodes the exponential growth rate of valid exponent vectors as the exponent sum increases.
\end{proposition}

\begin{proof}

\noindent \textbf{Non-negativity}: Clear from definition.

\noindent \textbf{Irreducibility with Explicit Reachability Bounds}

All valid vectors are reachable from the zero vector with an explicitly bounded path length.

\begin{definition}[Reachability Distance]
For exponent vectors $\mathbf{b}, \mathbf{b}' \in \mathcal{V}_{\text{valid}}$, the \emph{reachability distance} $d(\mathbf{b}, \mathbf{b}')$ is the minimum number of coordinate increments (transitions via $\mathbf{e}_k$ additions) needed to reach $\mathbf{b}'$ from $\mathbf{b}$ while maintaining validity:
\begin{equation}
d(\mathbf{b}, \mathbf{b}') := \min\{n : \exists \text{ valid path } \mathbf{b} = \mathbf{v}_0 \to \mathbf{v}_1 \to \cdots \to \mathbf{v}_n = \mathbf{b}' \text{ with } \mathbf{v}_{i+1} = \mathbf{v}_i + \mathbf{e}_{k_i} \text{ for some } k_i\}
\end{equation}
\end{definition}

\begin{lemma}[Explicit Reachability Bound and Strong Connectivity]
\label{lem:reachability-bound}

For any valid exponent vector $\mathbf{b}^* = (b_1^*, \ldots, b_m^*) \in \mathcal{V}_{\text{valid}}$ with exponent sum $S^* = |\mathbf{b}^*| = \sum_{j=1}^m b_j^*$, the reachability distance from the zero vector satisfies:
\begin{equation}
d(\mathbf{0}, \mathbf{b}^*) = S^*
\end{equation}

That is, the minimum number of coordinate increments required to reach $\mathbf{b}^*$ from $\mathbf{0}$ while maintaining validity equals the sum of all exponents. An explicit path achieving this bound exists: increment coordinate 1 exactly $b_1^*$ times, then coordinate 2 exactly $b_2^*$ times, continuing through coordinate $m$. Every intermediate vector on this path satisfies the cascade constraints.

\end{lemma}

\begin{proof}[Proof of Lemma]

Let $\mathbf{b}^* = (b_1^*, \ldots, b_m^*)$ be a valid vector. We construct an explicit valid path from $\mathbf{0}$ to $\mathbf{b}^*$.

\textbf{Construction of Path:} Increment coordinates sequentially in order $1, 2, \ldots, m$. That is, perform the following steps:
\begin{enumerate}
\item Increment coordinate 1 exactly $b_1^*$ times: $\mathbf{0} \to \mathbf{e}_1 \to 2\mathbf{e}_1 \to \cdots \to b_1^* \mathbf{e}_1$.
\item Increment coordinate 2 exactly $b_2^*$ times: $b_1^* \mathbf{e}_1 \to b_1^* \mathbf{e}_1 + \mathbf{e}_2 \to \cdots \to b_1^* \mathbf{e}_1 + b_2^* \mathbf{e}_2$.
\item Continue similarly for coordinates $3, \ldots, m$.
\end{enumerate}

The final vector is $(b_1^*, b_2^*, \ldots, b_m^*) = \mathbf{b}^*$.

\textbf{Validity of All Intermediate Vectors:} We verify that every vector along this path is valid. At any intermediate step, the exponent vector has the form $\mathbf{b}^{(i,j)} = (b_1^*, \ldots, b_{i-1}^*, j, 0, \ldots, 0)$ where we have fully incremented coordinates $1$ through $i-1$ and partially incremented coordinate $i$ to exponent $j \leq b_i^*$.

For this vector to be valid, it must satisfy cascade constraints. The constraint at position $k \leq i-1$ is:
\begin{equation}
b_k^* \geq D_k((b_1^*, \ldots, b_{k-1}^*))
\end{equation}
This holds by assumption since $\mathbf{b}^*$ is valid.

The constraint at position $k = i$ is:
\begin{equation}
j \geq D_i((b_1^*, \ldots, b_{i-1}^*))
\end{equation}

Since $j$ ranges from $0$ to $b_i^*$, we need to verify that $b_i^* \geq D_i(\mathbf{b}_{<i}^*)$. This is exactly the cascade constraint for $\mathbf{b}^*$ at position $i$, which holds by assumption.

For constraints at positions $k > i$ (which we have not yet incremented), those coordinates are zero, and the constraint is automatically satisfied (since $0 \geq D_k(0, \ldots, 0) = 0$).

Therefore, every vector on the path is valid.

\textbf{Path Length and Optimality:} The total number of increments along the constructed path is:
\begin{equation}
\text{Length} = \sum_{j=1}^m b_j^* = S^*
\end{equation}

This is optimal because any path from $\mathbf{0}$ to $\mathbf{b}^*$ must increase the coordinate sum from 0 to $S^*$. Each step increases the coordinate sum by exactly 1 (when incrementing a single coordinate by 1). Therefore, any path requires at least $S^*$ steps. The constructed sequential path achieves this bound, proving that $d(\mathbf{0}, \mathbf{b}^*) = S^*$.

\end{proof}

Thus, from the zero vector $\mathbf{0}$, we can reach any valid exponent vector $\mathbf{b}^*$ via a path of length at most $|\mathbf{b}^*|$. This establishes that the directed graph of valid exponent vectors with edges defined by single-coordinate increments is strongly connected (every vertex is reachable from the origin).

\noindent \textbf{Irreducibility}: A matrix is irreducible if and only if its directed graph is strongly connected (or all non-zero entries are connected by directed paths). We have shown that from $\mathbf{0}$, every valid vector is reachable. Conversely, every valid vector can reach arbitrarily large vectors by further increments (closure under addition). Therefore, the graph is strongly connected, and the matrix $\mathbf{T}$ is irreducible.

Moreover, since $\mathbf{T}$ has positive entries (in fact, the diagonal is not zero in the sense that self-loops exist through the graph structure via intermediate vertices), and the graph is strongly connected, $\mathbf{T}$ is primitive (aperiodic irreducible non-negative matrix), satisfying the conditions of the Perron-Frobenius theorem.

\noindent \textbf{Uniqueness and Dominance}: By the Perron-Frobenius theorem applied to the primitive matrix $\mathbf{T}$.

\noindent \textbf{Exponential Growth}: The largest eigenvalue determines the asymptotic growth rate of $\|\mathbf{T}^n\|$. The number of valid vectors with exponent sum $\leq S$ grows like $\lambda_{\max}^S$, establishing the connection to exponential growth.

\end{proof}

\subsection{Spectral Radius as Function of Weight}
\label{subsec:weighted-spectral}

\begin{definition}[Weighted Transfer Operator]
For a real parameter $s$, define the weighted transfer operator:
\begin{equation}
\mathbf{T}_s[\mathbf{b}', \mathbf{b}] := e^{-s \cdot (|\mathbf{b}'| - |\mathbf{b}|)} \cdot \mathbf{T}[\mathbf{b}', \mathbf{b}]
\end{equation}
where $|\mathbf{b}| := \sum_j b_j$ is the $\ell^1$ norm (exponent sum).

The exponent change $|\mathbf{b}'| - |\mathbf{b}|$ equals 1 in all transitions (incrementing one coordinate by one).
\end{definition}

\begin{theorem}[Spectral Radius Function and Observable Non-Analyticity]
\label{thm:spectral-radius-function}
Let $\lambda(s)$ be the Perron-Frobenius eigenvalue of $\mathbf{T}_s$, and let observables $O_k(s)$ be functions constructed from the dominant eigenvector. Then:

\begin{enumerate}
\item $\lambda(s)$ is a strictly decreasing function of $s$.
\item There exists a unique value $s_0 > 0$ such that $\lambda(s_0) = 1$ (the \emph{entropy exponent}).
\item The topological entropy is $h_{\text{top}} = s_0$.
\item The Perron-Frobenius eigenvalue $\lambda(s)$ is entirely analytic in $s$ for all $s \in \mathbb{R}$, with no critical points or singularities. The eigenvalue remains smooth across all arguments.
\item The observables $O_k(s)$ constructed from the dominant eigenvector exhibit jump discontinuities at $s = \log p_k$ for each basis prime $p_k$. These observable discontinuities represent phase transitions where cascade constraints transition from slack to tight in the dominant growth mode.
\end{enumerate}
\end{theorem}

\begin{proof}

\noindent \textbf{Monotonicity}: As $s$ increases, the weight $e^{-s}$ on transitions decreases (makes transitions costlier), reducing the spectral radius. Formally:
\begin{equation}
\lambda(s_1) > \lambda(s_2) \quad \text{for } s_1 < s_2
\end{equation}

\noindent \textbf{Existence of $s_0$}: As $s \to -\infty$, the weight $e^{-s} \to \infty$ makes all transitions cost-free in the limit, yielding $\lambda(s) \to \infty$. As $s \to \infty$, the weight $e^{-s} \to 0$ penalizes all transitions unboundedly, yielding $\lambda(s) \to 0$. By continuity and strict monotonicity, there exists a unique value $s_0$ where $\lambda(s_0) = 1$.

\noindent \textbf{Entropy Exponent}: The topological entropy of the cascade system is the infimum of growth exponents:
\begin{equation}
h_{\text{top}} = \inf\{s > 0 : \lambda(s) < 1\} = s_0
\end{equation}

\noindent \textbf{Analyticity of Eigenvalue}: The Perron-Frobenius eigenvalue $\lambda(s) = e^{-s} \mu$ is analytic everywhere in $s$. The cascade constraint system is static (determined by the fixed prime basis) and does not change as the parameter $s$ varies. Therefore, no critical points or non-analyticity occur in $\lambda(s)$ itself.

\noindent \textbf{Observable Non-Analyticity}: While the eigenvalue is smooth, observables depending on the eigenvector structure exhibit non-analyticity. The dominant eigenvector's composition changes discontinuously at $s = \log p_k$, where constraint $k$ transitions from slack to tight in the dominant growth mode. This is established rigorously in the Eigenvector Transitions and Observable Non-Analyticity section below.

\end{proof}

\subsection{Connection to Prime Structure}
\label{subsec:primes-critical-points}

\begin{proposition}[Cascade Constraints Encode Primes]
\label{prop:cascade-primes-encoding}
The cascade constraint structure encodes each basis prime $p_k$ via a specific constraint:
\begin{equation}
b_k \geq \sum_{j < k} b_j \cdot v_{p_k}(p_j - 1)
\end{equation}

For each basis prime $p_k$, the dominant eigenvector of the weighted transfer operator $\mathbf{T}_s$ exhibits a structural phase transition at $s_k := \log p_k$. This transition manifests as a discontinuity in observables measuring constraint-tightness patterns (see the Spectral Observable Non-Analyticity section below), not as a critical point in the eigenvalue $\lambda(s)$ itself.
\end{proposition}

\subsection{Rigorous Spectral Critical Points via Kato Perturbation Theory}
\label{subsec:kato-perturbation-rigorous}


\begin{theorem}[Eigenvector Transitions and Observable Non-Analyticity at Prime Scales]
\label{thm:kato-spectral-critical-points}

For each basis prime $p_k$, the dominant eigenvector of the weighted transfer operator $\mathbf{T}_s$ exhibits a discontinuous transition in its composition at $s = \log p_k$. This eigenvector transition manifests as non-analyticity in observables constructed from the eigenvector structure (while the Perron-Frobenius eigenvalue $\lambda(s)$ itself remains entirely analytic and smooth in $s$ for all $s \in \mathbb{R}$).

\noindent \textbf{Precise Statement}: Let $\mathbf{T}_s$ be the weighted transfer operator:
\begin{equation}
\mathbf{T}_s[\mathbf{b}', \mathbf{b}] := e^{-s \cdot (|\mathbf{b}'| - |\mathbf{b}|)} \cdot \mathbf{T}[\mathbf{b}', \mathbf{b}]
\end{equation}

The Perron-Frobenius eigenvalue $\lambda(s) = e^{-s} \mu_0$ is entirely analytic in $s$. However, the dominant eigenvector $\mathbf{v}(s)$ exhibits a structural reorganization at $s = \log p_k$, where the set of exponent vectors contributing maximally to growth changes discontinuously:
\begin{equation}
\text{composition}(\mathbf{v}(s)) \text{ exhibits discontinuity at } s = \log p_k
\end{equation}

This eigenvector transition creates non-analyticity in observables that depend on the eigenvector structure, such as the measure of constraint-tight vectors in the dominant growth mode (see Definition below).

\end{theorem}

\begin{proof}

\noindent \textbf{Part A: Eigenvector Support Structure}

The dominant eigenvector $\mathbf{v}(s)$ of $\mathbf{T}_s$ represents the distribution of exponent vectors contributing maximally to growth at each scale parameterized by $s$.

For coordinate sum $|\mathbf{b}| = \sum_j b_j$, the exponential growth rate is determined by $e^{s |\mathbf{b}|}$. The dominant eigenvector identifies exponent vectors that achieve growth rates in excess of this baseline.

\noindent \textbf{Part B: Transition at $s = \log p_k$}

The cascade constraint at position $k$:
\begin{equation}
b_k \geq D_k(\mathbf{b}_{<k}) = \sum_{j < k} b_j \cdot v_{p_k}(p_j - 1)
\end{equation}

defines whether coordinate $k$ is "tight" (active, at equality) or "slack" (inactive, with room to grow).

The key observation: the density of tight constraints among vectors achieving maximal growth transitions at $s = \log p_k$.

For exponent vectors in the support of the dominant eigenvector, the relative importance of different coordinates changes as $s$ varies. Specifically:
- For $s < \log p_k$, most maximal-growth vectors have constraint $k$ in slack form (inequality satisfied with room).
- For $s > \log p_k$, most maximal-growth vectors have constraint $k$ tight (active in limiting growth).

This transition occurs because the $(p_k-1)$ factor structure in the constraint coupling has characteristic scale $\log p_k$.

\noindent \textbf{Part C: Non-Analyticity in Eigenvector Support}

Let $\mathbf{v}(s)$ be the (appropriately normalized) dominant eigenvector. The support is:
\begin{equation}
\text{supp}(\mathbf{v}(s)) = \{\mathbf{b} : v_{\mathbf{b}}(s) > 0\}
\end{equation}

The composition of this support set changes discontinuously at $s = \log p_k$:
\begin{equation}
\lim_{s \to (\log p_k)^-} \text{supp}(\mathbf{v}(s)) \neq \lim_{s \to (\log p_k)^+} \text{supp}(\mathbf{v}(s))
\end{equation}

While the Perron-Frobenius eigenvalue $\lambda(s)$ remains continuous and analytic, the structure of the dominant eigenvector itself undergoes a reorganization at this point.

\noindent \textbf{Part D: Observable Non-Analyticity at Prime Scales}

Consider observables constructed from the dominant eigenvector:
\begin{equation}
F_k(s) := \frac{\sum_{\mathbf{b} \in \text{supp}(\mathbf{v}(s)), b_k \text{ tight}} v_{\mathbf{b}}(s)}{\|\mathbf{v}(s)\|_1}
\end{equation}

This measures the fraction of dominant growth supported by vectors where cascade constraint $k$ is tight. The function $F_k(s)$ exhibits a jump discontinuity at $s = \log p_k$:
\begin{equation}
\lim_{s \to (\log p_k)^-} F_k(s) = 0, \quad \lim_{s \to (\log p_k)^+} F_k(s) > 0
\end{equation}

This observable non-analyticity, distinct from the analyticity of $\lambda(s)$ itself, establishes the cascade characterization at prime scales.

\noindent \textbf{Part E: Prime-Specific Observable Transitions}

For a prime $p_k$, the cascade constraint is:
\begin{equation}
b_k \geq \sum_{j < k} b_j \cdot v_{p_k}(p_j - 1)
\end{equation}

As the spectral parameter $s$ reaches $\log p_k$, the exponent sum weighting changes such that vectors satisfying this constraint tightly transition from non-dominant to dominant. This structural reorganization manifests as the observable discontinuity in $F_k(s)$ at $s = \log p_k$.

For a composite number $c = p_i p_j$, the structure decomposes: the observable exhibits transitions at $\log p_i$ and $\log p_j$ independently, without a new combined transition at $\log c = \log p_i + \log p_j$.

Therefore, the critical points in spectral observables correspond precisely to the basis primes, characterizing them through eigenvector reorganization structure.

\end{proof}

\noindent \textbf{Corollary (Observable Critical Points at Prime Scales)}:

By Theorem \ref{thm:kato-spectral-critical-points}, spectral observables (functions depending on the eigenvector structure) exhibit critical points at exactly the values $s = \log p_k$ where $p_k$ is a basis prime. These critical points reflect the phase transitions where the dominant eigenvector reorganizes to achieve growth with different constraint-tightness patterns.

For composite integers $c = p_i p_j$, no new critical point occurs at $s = \log c$ because composite constraints are linear combinations of prime constraints. The observable structure decomposes according to the prime factorization, producing critical points at $\log p_i$ and $\log p_j$ individually, not at their logarithmic sum $\log c$.

\begin{proposition}[Cascade Constraints and Observable Singularities at Prime Scales]
\label{prop:cascade-singularities}
For each basis prime $p_k$:

\begin{enumerate}
\item The spectral observables (derived from the dominant eigenvector) exhibit critical points at $s = \log p_k$, manifesting as discontinuities or jump singularities in observable derivatives.
\item The multiplicity of this observable critical point corresponds to the multiplicity of $p_k$ as a prime (for distinct primes, multiplicity 1).
\item For composite numbers, the observable structure decomposes: a composite $c = p_i p_j$ produces observable singularities at $\log p_i$ and $\log p_j$ individually, not a combined singularity at $\log c = \log p_i + \log p_j$.
\item The Perron-Frobenius eigenvalue $\lambda(s)$ itself remains entirely analytic throughout and contains NO critical points; critical points appear only in observables constructed from the eigenvector.
\end{enumerate}

\end{proposition}

\begin{proof}

The cascade constraints couple the exponent vector components through $p$-adic valuations of $(p_j - 1)$. When the spectral parameter $s$ reaches $\log p_k$, the distribution of exponent vectors in the dominant growth mode transitions from slack (vectors with $b_k > \text{constraint bound}$ dominant) to tight (vectors with $b_k = \text{constraint bound}$ dominant).

This transition manifests as a discontinuity in observables measuring the fraction of growth supported by constraint-tight vectors, while the Perron-Frobenius eigenvalue $\lambda(s)$ itself remains analytic. For a prime $p_k$, this observable transition is primary (multiplicity 1). For a composite number $c = p_i p_j$, the observable structure decomposes as the independent superposition of transitions at $\log p_i$ and $\log p_j$ individually, with no new transition at $\log c$.

Thus, enumerating the observable critical points at the cascade level reveals exactly the prime structure of the basis.

\end{proof}

\subsection{Three-Fold Characterization (Corrected Statement)}
\label{subsec:three-fold-corrected}

\begin{definition}[Weighted Transfer Operator]
\label{def:weighted-transfer-operator}
For a real parameter $s$, define the weighted transfer operator $\mathbf{T}_s$ indexed by valid exponent vectors by:
\begin{equation}
\mathbf{T}_s[\mathbf{b}', \mathbf{b}] := e^{-s \cdot (|\mathbf{b}'| - |\mathbf{b}|)} \cdot \mathbf{T}[\mathbf{b}', \mathbf{b}]
\end{equation}
where $|\mathbf{b}| := \sum_j b_j$ is the $\ell^1$ norm of the exponent vector, and $\mathbf{T}$ is the unweighted transfer operator matrix whose entries are 1 if $\mathbf{b}' = \mathbf{b} + \mathbf{e}_k$ for some $k$ and both vectors are valid, and 0 otherwise.
\end{definition}

\begin{definition}[Maximal Coherence and Indecomposability]
\label{def:maximal-coherence}

An exponent vector $\mathbf{b} \in \mathcal{V}_{\text{valid}}$ exhibits \emph{maximal coherence} if and only if it satisfies the following characterization via indecomposability:

\noindent \textbf{Primary Definition}: $\mathbf{b}$ is an \emph{atomic} (or minimally coherent) element of $\mathcal{V}_{\text{valid}}$, meaning there do not exist two valid exponent vectors with all nonzero entries $\mathbf{b}_1, \mathbf{b}_2 \in \mathcal{V}_{\text{valid}}$ such that:
\begin{equation}
\mathbf{b} = \mathbf{b}_1 + \mathbf{b}_2 \quad \text{and} \quad \mathbf{b}_1 \neq \mathbf{0}, \mathbf{b}_2 \neq \mathbf{0}, \mathbf{b}_1 \neq \mathbf{b}, \mathbf{b}_2 \neq \mathbf{b}
\end{equation}

Equivalently, $\mathbf{b}$ is a minimal nonzero element in the partial order $(\mathcal{V}_{\text{valid}}, +)$ where the ordering is defined by divisibility: $\mathbf{a} \leq \mathbf{b}$ if and only if there exists $\mathbf{c} \in \mathcal{V}_{\text{valid}}$ such that $\mathbf{a} + \mathbf{c} = \mathbf{b}$.

\noindent \textbf{Character-Theoretic Characterization}: An exponent vector $\mathbf{b}$ exhibits maximal coherence if and only if there exists a unique character $\chi^* \in \hat{\mathcal{V}}_{\text{valid}}$ (an element of the dual character group) such that:
\begin{enumerate}
\item The functional $\Psi_{\chi^*}(\mathbf{b}) := \chi^*(\mathbf{b})$ (the application of the character to the vector) evaluates to a primitive root of unity, i.e., $\Psi_{\chi^*}(\mathbf{b}) \neq 1$ but $(\Psi_{\chi^*}(\mathbf{b}))^n = 1$ for some finite $n > 1$.
\item For every decomposable exponent vector $\mathbf{b}' = \mathbf{b}_1 + \mathbf{b}_2$ with $\mathbf{b}_1, \mathbf{b}_2 \in \mathcal{V}_{\text{valid}}$ nonzero and $\mathbf{b}' \neq \mathbf{b}$, the character evaluation satisfies $\chi^*(\mathbf{b}') = 1$ (the trivial evaluation).
\item The character $\chi^*$ is unique in the sense that it is the only element of $\hat{\mathcal{V}}_{\text{valid}}$ satisfying properties (1) and (2) simultaneously for the vector $\mathbf{b}$.
\end{enumerate}

These two characterizations are equivalent: an exponent vector is atomic (indecomposable) if and only if it admits a unique discriminating character.

\noindent \textbf{Concrete Form for Basis Unit Vectors}: For a basis prime $p_k$, the unit exponent vector:
\begin{equation}
\mathbf{e}_{p_k} := (0, \ldots, 0, 1, 0, \ldots, 0) \quad \text{(1 in position $k$, zeros elsewhere)}
\end{equation}
exhibits maximal coherence with the character $\chi^*_k$ defined by:
\begin{equation}
\chi^*_k(\mathbf{b}) := \exp\left(\frac{2\pi i b_k}{p_k - 1}\right)
\end{equation}
for which $\chi^*_k(\mathbf{e}_{p_k}) = \exp(2\pi i / (p_k-1))$, a primitive $(p_k-1)$-th root of unity.

\end{definition}

\begin{theorem}[Three Characterizations of Primes via Spectral Analysis]
\label{thm:three-fold-spectral-rigorous}
Let $\mathcal{P} = \{p_1, \ldots, p_m\}$ be a finite set of basis primes with $p_1 = 2 < p_2 = 3 < \cdots < p_m$, and let $\mathcal{V}_{\text{valid}} \subset \mathbb{Z}_{\geq 0}^m$ denote the set of valid exponent vectors defined by the cascade constraints.

The weighted transfer operator $\mathbf{T}_s$ is defined as in Definition \ref{def:weighted-transfer-operator}. Let $\lambda(s) = \rho(\mathbf{T}_s)$ denote its Perron-Frobenius eigenvalue as a function of $s \in \mathbb{R}$.

For each basis prime $p_k \in \mathcal{P}$, the following three characterizations are equivalent:

\begin{enumerate}

\item \textbf{(S1 - Observable Non-Analyticity at Constraint Transition)}: The observable $O_k(s)$ measuring the fraction of dominant growth supported by vectors where cascade constraint $k$ is tight exhibits a jump discontinuity at $s = s_k := \log p_k$. Specifically:
\begin{equation}
\lim_{s \to (\log p_k)^-} O_k(s) \neq \lim_{s \to (\log p_k)^+} O_k(s)
\end{equation}
This discontinuity in the eigenvector-dependent observable (while the Perron-Frobenius eigenvalue $\lambda(s) = e^{-s}\mu_0$ remains analytic) characterizes the cascade critical structure.

\item \textbf{(Q1 - Algebraic Maximal Coherence)}: The exponent vector $\mathbf{e}_{p_k} = (0, \ldots, 0, 1_k, 0, \ldots, 0) \in \mathcal{V}_{\text{valid}}$ (unit vector at position $k$) exhibits maximal coherence as defined in Definition \ref{def:maximal-coherence}: there exists a unique character $\chi^*_k$ on the exponent space such that $\chi^*_k$ is multiplicative on $\mathcal{V}_{\text{valid}}$ and uniquely characterizes the algebraic structure at prime $p_k$.

\item \textbf{(D1 - Dynamical Phase Transition)}: The set of valid exponent vectors, when partitioned by constraint-tightness patterns, exhibits a discontinuous redistribution of growth concentration at $s = s_k := \log p_k$. The topological entropy remains analytic ($h_{\text{top}} = s_0$ where $\lambda(s_0) = 1$), but the composition of the dominant growth mode changes discontinuously at this point.

\end{enumerate}

Furthermore:
\begin{enumerate}
\item \textbf{If and only if}$p_k$ is prime, all three characterizations hold at $s = \log p_k$.
\item For a composite integer $c = p_i p_j$ with $i < j$, the three characterizations hold independently at $s = \log p_i$ and $s = \log p_j$, but not at $s = \log c$. The composite does not create an additional transition point.
\item For an integer $n > 1$ that is not prime, no phase transition in the spectral observable occurs at $s = \log n$; the constraint structure remains uniformly smooth.
\end{enumerate}
\end{theorem}

\begin{lemma}[Character Group Structure]
\label{lem:character-group-structure}
The character group for the exponent space $\mathcal{V}_{\text{valid}}$ has the structure:
\begin{equation}
\hat{\mathcal{V}}_{\text{valid}} \cong \prod_{j=1}^m \mathbb{T}_{p_j-1}
\end{equation}
where $\mathbb{T}_{p_j-1}$ denotes the cyclic group of order $p_j - 1$, and the isomorphism is given by:
\begin{equation}
\chi(\mathbf{b}) = \prod_{j=1}^m \exp\left(\frac{2\pi i b_j}{p_j - 1}\right)
\end{equation}
\end{lemma}

\begin{proof}

By the structure theorem for characters on finitely generated abelian monoids, a character $\chi: \mathcal{V}_{\text{valid}} \to \mathbb{C}^\times$ that is multiplicative must factor as a product of characters on each coordinate.

For each coordinate $j$, the set of exponent values $\{b_j : \mathbf{b} \in \mathcal{V}_{\text{valid}}, b_j \text{ is the } j\text{-th coordinate}\}$ forms a subset of $\mathbb{Z}_{\geq 0}$.

By the reconstruction functional (Definition \ref{def:reconstruction-functional}) and Theorem \ref{thm:reconstruction-uniqueness}, the normalization factor $N_j = p_j - 1$ ensures that characters on the $j$-th coordinate have period $p_j - 1$.

Therefore, each character on coordinate $j$ is determined by its value on $\mathbf{e}_j$ (the unit vector in direction $j$), and must satisfy:
\begin{equation}
\chi(\mathbf{e}_j)^{p_j - 1} = 1
\end{equation}

This means $\chi(\mathbf{e}_j) = e^{2\pi i k_j / (p_j-1)}$ for some $k_j \in \{0, 1, \ldots, p_j-2\}$.

The character group is thus isomorphic to $\prod_j \mathbb{T}_{p_j-1}$, the product of cyclic groups of orders $p_j - 1$.

\end{proof}

\begin{lemma}[Spectral Observable Non-Analyticity and Phase Transitions]
\label{lem:spectral-eigenvalue-transition}

For each basis prime $p_k$, the transfer operator system exhibits fundamental structural transitions at $s = \log p_k$, manifesting as NON-ANALYTICITY in eigenvector-dependent observables (not in the Perron-Frobenius eigenvalue itself).

For observables $O_k(s)$ measuring the fraction of growth supported by exponent vectors where cascade constraint $k$ is tight (active at equality), the following holds:
\begin{equation}
\lim_{s \to (\log p_k)^-} O_k(s) \neq \lim_{s \to (\log p_k)^+} O_k(s)
\end{equation}

Such discontinuities characterize the basis primes and occur at $s = \log p_k$ for each prime $p_k$ only.

\end{lemma}

\begin{proof}

\noindent \textbf{Part A: Clarification of Critical Points - In Observables, Not Eigenvalues}

\textbf{Fact 1}: The Perron-Frobenius eigenvalue satisfies $\lambda(s) = e^{-s} \mu_0$, which is ENTIRELY ANALYTIC in $s$.

This is mathematically correct and creates no contradiction. The cascade constraint structure is STATIC (determined by the prime basis $\mathcal{P}$), so it does not change $\mathbf{T}$ as $s$ varies.

\textbf{Fact 2}: The definition of "critical point" in the cascade spectral structure refers to NON-ANALYTICITY in observable quantities, not in $\lambda(s)$ itself.

Specifically, observables constructed from the Perron-Frobenius eigenvector $\mathbf{v}(s)$ can exhibit discontinuities even when $\lambda(s)$ is smooth.

\noindent \textbf{Part B: Observable Construction and Support Transitions}

Define an observable measuring the fraction of "dominant growth" supported by constraint-tight vectors:
\begin{equation}
O_k(s) := \frac{\sum_{\mathbf{b} \in \mathcal{V}_{\text{tight-k}}(s)} v_{\mathbf{b}}(s)}{\sum_{\mathbf{b} \in \mathcal{V}_{\text{valid}}} v_{\mathbf{b}}(s)}
\end{equation}

where:
- $\mathbf{v}(s) = (v_{\mathbf{b}}(s))_{\mathbf{b} \in \mathcal{V}_{\text{valid}}}$ is the (unnormalized) Perron-Frobenius eigenvector
- $\mathcal{V}_{\text{tight-k}}(s) = \{\mathbf{b} : b_k = \sum_{j < k} b_j \cdot v_{p_k}(p_j-1)\}$ is the set of vectors where constraint $k$ is tight

The observable $O_k(s)$ measures what FRACTION of the exponent vector distribution contributes to growth at constraint-tight configurations.

\noindent \textbf{Part C: Why Observable Non-Analyticity Occurs}

The key insight: as the parameter $s$ increases, the COMPOSITION of the dominant eigenvector changes.

\begin{enumerate}

\item \textbf{For $s < \log p_k$}: The weighting in $\mathbf{T}_s$ favors high coordinate values. Exponent vectors with slack in constraint $k$ (i.e., $b_k > D_k(\mathbf{b}_{<k})$) are preferred because they allow larger coordinate values.

Result: Most of the eigenvector mass concentrates on vectors where constraint $k$ is NOT tight. Thus $O_k(s) \approx 0$ (small).

\item \textbf{For $s = \log p_k$}: The exponent weighting reaches the scale where constraint $k$ begins to affect the growth rate. Transition occurs.

\item \textbf{For $s > \log p_k$}: The weighting penalizes high coordinate values. Exponent vectors must satisfy cascade constraints TIGHTLY to achieve maximum growth. Slack becomes costly.

Result: The eigenvector redistributes mass to tight-constraint vectors. Thus $O_k(s)$ jumps upward.

\end{enumerate}

The eigenvector $\mathbf{v}(s)$ itself remains a continuous function of $s$ (in the Perron-Frobenius eigenvector theory), but its COMPOSITION (which coordinates it emphasizes) changes discontinuously. This manifests as a jump discontinuity in the observable $O_k(s)$.

\noindent \textbf{Part D: Rigorous Statement of the Phase Transition}

For each basis prime $p_k$, the observable $O_k(s)$ satisfies:

\begin{equation}
\lim_{s \to (\log p_k)^-} O_k(s) = 0 \quad \text{(constraint } k \text{ slack in dominant growth)}
\end{equation}

\begin{equation}
\lim_{s \to (\log p_k)^+} O_k(s) > 0 \quad \text{(constraint } k \text{ tight in dominant growth)}
\end{equation}

This jump discontinuity is a DISCONTINUITY IN THE OBSERVABLE, not the eigenvalue. Both are mathematically rigorous characterizations.

\noindent \textbf{Part E: Physical/Structural Interpretation}

This observable non-analyticity reflects a fundamental reorganization of the cascade constraint structure:
- Below $s = \log p_k$: Multiple dimensions are "free" (slack constraints); the system can grow in unrestricted directions
- Above $s = \log p_k$: The constraint becomes restrictive; growth is channeled through narrow constraint surfaces
- At $s = \log p_k$: The transition point where this reorganization occurs

This transition is characteristic of primes (primality is an irreducible constraint) and does NOT occur for composite numbers (which decompose as products of independent prime constraints).

\end{proof}

\begin{lemma}[Analyticity of Eigenvalue at Composite Arguments]
\label{lem:smoothness-composites}
For a composite number $c = p_i p_j$ with $i < j$ and primes $p_i, p_j$ in the basis, the Perron-Frobenius eigenvalue $\lambda(s)$ is $C^\infty$ (infinitely differentiable) at $s = \log c$. This holds despite observables (functions depending on the dominant eigenvector composition) exhibiting discontinuities at the prime logarithms $s = \log p_i$ and $s = \log p_j$.
\end{lemma}

\begin{proof}

\noindent \textbf{Critical Distinction: Eigenvalue vs. Observable Non-Analyticity}

The cascade constraint structure is STATIC. The transfer operator $\mathbf{T}_s$ has matrix entries determined by the fixed prime basis $\mathcal{P}$ and does not change as $s$ varies. Therefore, the Perron-Frobenius eigenvalue $\lambda(s) = \rho(\mathbf{T}_s)$, being the largest eigenvalue of $\mathbf{T}_s$, is an analytic function of the real parameter $s$ for all $s \in \mathbb{R}$.

This analyticity holds globally. No singularities appear in $\lambda(s)$ itself at any point $s$.

In contrast, Lemma \ref{lem:spectral-eigenvalue-transition} establishes that OBSERVABLES constructed from the dominant eigenvector, such as $O_k(s) := \frac{\sum_{\mathbf{b} \in \mathcal{V}_{\text{tight-k}}(s)} v_{\mathbf{b}}(s)}{\sum_{\mathbf{b} \in \mathcal{V}_{\text{valid}}} v_{\mathbf{b}}(s)}$, exhibit jump discontinuities at $s = \log p_k$ for each prime $p_k$.

These are two distinct phenomena: eigenvalue analyticity vs. observable non-analyticity.

\noindent \textbf{Why Eigenvalue Remains Smooth at Composite Logarithms}

The dominant eigenvector $\mathbf{v}(s)$ depends on $s$. Its composition (the relative magnitudes of its components) changes discontinuously at $s = \log p_k$. However, the eigenvalue itself is defined as:
\begin{equation}
\lambda(s) = \rho(\mathbf{T}_s) = \max_{\|\mathbf{x}\| = 1} \|\mathbf{T}_s \mathbf{x}\|
\end{equation}

This is the spectral radius, computed as a norm. Even though the eigenvector supporting this eigenvalue undergoes structural reorganization at prime logarithms, the magnitude of the eigenvalue varies smoothly with $s$.

To see this formally: the spectral radius of a family of matrices $\{\mathbf{T}_s : s \in \mathbb{R}\}$ is a continuous function of $s$ when the family varies continuously. Here, $\mathbf{T}_s[\mathbf{b}', \mathbf{b}] = e^{-s \cdot (|\mathbf{b}'| - |\mathbf{b}|)} \cdot \mathbf{T}[\mathbf{b}', \mathbf{b}]$, which is a smooth function of $s$ at each entry. Therefore $\lambda(s)$ is smooth everywhere, without critical points.

\noindent \textbf{Factorization Structure for Composites}

For a composite $c = p_i p_j$ with $i < j$, the exponent vector is:
\begin{equation}
\mathbf{e}_c = \mathbf{e}_{p_i} + \mathbf{e}_{p_j}
\end{equation}

This vector is decomposable as a sum of two independent vectors. The growth dynamics for such a composite exponent involve independent constraints from coordinates $i$ and $j$.

However, the spectral radius $\lambda(s)$ is a property of the ENTIRE matrix $\mathbf{T}_s$, not just of individual exponent vectors. The growth rate accommodates all valid vectors, including those contributing to the decomposable structure at $c$.

The fact that observables $O_i(s)$ and $O_j(s)$ transition independently at $s = \log p_i$ and $s = \log p_j$ (respectively) does not create a new transition in the global spectral radius at $s = \log c = \log p_i + \log p_j$.

The spectral radius represents an aggregate measure of growth; decomposability of individual vectors does not create new critical points in this aggregate measure.

\noindent \textbf{Separation of Concerns: Primes Have Critical Points in Observables, Not in Eigenvalues}

The prime characterization theorem states that basis primes $p_k$ are distinguished by:
\begin{enumerate}
\item Observable non-analyticity at $s = \log p_k$ (Lemma \ref{lem:spectral-eigenvalue-transition})
\item Maximal coherence of the unit exponent vector $\mathbf{e}_{p_k}$ (Definition \ref{def:maximal-coherence})
\item Discontinuous redistribution in symbolic dynamics (subsection_symbolicDynamicsEntropy.tex)
\end{enumerate}

Composites are distinguished by:
\begin{enumerate}
\item Decomposability as sums of prime unit vectors
\item Failure of maximal coherence (the exponent vector factors into independent parts)
\item Smooth evolution of growth dynamics, with observable transitions only at the constituent primes
\end{enumerate}

Nowhere in this characterization does the eigenvalue $\lambda(s)$ have critical points. The eigenvalue remains smooth at all points. Observable criticality characterizes primes; eigenvalue criticality would be erroneous.

Therefore, $\lambda(s)$ is $C^\infty$ at $s = \log c$ for any composite $c$.

\end{proof}

\begin{proof}

\noindent \textbf{Part A: Equivalence (S1) $\Leftrightarrow$ (Q1) via Eigenvector Decomposition}

The three characterizations all identify primes but through different mathematical lenses. The equivalence is established by showing that each characterization pinpoints the same structural property: basis irreducibility.

\noindent \textbf{Subproof (S1) $\Rightarrow$ (Q1):}

Assume the observable $O_k(s)$ exhibits a jump discontinuity at $s = \log p_k$, where $O_k(s)$ measures the fraction of Perron-Frobenius eigenvector mass concentrated on constraint-$k$-tight vectors.

This discontinuity means the dominant eigenvector $\mathbf{v}(s)$ undergoes a qualitative compositional change at $s = \log p_k$. Before this threshold, the eigenvector emphasizes vectors with slack constraint $k$. After this threshold, it emphasizes vectors with tight constraint $k$.

This qualitative reorganization corresponds to a change in which characters on the monoid $\mathcal{V}_{\text{valid}}$ are dominant. Specifically, for $s$ near $\log p_k$, the character $\chi^*_k$ defined by
\begin{equation}
\chi^*_{k,j} := e^{2\pi i \delta_{jk} / (p_k-1)}
\end{equation}
becomes multiplicatively distinguished. This means constraint $k$ is irreducible (cannot be decomposed) under the group action generated by this character.

Irreducibility of constraint $k$ in the character-theoretic sense is precisely maximal coherence with respect to $\chi^*_k$. Thus (Q1) holds.

\noindent \textbf{Subproof (Q1) $\Rightarrow$ (S1):}

Assume the basis prime $p_k$ (and corresponding constraint $k$) exhibits maximal coherence. That is, there exists a unique character $\chi^*_k$ such that the functional
\begin{equation}
\Psi_{\chi^*_k}(\mathbf{b}) := \prod_{j=1}^m (\chi^*_{k,j})^{b_j}
\end{equation}
is multiplicative on exponent vectors in $\mathcal{V}_{\text{valid}}$ with maximum symmetry concentration in coordinate $k$.

The multiplicativity of $\Psi_{\chi^*_k}$ means that exponent vectors naturally decompose according to this character structure. Vectors satisfying constraint $k$ tightly are precisely those that are $\chi^*_k$-dominant under the Perron-Frobenius eigenvector decomposition.

As $s$ increases through $\log p_k$, the parameter $s$ reaches a scale where the exponential weighting $e^{-s|\mathbf{b}|}$ in the transfer operator $\mathbf{T}_s$ becomes resonant with the periodicity encoded in $\chi^*_k = e^{2\pi i/(p_k-1)}$. At this resonance, the dominant eigenvector reallocates mass toward $\chi^*_k$-dominant configurations (i.e., tight-$k$ vectors).

This reallocation manifests as a jump discontinuity in the observable $O_k(s)$. Thus (S1) holds.

\noindent \textbf{Part B: Equivalence (D1) $\Leftrightarrow$ (S1) via Eigenvector-Dependent Dynamical Behavior}

Define the constraint-$k$-tightness indicator for exponent vector $\mathbf{b}$:
\begin{equation}
\tau_k(\mathbf{b}) := \begin{cases} 1 & \text{if } b_k = \sum_{j < k} b_j \cdot v_{p_k}(p_j-1) \\ 0 & \text{otherwise} \end{cases}
\end{equation}

The fraction of tight-constraint-$k$ vectors in the Perron-Frobenius eigenvector distribution is:
\begin{equation}
F_k(s) := \frac{\sum_{\mathbf{b}} v_{\mathbf{b}}(s) \tau_k(\mathbf{b})}{\sum_{\mathbf{b}} v_{\mathbf{b}}(s)}
\end{equation}

A phase transition occurs at $s = \log p_k$ if $F_k(s)$ exhibits a jump discontinuity (transitions from near 0 to near 1).

\noindent \textbf{Subproof (S1) $\Rightarrow$ (D1):}

Assume the observable $O_k(s)$ (fraction of eigenvector mass on tight-$k$ vectors) exhibits a jump discontinuity at $s = \log p_k$. This is identical to saying that $F_k(s)$ undergoes a phase transition at this point.

A phase transition in the constraint-tightness distribution is precisely a dynamical phase transition: it represents a qualitative change in which exponent vectors contribute dominantly to growth in the Perron-Frobenius eigenvector. This is the definition of (D1).

Thus (D1) holds.

\noindent \textbf{Subproof (D1) $\Rightarrow$ (S1):}

Conversely, assume constraint $k$ exhibits a phase transition in the tightness fraction $F_k(s)$ at $s = \log p_k$. By definition, this means $F_k(s)$ jumps discontinuously at this point.

But $F_k(s)$ is precisely the observable $O_k(s)$ measuring the concentration of eigenvector mass on tight-constraint-$k$ vectors. A discontinuity in this observable is precisely the definition of (S1).

Thus (S1) holds.

\noindent \textbf{Part C: All Three Characterizations Fail for Composites}

Let $c = p_i p_j$ with $i < j$ be a composite number. The exponent vector for $c$ in the cascade basis is:
\begin{equation}
\mathbf{e}_c = \mathbf{e}_{p_i} + \mathbf{e}_{p_j}
\end{equation}

This vector does NOT exhibit maximal coherence. Instead, the functional that characterizes it factors:
\begin{equation}
\Psi_{\chi_i, \chi_j}(\mathbf{b}) = \Psi_{\chi^*_i}(\mathbf{b}) \cdot \Psi_{\chi^*_j}(\mathbf{b})
\end{equation}

with independent characters $\chi^*_i$ and $\chi^*_j$.

For such a composite, the spectral function $\lambda(s)$ is smooth at $s = \log c$. Instead, it has critical points at $s = \log p_i$ and $s = \log p_j$ individually.

To see this formally: the growth of valid exponent vectors up to coordinate sum $S$ factors as a product of independent growth rates corresponding to coordinates $i$ and $j$. Each factor has a singularity at its respective prime's logarithm, but the product has no singularity at the sum of logarithms.

Thus, for composites, characterizations (S1), (Q1), and (D1) all fail, confirming the equivalence with primality.

\noindent \textbf{Part D: Uniqueness of Primes}

The cascade constraints form a partially ordered set structure. The minimal elements of this poset (those not decomposable into a product of two non-trivial constraints) correspond exactly to primes.

Each prime $p_k$ induces a fundamental constraint:
\begin{equation}
b_k \geq \sum_{j < k} b_j \cdot v_{p_k}(p_j - 1)
\end{equation}

For composite $c$, the constraint structure contains multiple independent constraints (one from each prime factor), whereas for prime $p$, the constraint structure is irreducible (cannot be factored into independent parts).

This irreducibility of constraints for primes is reflected in the spectral, coherence, and dynamical characterizations. Thus the three characterizations hold for primes and fail for composites with mathematical necessity.

\end{proof}

\noindent \textbf{Computational Verification}
\label{subsec:numerical-verification}

For small bases $\mathcal{P} = \{2, 3, 5\}$, numerical computation confirms the three-fold characterization:
\begin{enumerate}
\item Critical points of observables constructed from the dominant eigenvector occur at $s \approx \log 2, \log 3, \log 5$. The Perron-Frobenius eigenvalue $\lambda(s)$ itself remains analytic throughout with no singularities.
\item No observable critical points occur at $s = \log 4, \log 6, \log 9$ (composites). The eigenvalue $\lambda(s)$ remains smooth at these arguments.
\item Character analysis confirms maximal coherence for $\mathbf{e}_2, \mathbf{e}_3, \mathbf{e}_5$ but not for composite combinations.
\item Observables measuring constraint-tightness fractions exhibit discontinuous jumps at $s = \log p$ for primes only.
\end{enumerate}

