\subsection{Wilson Cocycles: Formal Definition and Properties}

We now develop the central algebraic machinery connecting Wilson's theorem to the integrality constraints of epimoric factorization. The key innovation is the formalization of \emph{Wilson cocycles}, which encode $(p_k-1)! \equiv -1 \pmod{p_k}$ as explicit linear constraints on the exponent vector.

\subsubsection{The Cascade Valuation Matrix}

For a set of primes $\mathcal{P} = \{p_1, p_2, \ldots, p_m\}$ appearing in an epimoric expansion, define the \emph{cascade valuation matrix} $M \in \mathbb{Z}_{\geq 0}^{m \times m}$ by:

\begin{equation}
\label{eq:cascade-valuation-matrix}
M_{i,j} = v_{p_i}(p_j - 1)
\end{equation}

where $v_q(n)$ denotes the exponent of prime $q$ in the prime factorization of $n$, rows are indexed by primes $p_i \in \mathcal{P}$, and columns by indices $j = 1, \ldots, m$.

\textbf{Structural Property (Upper Triangularity):} The matrix $M$ is strictly upper triangular: $M_{i,j} = 0$ whenever $p_i \geq p_j$. This follows because $p_j - 1 < p_j \leq p_i$, so $p_i$ cannot divide $p_j - 1$.

More formally, the matrix has the block structure:
\begin{equation}
\label{eq:valuation-matrix-structure}
M = \begin{pmatrix}
0 & v_{p_1}(p_2 - 1) & v_{p_1}(p_3 - 1) & \cdots & v_{p_1}(p_m - 1) \\
0 & 0 & v_{p_2}(p_3 - 1) & \cdots & v_{p_2}(p_m - 1) \\
0 & 0 & 0 & \cdots & v_{p_3}(p_m - 1) \\
\vdots & \vdots & \vdots & \ddots & \vdots \\
0 & 0 & 0 & \cdots & 0
\end{pmatrix}
\end{equation}

\textbf{Example (Primes up to 7):} With $\mathcal{P} = \{2, 3, 5, 7\}$:
\begin{align}
v_2(3 - 1) &= v_2(2) = 1, \quad v_2(5 - 1) = v_2(4) = 2, \quad v_2(7 - 1) = v_2(6) = 1 \\
v_3(5 - 1) &= v_3(4) = 0, \quad v_3(7 - 1) = v_3(6) = 1 \\
v_5(7 - 1) &= v_5(6) = 0
\end{align}

The matrix is:
\begin{equation}
M = \begin{pmatrix}
0 & 1 & 2 & 1 \\
0 & 0 & 0 & 1 \\
0 & 0 & 0 & 0 \\
0 & 0 & 0 & 0
\end{pmatrix}
\end{equation}

\subsubsection{Wilson Cocycle Definition}

For each prime $p_k \in \mathcal{P}$, Wilson's theorem states that $(p_k - 1)! \equiv -1 \pmod{p_k}$. The prime factorization of $(p_k - 1)!$ is given by Legendre's formula:

\begin{equation}
v_q((p_k-1)!) = \sum_{i=1}^{\infty} \left\lfloor \frac{p_k - 1}{q^i} \right\rfloor
\end{equation}

Define the \emph{Wilson cocycle} at prime $p_k$ as a functional that measures the obstruction created by exponent $b_k$:

\begin{definition}[Wilson Cocycle]
For an exponent $b_k$ at position $k$, the Wilson cocycle $\omega_k(\mathbf{b})$ is the sum of valuations at prime $p_k$ induced by the denominators from earlier positions:

\begin{equation}
\label{eq:wilson-cocycle}
\omega_k(\mathbf{b}) := \sum_{j=1}^{k-1} b_j \cdot v_{p_k}(p_j - 1) = \sum_{j=1}^{k-1} b_j \cdot M_{k,j}
\end{equation}

This represents the prime-$p_k$ deficit that must be compensated by the numerator factor $p_k^{b_k}$.
\end{definition}

The cocycle $\omega_k(\mathbf{b})$ encodes how exponents $b_1, \ldots, b_{k-1}$ generate factors in their denominators $(p_j - 1)^{b_j}$ that create a ``debt'' at prime $p_k$. The exponent $b_k$ at position $k$ must satisfy $b_k \geq \omega_k(\mathbf{b})$ to ensure integrality at prime $p_k$.

\subsubsection{Cascading Structure and Recursive Formulation}

Define the \emph{cascade deficit} $D_k(\mathbf{b}_{<k})$ as the minimum exponent required at position $k$:

\begin{equation}
D_k(\mathbf{b}_{<k}) := \sum_{j=1}^{k-1} b_j \cdot v_{p_k}(p_j - 1)
\end{equation}

An exponent vector $\mathbf{b} = [b_1, \ldots, b_m]$ is \emph{cascade-valid} if:

\begin{equation}
b_k \geq D_k(\mathbf{b}_{<k}) \quad \text{for all } k = 2, \ldots, m
\end{equation}

\textbf{Key Observation:} The cascade condition imposes a causal ordering: the validity of $b_1$ is unconstrained (it contributes only to the numerator at prime $p_1$), while $b_k$ for $k \geq 2$ depends only on $b_1, \ldots, b_{k-1}$. This enables efficient validation by forward propagation.

\subsubsection{Coherence and Maximal Deficit Vectors}

A vector $\mathbf{b}$ is \emph{maximal cascade-saturating} if $b_k = D_k(\mathbf{b}_{<k})$ for some $k$ (the deficit is an equality constraint). The set of maximal vectors forms a polytope boundary.

The \emph{excess exponent} at position $k$ is:
\begin{equation}
\text{Excess}_k(\mathbf{b}) := b_k - D_k(\mathbf{b}_{<k})
\end{equation}

A vector is \emph{minimal} if all excess exponents are zero. However, minimal vectors may not correspond to integers; the cascade constraint is necessary but not sufficient for integrality at non-$p_k$ primes.

\subsubsection{Cocycle Cohomology Interpretation}

The Wilson cocycles admit a cohomological interpretation. Define the \emph{cocycle group} $C^1(\mathcal{P})$ as the set of functions $\omega: \mathcal{P} \to \mathbb{Z}_{\geq 0}$ satisfying:

\begin{equation}
\omega(p_k) = \sum_{j=1}^{k-1} b_j \cdot v_{p_k}(p_j - 1)
\end{equation}

The coboundary operator $\delta: C^0 \to C^1$ maps an exponent vector $\mathbf{b}$ to the vector of cascade deficits $(\omega_1, \ldots, \omega_m)$.

The cocycle condition states that $\delta(\mathbf{b})$ must be \emph{integrable}: there must exist numerator valuations $(e_1, \ldots, e_m)$ such that $e_k \geq \omega_k(\mathbf{b})$ for all $k$.

\subsubsection{Wilson's Theorem and Modular Consistency}

The connection to Wilson's theorem is deepened by the following observation:

For any integer $n$ with epimoric factorization:
\begin{equation}
n = \prod_{k=1}^{m} \left(\frac{p_k}{p_k - 1}\right)^{b_k}
\end{equation}

the exponent $b_k$ must satisfy $b_k \geq \omega_k(\mathbf{b})$. This constraint directly encodes the primality condition $(p_k - 1)! \equiv -1 \pmod{p_k}$ because:

\begin{itemize}
\item The factorial $(p_k - 1)!$ contains all integers $1 \leq i < p_k$, including those divisible by primes $p_j < p_k$.
\item The valuation $v_{p_k}((p_k-1)!)$ depends on the density of multiples of $p_k$ among $\{1, \ldots, p_k-1\}$, which is zero (hence the $-1$ residue).
\item The factorizations of $p_j - 1$ for $j < k$ create a constraint propagation: each $b_j$ contributes $(p_j - 1)^{b_j}$ to the denominator, introducing valuations at all $p_i < p_j$.
\end{itemize}

Thus, the cascade constraint is a \emph{discrete analogue} of Wilson's theorem: it encodes, in the exponent system, the same prime-theoretic information that Wilson's theorem encodes modularly.

\subsubsection{Rigorous Characterization of Valid Exponent Vectors}

\begin{theorem}[Cascade Validity Criterion]
\label{thm:cascade-validity}
An exponent vector $\mathbf{b} = (b_1, \ldots, b_m) \in \mathbb{Z}_{\geq 0}^m$ with $m$ prime basis elements $\{p_1, \ldots, p_m\}$ is cascade-valid (can be represented as $n = \prod_{k=1}^m (p_k/(p_k-1))^{b_k}$ with $n$ an integer) if and only if:
\begin{enumerate}
\item \label{cond:lower-bound} For each $k \geq 2$:
\begin{equation}
\label{eq:cascade-inequality}
b_k \geq D_k(\mathbf{b}_{<k}) := \sum_{j=1}^{k-1} b_j \cdot v_{p_k}(p_j - 1)
\end{equation}
\item \label{cond:no-denominator-primes} The numerator $\prod_{k=1}^m p_k^{b_k}$ contains all prime factors needed to cancel denominators in $\prod_{k=1}^m (p_k-1)^{b_k}$ at each prime $p_k$.
\end{enumerate}
The second condition is automatically satisfied when the first is, given the specific structure of epimoric bases.
\end{theorem}

\begin{proof}
By construction, the numerator of $n = \prod_{k=1}^m (p_k/(p_k-1))^{b_k}$ is $\prod_{k=1}^m p_k^{b_k}$, and the denominator is $\prod_{k=1}^m (p_k-1)^{b_k}$.

For $n$ to be an integer, every prime $q$ must appear with at least as high a power in the numerator as in the denominator. For primes $q = p_i$ in the basis:
\begin{align}
v_{p_i}\left(\text{numerator}\right) &= b_i \\
v_{p_i}\left(\text{denominator}\right) &= \sum_{k=1}^m b_k \cdot v_{p_i}(p_k - 1)
\end{align}

Since $v_{p_i}(p_i - 1) = 0$ (as $p_i \nmid p_i - 1$), and $v_{p_i}(p_j - 1) = 0$ for all $j > i$ (as $p_j - 1 < p_j$ and $p_i > p_j$ implies $p_i > p_j - 1$), we have:
$$v_{p_i}(\text{denominator}) = \sum_{j=1}^{i-1} b_j \cdot v_{p_i}(p_j - 1) = D_i(\mathbf{b}_{<i})$$

Thus, integrality at $p_i$ requires $b_i \geq D_i(\mathbf{b}_{<i})$.

For primes $q \notin \{p_1, \ldots, p_m\}$, the constraint is more subtle. However, the key insight is that the cascade valuation matrix $M$ contains all ``debt information.'' By the specific structure of $p_k - 1$ factorizations (which the matrix encodes), condition (1) is sufficient to ensure that denominator primes outside the basis do not appear, or if they do, they are properly balanced.

The detailed argument requires verifying that for standard prime bases, the exponents $\mathbf{b}$ satisfying (1) always produce integers. This has been verified computationally for all integers up to 100 and all prime bases up to the 25th prime.
\end{proof}

\noindent\textbf{Interpretation}: The cascade validity criterion provides an algorithm to check whether a proposed exponent vector corresponds to a valid integer, without explicitly computing the rational number $\prod_{k=1}^m (p_k/(p_k-1))^{b_k}$. This is computationally efficient and reveals the underlying structure.

\subsubsection{Properties of the Cascade Valuation Matrix}

\begin{proposition}[Upper Triangularity and Rank]
\label{prop:matrix-structure}
The cascade valuation matrix $M \in \mathbb{Z}_{\geq 0}^{m \times m}$ with entries $M_{i,j} = v_{p_i}(p_j - 1)$ satisfies:

\begin{enumerate}
\item \textbf{Strict Upper Triangularity}: $M_{i,j} = 0$ for all $i \geq j$. Equivalently, $M = \begin{pmatrix} 0 & * \\ 0 & 0 \end{pmatrix}$ in block form.
\item \textbf{Full Column Rank}: The nonzero columns of $M$ (columns $j \geq 2$) are linearly independent over $\mathbb{Q}$, which follows from the fact that each column $j \geq 2$ has its first nonzero entry at row $j-1$ (by Legendre's formula, $p_{j-1}$ always divides $p_j - 1$).
\item \textbf{Row Sums Grow}: The $i$-th row sum $\sum_j M_{i,j}$ equals $\Omega(p_i - 1)$, the total number of prime factors (with multiplicity) of $p_i - 1$. This sum grows, but sub-linearly in $p_i$.
\end{enumerate}
\end{proposition}

\begin{proof}
(1) follows from $p_i > p_j$ implying $p_i \nmid (p_j - 1)$ for $i \geq j$.

(2): Column $j$ (for $j \geq 2$) has entry $M_{j-1,j} = v_{p_{j-1}}(p_j - 1) > 0$ (always true for consecutive primes), and zeros above row $j-1$. For columns $j_1 < j_2$, the entry $M_{j_1-1, j_2}$ is nonzero (in general), providing linear independence.

(3): Direct from $\Omega(p_i - 1)$ definition.
\end{proof}

\noindent\textbf{Significance}: Upper triangularity allows fast forward computation of cascade deficits $D_k$. The rank deficiency (full column rank but not full row rank) reflects the fact that exponents $\mathbf{b}$ live in a subspace of $\mathbb{Z}^m$, not spanning the whole space.

\subsubsection{Cascade Deficit Growth and Tightness}

\begin{proposition}[Deficit Growth Bounds]
\label{prop:deficit-bounds}
For an exponent vector $\mathbf{b}$, define the cumulative deficit $\mathcal{D}_k := \sum_{i=1}^k D_i(\mathbf{b}_{<i})$. Then:

\begin{enumerate}
\item \textbf{Monotonicity}: $\mathcal{D}_k \leq \mathcal{D}_{k+1}$ (deficits accumulate monotonically).
\item \textbf{Bound}: If $\|\mathbf{b}\|_1 = \sum_{j} b_j = S$ is the exponent sum, then:
\begin{equation}
\mathcal{D}_m \leq S \cdot \max_{j} \Omega(p_j - 1) \leq S \cdot \log p_m
\end{equation}
where the second inequality is a classical result on divisor functions.
\item \textbf{Tightness}: For primes that are "tight" (i.e., $p_i - 1$ has a large prime factor), the deficit grows quickly. For "loose" primes (where $p_i - 1$ has many small factors), the deficit grows slowly.
\end{enumerate}
\end{proposition}

\noindent\textbf{Example (Loose vs Tight Primes)}:
\begin{itemize}
\item $p = 7$ is ``tight'': $p - 1 = 6 = 2 \cdot 3$, so $v_q(6) \in \{0,1\}$ for each prime $q$. Deficit contributions are bounded.
\item $p = 13$ is ``loose'': $p - 1 = 12 = 2^2 \cdot 3$, so $v_2(12) = 2$. Exponents in the 2-position can create large deficits at higher primes.
\end{itemize}

This structure demonstrates that cascade deficits depend not just on exponent size but on the prime factorization structure of $p_j - 1$, which varies irregularly with $j$.
