\section{Entangled vs Independent Multiplicative Bases}

The most fundamental distinction between the prime multiplicative basis and all ratio-based multiplicative bases (epimoric and epimeric systems) is the property of \textbf{base independence versus base entanglement}.

\subsection{Independence in the Prime Basis}

In standard prime factorization, the bases are \textbf{orthogonal} (independent). Each prime $p_k$ contributes independently to the factorization:

\begin{equation}
n = 2^{a_1} \cdot 3^{a_2} \cdot 5^{a_3} \cdot 7^{a_4} \cdots
\end{equation}

Increasing the exponent $a_3$ (the power of $5$) leaves all other exponents unchanged. The prime bases are \textit{mutually independent}.

This independence is why standard prime factorization is so mathematically clean: we can analyze the contribution of each prime separately without worrying about constraints or dependencies on other primes.

\subsection{Entanglement in Epimoric and Epimeric Bases}

In contrast, ratio-based multiplicative bases exhibit \textbf{entanglement}: the bases are not independent. Consider the canonical epimoric system with ratios $2/1, 3/2, 5/4, 7/6, \ldots$:

\begin{equation}
n = \left(\frac{2}{1}\right)^{b_1} \cdot \left(\frac{3}{2}\right)^{b_2} \cdot \left(\frac{5}{4}\right)^{b_3} \cdots
\end{equation}

The key observation: \textbf{the denominators of these ratios overlap with the numerators of earlier ratios}.

- The denominator of $3/2$ is $2$, which is the numerator of $2/1$
- The denominator of $5/4$ is $4 = 2^2$, which contains the prime $2$ from $2/1$
- The denominator of $7/6$ is $6 = 2 \cdot 3$, which contains both $2$ and $3$ from earlier ratios

\subsection{The Cancellation Mechanism}

To produce an integer, the following must occur: every prime appearing in any denominator must also appear in the numerators with at least the same total multiplicity.

For example, to represent $n = 3$ in canonical epimoric form:

\begin{equation}
3 = \left(\frac{2}{1}\right)^{b_1} \cdot \left(\frac{3}{2}\right)^{b_2} \cdot \left(\frac{5}{4}\right)^{b_3} \cdots
\end{equation}

If we set $b_2 = 1$, we get numerator $3$ and denominator $2$. But we need to cancel the $2$ in the denominator. We do this by setting $b_1 = 1$, which contributes a factor of $2$ to the numerator. So:

\begin{equation}
3 = \left(\frac{2}{1}\right)^{1} \cdot \left(\frac{3}{2}\right)^{1} = \frac{2 \cdot 3}{1 \cdot 2} = 3
\end{equation}

Therefore, $3 = [1, 1]_{\text{can-epim}}$.

The representation of $3$ requires both $b_1$ and $b_2$ simultaneously. The exponents are \textit{entangled}: changing one determines constraints on the others to produce a valid integer.

\subsection{Consequences of Entanglement}

The entanglement of bases in ratio systems creates several profound mathematical consequences:

\subsubsection{Constraint Polytope}

Not every exponent vector produces an integer. The valid vectors satisfy a system of linear inequalities (the \textbf{integrality constraints}):

\begin{equation}
v_q\left(\prod_{k} p_k^{b_k}\right) \geq v_q\left(\prod_{k} (p_k-1)^{b_k}\right) \quad \text{for all primes } q
\end{equation}

This defines a \textbf{constraint polytope} in the exponent space. Only lattice points in this polytope correspond to valid integers.

\subsubsection{Semi-Regular Distribution}

The standard omega function $\Omega(n) = \sum_k a_k$ (total multiplicity in prime factorization) exhibits chaotic behavior as $n$ ranges over the integers. The epimoric omega function $\Omega_E(n) = \sum_k b_k$ shows structured behavior bounded by the constraint polytope. The cascade constraint system enforces a deterministic structure on the set of valid exponent vectors, resulting in a bounded distribution of integers along valid coordinate paths.

\subsubsection{Structure Encodes Prime Distribution}

The shape, geometry, and topology of the constraint polytope directly encode information about the prime sequence itself:

\begin{itemize}
\item Forbidden regions (vectors with no valid integers) correspond to prime gaps
\item Density variations in valid vectors correlate with twin prime frequency
\item The recursive structure of constraints mirrors the nested structure of prime factorizations
\end{itemize}

\subsection{The Trade-off: Cleanliness vs Structure}

Standard prime factorization is \textit{algebraically clean} because bases are independent. Any exponent vector produces a valid integer with no constraints.

Ratio-based systems (epimoric and epimeric) are \textit{mathematically messier} because bases are entangled. We must satisfy complex divisibility constraints to ensure the product is an integer.

However, this messiness is structure-rich: the entanglement creates a deeply interconnected web that encodes information about the primes themselves. The constraint polytope reveals hidden geometric structures underlying the prime sequence.

This duality captures a fundamental tension in mathematics:
\begin{enumerate}
\item \textbf{Prime factorization}: Simple, clean, independent bases; no internal structure
\item \textbf{Ratio-based systems}: Complex, entangled bases; rich structure encoding prime properties
\end{enumerate}

Both viewpoints are valuable and complementary. The prime basis is the tool for general arithmetic; the ratio-based systems are the tool for investigating prime distribution and harmonic/musical structures.
