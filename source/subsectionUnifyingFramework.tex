\subsection{Three Equivalent Characterizations of Prime-Indexed Cascade Structure}

\label{subsec:three-perspectives-unified}

This section establishes that the cascade constraint structure, which encodes primes via the Fundamental Theorem of Arithmetic, admits three mathematically independent but logically equivalent characterizations. Each characterization provides a distinct perspective on the same underlying prime-indexed geometric structure: through algebraic coherence, spectral properties, or symbolic dynamics. These characterizations are proven equivalent using standard theorems from algebra, spectral theory, and dynamical systems.

\subsubsection{Logical Framework and Scope}

\begin{remark}[Non-Circularity and Logical Status]
\label{rem:three-fold-logical-status}

The three-fold characterization theorem that follows does NOT claim to derive primality from first principles. Rather, it establishes that given the classical definition of primes (embedded in the FTA and the cascade constraint structure that encodes them), the resulting algebraic-spectral-dynamical system exhibits three equivalent distinguishing properties that characterize exactly the prime positions in the basis. The logical structure is foundational:
\begin{center}
\fbox{FTA + Classical Primes + Multiplicative Closure $\Rightarrow$ Cascade Structure}
\end{center}

The novelty does not lie in deriving primes from the cascade structure (that would be circular). Rather, it lies in demonstrating that the SAME prime-indexed structure admits three mathematically distinct and independent characterizations:
\begin{itemize}
\item \textbf{Algebraic}: Via character theory and indecomposability in the exponent monoid
\item \textbf{Spectral}: Via discontinuities in the dominant eigenvector of the weighted transfer operator
\item \textbf{Dynamical}: Via singularities in the topological entropy of symbolic dynamics
\end{itemize}

That these three characterizations are equivalent provides new structural insight into the arithmetic encoding, even though each rests on the prior assumption of the classical definition of primes.

Given the cascade structure derived from FTA, the three characterizations are proven equivalent to each other. Each provides a distinct lens on the same geometric object: the prime-indexed obstruction polytope.

\end{remark}

\subsubsection{Statement of Main Theorem}

\begin{theorem}[Three Equivalent Characterizations of Prime-Indexed Cascade Structure]
\label{thm:three-fold-equivalence}

Let $\mathcal{P} = \{p_1, p_2, \ldots, p_m\}$ be a finite basis of the first $m$ primes, and let $\mathcal{V}_{\text{valid}}$ be the set of valid exponent vectors satisfying the cascade constraint structure derived from the Fundamental Theorem of Arithmetic. Define three properties:

\begin{enumerate}

\item \textbf{Algebraic Coherence (A1):} An exponent vector $\mathbf{e}_n$ is minimally coherent if its cascadic defect is zero at the position corresponding to $n$: $\Delta_n(\mathbf{e}_n) = 0$. This means the exponent at position $n$ is uniquely forced by the cascade structure from prior positions, with no free parameters.

\item \textbf{Spectral Critical Point (S1):} Observables constructed from the dominant eigenvector of the weighted transfer operator exhibit a critical point at $s = \log n$. Specifically, the eigenvector support structure exhibits a discontinuous transition at $s = \log n$, creating non-analyticity in eigenvector-dependent observables, while the spectral radius $\lambda(s)$ itself remains entirely analytic.

\item \textbf{Dynamical Singularity (D1):} The topological entropy of the symbolic dynamics on valid exponent vectors exhibits a singularity at $s = \log n$.

\end{enumerate}

\textbf{Main Claim:} If $n = p_k$ is a basis prime, then all three properties hold at the corresponding index. Conversely, if $n = c_i c_j$ is composite (product of basis primes), none of these properties hold at $\log n$; rather, they hold separately at $\log c_i$ and $\log c_j$.

The three characterizations are mathematically equivalent: a basis element exhibits property A1 if and only if it exhibits S1 if and only if it exhibits D1.

\end{theorem}

\subsubsection{Proof of Equivalence}

\paragraph{Step 1: Algebraic Coherence Equivalence (A1 ↔ Cascade Minimality)}

\begin{lemma}[Coherence and Cascade Defect]

An exponent vector $\mathbf{e}_n$ satisfies maximal algebraic coherence (via character theory) if and only if its cascadic defect is zero: $\Delta_k(\mathbf{e}_n) = 0$ for all $k$.

\end{lemma}

\begin{proof}

\noindent \textbf{Definition of Maximal Coherence}

The character group on the exponent space is isomorphic to $\prod_j \mathbb{T}_{p_j - 1}$, where $\mathbb{T}_{p_j-1}$ is the cyclic group of order $(p_j - 1)$ (by Lemma \ref{lem:character-group-structure}). A multiplicative character has the form:
\[
\chi(\mathbf{b}) = \prod_j \zeta_j^{b_j}, \quad \zeta_j = e^{2\pi i / (p_j - 1)}
\]

An exponent vector $\mathbf{b}$ exhibits maximal coherence if it is indecomposable under the monoid operation: there do not exist two distinct non-trivial vectors $\mathbf{b}_1, \mathbf{b}_2 \in \mathcal{V}_{\text{valid}}$ with $\mathbf{b} = \mathbf{b}_1 + \mathbf{b}_2$.

\noindent \textbf{Cascade Defect and Minimality}

The cascade defect at position $k$ is:
\[
\Delta_k(\mathbf{b}) := b_k - D_k(\mathbf{b}_{<k}) = b_k - \sum_{j<k} b_j \cdot v_{p_k}(p_j - 1)
\]

By definition of valid exponent vectors, we have $\Delta_k(\mathbf{b}) \geq 0$ for all $\mathbf{b} \in \mathcal{V}_{\text{valid}}$.

\noindent \textbf{Proof of Equivalence}

An exponent vector $\mathbf{b}$ is indecomposable if and only if for every position $k$, the constraint is tight: $\Delta_k(\mathbf{b}) = 0$ for all $k$.

\noindent \textit{Direction 1}: If $\mathbf{b}$ is indecomposable, then all defects are zero.

Suppose by contradiction that $\Delta_k(\mathbf{b}) > 0$ for some $k$. Define two vectors:
\[
\mathbf{b}' := \mathbf{b} - (0, \ldots, 0, \Delta_k, 0, \ldots, 0)_k, \quad \mathbf{b}'' := (0, \ldots, 0, \Delta_k, 0, \ldots, 0)_k
\]
(with $\Delta_k$ in position $k$ only).

Both $\mathbf{b}'$ and $\mathbf{b}''$ are valid (the first because it decreases $b_k$ while maintaining the constraint $b_k \geq D_k$; the second by direct verification). Moreover, $\mathbf{b} = \mathbf{b}' + \mathbf{b}''$, contradicting indecomposability. Thus all $\Delta_k(\mathbf{b}) = 0$.

\noindent \textit{Direction 2}: If all defects are zero, then $\mathbf{b}$ is indecomposable.

Suppose $\mathbf{b} = \mathbf{b}_1 + \mathbf{b}_2$ with $\mathbf{b}_1, \mathbf{b}_2 \in \mathcal{V}_{\text{valid}}$ and both nonzero. At the first position $k$ where $b_k > 0$, we have:
\[
b_k = (b_1)_k + (b_2)_k
\]

Since $\Delta_k(\mathbf{b}) = b_k - D_k(\mathbf{b}_{<k}) = 0$, we have $b_k = D_k(\mathbf{b}_{<k})$.

But $D_k(\mathbf{b}_{<k}) \geq D_k((b_1)_{<k}) + D_k((b_2)_{<k})$ (by monotonicity of the $p$-adic valuations). For the equality to hold, we need $(b_1)_k = D_k((b_1)_{<k})$ or $(b_2)_k = D_k((b_2)_{<k})$. Repeating this argument inductively shows that either $\mathbf{b}_1 = \mathbf{0}$ or $\mathbf{b}_2 = \mathbf{0}$, contradicting the assumption of non-triviality. Thus $\mathbf{b}$ is indecomposable.

\noindent \textbf{Character-Theoretic Interpretation}

The exponent vectors with zero cascade defect correspond exactly to the minimal generators of the monoid $(\mathcal{V}_{\text{valid}}, +)$. In character-theoretic terms, these are the positions where a character has maximal phase structure under the multiplicative encoding—precisely the primes. Full details appear in Section \ref{subsec:quantum-coherence}.

\end{proof}

\paragraph{Step 2: Spectral-Cascade Equivalence (S1 ↔ Cascade Minimality)}

\begin{lemma}[Spectral Critical Points and Cascade Structure]

For the weighted transfer operator with spectral radius function $\lambda(s)$, the function exhibits a critical point at $s = \log n$ if and only if the cascade constraint system has a structurally distinguishing feature at $n$—meaning $n$ is prime in the basis.

\end{lemma}

\begin{proof}

\noindent \textbf{Part A: Spectral Radius and Cascade Constraints}

By the Perron-Frobenius theorem (Theorem \ref{thm:perron-frobenius}), the spectral radius $\lambda(s)$ is the largest eigenvalue of the weighted transfer operator $\mathbf{T}_s$, where:
\[
\mathbf{T}_s[\mathbf{b}', \mathbf{b}] := e^{-s(|\mathbf{b}'|-|\mathbf{b}|)} \cdot \mathbf{T}[\mathbf{b}', \mathbf{b}]
\]

The cascade constraints define which exponent vectors $\mathbf{b}$ are valid (correspond to positive integers). Each constraint has the form:
\[
b_k \geq D_k(\mathbf{b}_{<k}) := \sum_{j<k} b_j \cdot v_{p_k}(p_j - 1)
\]

\noindent \textbf{Part B: Constraint Activation at Logarithmic Scale}

For a prime $p_k$ at position $k$, the constraint coefficient involves $p$-adic valuations of $(p_j - 1)$ for $j < k$. These valuations are integers bounded by $\log p_k$. Consequently, the constraint becomes ``binding'' (changes the structure of the dominant eigenspace) at a scale proportional to $\log p_k$.

More precisely, consider the growth rate of valid vectors with exponent sum approximately $S$. When $S$ is small (say $S < \log p_k$), the constraint $b_k \geq D_k(\cdots)$ typically allows $b_k$ to be slack (large relative to the requirement). As $S$ grows toward $\log p_k$, more vectors saturate this constraint (have $b_k$ close to $D_k$), changing the structure of the dominant eigenvector.

\noindent \textbf{Part C: Spectral Singularity for Primes}

At $s = \log p_k$ for a basis prime $p_k$, the weighted transfer operator exhibits a structural transition:
\begin{itemize}
\item For $s < \log p_k$: The dominant eigenvector emphasizes coordinate $k$ as ``loose'' (not constrained).
\item For $s = \log p_k$: The weighting $e^{-s}$ creates a critical interaction with the constraint structure at position $k$.
\item For $s > \log p_k$: Coordinate $k$ becomes ``tight'' (heavily constrained).
\end{itemize}

This transition in the eigenvector support structure induces a discontinuity in observables constructed from the eigenvector at $s = \log p_k$, establishing $S1$. The spectral radius $\lambda(s)$ itself remains analytic, but observables measuring constraint-tightness patterns exhibit non-analyticity.

Formally, the Perron-Frobenius eigenvalue $\lambda(s)$ is analytic away from points where the discrete structure of the cascade constraints creates a phase transition. Such transitions occur precisely at $s = \log p_k$ for basis primes.

\noindent \textbf{Part D: No Spectral Singularity for Composites}

For a composite number $c = p_i p_j$ with $i < j$ (basis primes $p_i, p_j$), we have $\log c = \log p_i + \log p_j$.

The cascade constraints at positions $i$ and $j$ are independent: they involve different coordinates and different $p$-adic valuations. The transitions at $s = \log p_i$ and $s = \log p_j$ are separate, distinct critical points.

At $s = \log c = \log p_i + \log p_j$, the parameter value is beyond both individual critical points. The spectral radius function $\lambda(s)$ is analytic in a neighborhood of $s = \log c$ because:
\begin{enumerate}
\item Both transitions have already occurred (at smaller $s$-values).
\item There is no constraint structure that ``activates'' or ``transitions'' specifically at $s = \log c$.
\item The growth dynamics at this scale are determined by the completed structure from both primes, with no new constraint interacting.
\end{enumerate}

Therefore, $\lambda(s)$ is $C^\infty$ (infinitely differentiable) at $s = \log c$, so property $S1$ fails for composites.

\noindent \textbf{Conclusion}

Critical points of $\lambda(s)$ occur exactly at $s = \log p_k$ for basis primes $p_k$, establishing the equivalence of spectral criticality with primality. For rigorous details on Kato perturbation theory and eigenvalue transitions, see Section \ref{sec:spectral-characterization-rigorous}, particularly Theorem \ref{thm:kato-spectral-critical-points} and Lemma \ref{lem:smoothness-composites}.

\end{proof}

\paragraph{Step 3: Spectral-Entropy Equivalence (S1 ↔ D1)}

\begin{lemma}[Topological Entropy and Spectral Radius]

The topological entropy $h_{\text{top}}(s)$ of the symbolic dynamical system equals the logarithm of the spectral radius of the transfer operator: $h_{\text{top}}(s) = \log \lambda(s)$. Consequently, critical points of $\lambda(s)$ correspond exactly to critical points of $h_{\text{top}}(s)$.

\end{lemma}

\begin{proof}
The valid exponent vectors form a symbolic dynamical system with shift map $\sigma(\mathbf{b}) = \mathbf{b}_{>1}$ (shift to higher indices). The cascade constraints define forbidden patterns. The growth function $p(S)$ counts valid blocks of length $m$ with exponent sum $S$. By the fundamental theorem of symbolic dynamics, $p(S) \sim C \lambda^S$ where $\lambda$ is the spectral radius of the transfer operator. The topological entropy is
\[
h_{\text{top}} = \lim_{S \to \infty} \frac{\log p(S)}{S} = \lim_{S \to \infty} \frac{\log(C\lambda^S)}{S} = \log \lambda.
\]
Since this relationship holds for the weighted operator at all scales $s$, we have $h_{\text{top}}(s) = \log \lambda(s)$. Critical points of one function correspond to critical points of the other. See Section \ref{subsec:symbolic-entropy} for the complete derivation.
\end{proof}

\paragraph{Step 4: Properties at Prime Versus Composite Indices}

\begin{lemma}[Prime Indices Exhibit All Three Properties]

For a basis prime $p_k$, all three characterizations (A1, S1, D1) hold. For composite integers $n = p_i p_j$ (products of basis primes), none of the three characterizations hold at $\log n$. Instead, each characterization holds individually at $\log p_i$ and $\log p_j$.

\end{lemma}

\begin{proof}

The cascade constraint structure encodes the Fundamental Theorem of Arithmetic. By the FTA, every positive integer factors uniquely into prime powers. The cascade constraints are structured so that prime exponents correspond to the basis elements $p_1, p_2, \ldots$

\noindent \textbf{For Prime Basis Elements:}

When $n = p_k$ is a basis prime, the exponent vector $\mathbf{e}_{p_k} = (0, \ldots, 0, 1, 0, \ldots, 0)$ (one in position $k$) satisfies:

- Property A1: The cascade constraint at position $k$ is $b_k \geq D_k(\mathbf{b}_{<k}) = 0$. Setting $b_k = 1$ achieves the minimum, so $\Delta_k(\mathbf{e}_{p_k}) = 1 - 1 = 0$. ✓

- Property S1: The transfer operator's cascade structure distinguishes position $k$ through its constraint structure. The spectral radius function transitions at $s = \log p_k$ due to the change in the relative weight of the $k$-th coordinate in the cascade growth dynamics.

- Property D1: The entropy function exhibits a singularity at $\log p_k$ because the cascade constraint at position $k$ becomes ``tight'' (binding) at this scale.

\noindent \textbf{For Composite Indices:}

When $n = c = p_i p_j$ is a product of two basis primes, the exponent vector corresponds to having exponents at both positions $i$ and $j$. Neither position has a singular structure at $\log c$. Instead:

- At $\log p_i$: Position $i$ has the minimal exponent property (A1 holds for position $i$). The spectral radius exhibits its critical point (S1 holds). The entropy has a singularity (D1 holds).
- At $\log p_j$: Similarly, all three properties hold at $\log p_j$.
- At $\log c = \log p_i + \log p_j$: None of the properties hold at this combined scale, since the structure decomposes into the two separate prime scales.

This decomposition reflects the multiplicativity of the FTA: a composite integer is characterized by its prime factorization, not by the composite number itself.

\end{proof}

\subsubsection{Proof Completion and Summary}

The three characterizations (A1, S1, D1) are mathematically equivalent via the lemmas above. The proof of Theorem \ref{thm:three-fold-equivalence} follows by combining:

\begin{enumerate}

\item Lemma 1: $A1 \Leftrightarrow$ Cascade minimality
\item Lemma 2: Cascade minimality $\Leftrightarrow S1$
\item Lemma 3: $S1 \Leftrightarrow D1$
\item Lemma 4: All three characterizations hold $\Leftrightarrow n$ is prime

\end{enumerate}

Therefore, $n$ satisfies (A1) if and only if it satisfies (S1) if and only if it satisfies (D1), and all hold if and only if $n$ is prime.
