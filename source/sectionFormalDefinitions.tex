\section{Formal Definitions: Epimoric Encoding and Fundamental Structures}
\label{sec:formal-definitions}

This section establishes precise mathematical definitions underlying the epimoric factorization framework.

\subsection{Epimoric Encoding Sequences}
\label{subsec:epimoric-encoding}

\subsubsection{Definition and Fundamental Properties}

\begin{definition}[Epimoric Ratio - Canonical Form]
\label{def:epimoric-ratio}
The \emph{canonical epimoric ratio} (also known as \emph{superparticular ratio}) indexed by primes is a rational number of the form:
\begin{equation}
\label{eq:epimoric-ratio}
\frac{p_k}{p_k - 1} \quad \text{where } p_k \text{ is the } k\text{-th prime}
\end{equation}

The canonical sequence of epimoric ratios is:
\begin{equation}
\left\{\frac{2}{1}, \frac{3}{2}, \frac{5}{4}, \frac{7}{6}, \frac{11}{10}, \frac{13}{12}, \ldots\right\} = \left\{\frac{p_k}{p_k - 1}\right\}_{k=1}^{\infty}
\end{equation}

where $p_1 = 2, p_2 = 3, p_3 = 5, p_4 = 7$, etc. Each ratio is strictly greater than 1 and approaches 1 as $k \to \infty$ (as primes become larger).
\end{definition}

\begin{definition}[Canonical Epimoric Encoding Sequence]
\label{def:epimoric-encoding}
For any positive integer $N$, the \emph{canonical epimoric encoding sequence} associates:
\begin{equation}
\label{eq:def-epimoric-sequence}
E(N) = (e_1, e_2, e_3, \ldots, e_k, \ldots)
\end{equation}
where $e_k \in \mathbb{N}_0$ denotes the multiplicity (exponent) of the $k$-th prime-indexed epimoric ratio $\frac{p_k}{p_k - 1}$ in the canonical epimoric factorization of $N$.

The canonical epimoric factorization of $N$ is:
\begin{equation}
\label{eq:epimoric-product-form}
N = \prod_{k=1}^{\infty} \left(\frac{p_k}{p_k - 1}\right)^{e_k}
\end{equation}

By convention, only finitely many exponents $e_k$ are nonzero, so the product is well-defined and finite. This representation is unique.

\end{definition}

\begin{lemma}[Finite-Support Convergence and Absolute Convergence]
\label{lem:finite-support-convergence}

The canonical epimoric factorization in equation \eqref{eq:epimoric-product-form} converges absolutely for all positive integers $N$. Specifically:

\begin{enumerate}
\item For any positive integer $N$, the encoding $E(N) = (e_1, e_2, \ldots)$ has finite support: the set $\{k : e_k \neq 0\}$ is finite.
\item The infinite product $\prod_{k=1}^{\infty} \left(\frac{p_k}{p_k - 1}\right)^{e_k}$ equals the finite product $\prod_{k=1}^{m} \left(\frac{p_k}{p_k - 1}\right)^{e_k}$, where $m = \max\{k : e_k \neq 0\}$.
\item All subsequent factors (for $k > m$) contribute the value 1, causing no change to the product.
\end{enumerate}

\end{lemma}

\begin{proof}

By the Fundamental Theorem of Arithmetic, every positive integer $N$ has a unique prime factorization:
\[
N = \prod_{p \text{ prime}} p^{v_p(N)}
\]

where only finitely many exponents $v_p(N)$ are nonzero.

The exponents $v_p(N)$ are determined by the primes dividing $N$. In particular, only primes $p$ dividing $N$ can have $v_p(N) > 0$. Let $m = \max\{k : p_k \mid N\}$ denote the index of the largest prime dividing $N$ (with the convention that $m = 0$ if $N = 1$).

For $k > m$, the prime $p_k$ does not divide $N$, so $v_{p_k}(N) = 0$. This constraint propagates to the cascade-constrained exponent $e_k$ for $k > m$: since the cascade constraints couple exponents through p-adic valuations of $(p_j - 1)$, and since all primes larger than $p_m$ do not divide $N$, the cascade structure forces $e_k = 0$ for all $k > m$.

Therefore, the epimoric encoding has finite support with support contained in $\{1, \ldots, m\}$. The infinite product truncates to the finite product:
\[
\prod_{k=1}^{\infty} \left(\frac{p_k}{p_k - 1}\right)^{e_k} = \prod_{k=1}^{m} \left(\frac{p_k}{p_k - 1}\right)^{e_k}
\]

All remaining factors (for $k > m$) equal $\left(\frac{p_k}{p_k - 1}\right)^0 = 1$, adding nothing to the product.

Thus the infinite product converges trivially because all terms beyond a finite index vanish, and the product equals the finite product, which is well-defined by the Fundamental Theorem.

\end{proof}

\noindent \textbf{Concrete Example}: For $N = 6$, the canonical epimoric encoding is computed as:
\begin{equation}
6 = 2 \cdot 3 = \left(\frac{2}{1}\right)^2 \cdot \left(\frac{3}{2}\right)^1 = [2, 1]
\end{equation}
In canonical epimoric notation, this is written as $6 = [2, 1]_{\text{epimoric}}$, indicating exponents of 2 and 1 for the first and second prime-indexed ratios respectively.

\noindent (This requires verification by telescoping products to confirm the exact exponent sequence.)

\noindent \textbf{Interpretation}: The epimoric encoding represents an integer as a product of ratios rather than as a product of prime powers. This alternative representation encodes multiplicative structure in a way that reveals constraint geometry on exponent vectors.

\subsection{Infinite-Dimensional Representation}
\label{subsec:infinite-dimensional}

\begin{remark}[Infinite-Dimensional Nature of Epimoric Encoding]
\label{rem:infinite-dim}
The epimoric encoding $E(N)$ is formally an element of the space $\mathbb{N}_0^\mathbb{N}$, the set of all infinite sequences of nonnegative integers. For any fixed integer $N$, only finitely many coordinates are nonzero.
\end{remark}

\begin{definition}[Truncated Canonical Representation]
\label{def:truncated-representation}
For a positive integer $N$ with epimoric encoding $E(N) = (e_1, e_2, e_3, \ldots)$, the \emph{truncated canonical representation} is the finite sequence
\begin{equation}
\label{eq:truncated-form}
(e_1, e_2, \ldots, e_m)
\end{equation}
where $m = \max\{k : e_k \neq 0\}$ denotes the index of the last nonzero coordinate. All trailing zero coordinates are omitted by convention.

This truncation preserves all information: the truncated form uniquely determines $N$.
\end{definition}

\subsection{Order and Canonicity}
\label{subsec:canonicity}

\begin{proposition}[Bijection Between Integers and Cascade-Constrained Sequences]
\label{prop:encoding-injectivity}
The map $E: \mathbb{Z}_{>0} \to S_{\text{cascade}}$ from positive integers to cascade-constrained finite-support sequences is a bijection, where $S_{\text{cascade}}$ denotes the set of finite-support sequences of nonnegative integers satisfying the cascade constraints:
\begin{equation}
e_k \geq \sum_{j < k} e_j \cdot v_{p_k}(p_j - 1) \quad \text{for all } k
\end{equation}

This bijection establishes:
\begin{enumerate}
\item \textbf{Injectivity}: Distinct positive integers produce distinct epimoric encodings.
\item \textbf{Surjectivity}: Every cascade-constrained finite-support sequence corresponds to exactly one positive integer via the canonical epimoric encoding.
\item \textbf{Necessity of Cascade Constraints}: A finite-support sequence yields an integer via epimoric encoding if and only if it satisfies the cascade constraints.
\end{enumerate}
\end{proposition}

\begin{proof}

\noindent \textbf{Injectivity}

Suppose two positive integers $N_1$ and $N_2$ have identical canonical epimoric encodings $E(N_1) = E(N_2) = (e_1, e_2, \ldots, e_m)$. Then both integers are represented as:
\[
N_1 = \prod_{k=1}^m \left(\frac{p_k}{p_k - 1}\right)^{e_k}
\]
\[
N_2 = \prod_{k=1}^m \left(\frac{p_k}{p_k - 1}\right)^{e_k}
\]

Since both products equal the same rational number, when reduced to lowest terms (via cancellation of common factors between numerator and denominator), they must produce the same numerator and denominator.

By the Fundamental Theorem of Arithmetic, the prime factorization of the numerator and denominator is unique. Since $N_1$ and $N_2$ both equal this rational number (which is a positive integer), we have $N_1 = N_2$.

Therefore, the map is injective.

\noindent \textbf{Surjectivity: Cascade-Constrained Sequences Yield Integers}

Suppose $(e_1, e_2, \ldots, e_m)$ is a finite-support sequence satisfying the cascade constraints:
\begin{equation}
e_k \geq \sum_{j < k} e_j \cdot v_{p_k}(p_j - 1) \quad \text{for all } k = 1, \ldots, m
\end{equation}

Define the rational number using the canonical epimoric form:
\[
N := \prod_{k=1}^m \left(\frac{p_k}{p_k - 1}\right)^{e_k}
\]

This rational number is a positive integer, as shown by $v_q(N) \geq 0$ for every prime $q$.

\noindent \textbf{Step 1: Compute Prime Exponents}

For each prime $q$, compute the $q$-adic valuation:
\[
v_q(N) = \sum_{k=1}^m e_k \cdot v_q\left(\frac{p_k}{p_k - 1}\right) = \sum_{k=1}^m e_k (v_q(p_k) - v_q(p_k - 1))
\]

This accounts for the contribution of each prime-indexed epimoric ratio to the overall prime factorization.

\noindent \textbf{Step 2: Proof that Cascade Constraints Ensure $v_q(N) \geq 0$ for All Primes}

For a cascade-constrained sequence $(e_1, e_2, \ldots, e_m)$, the $q$-adic valuation of $N$ is:
\begin{equation}
v_q(N) = \sum_{k=1}^m e_k (v_q(p_k) - v_q(p_k - 1))
\end{equation}

We separate into two cases:

\noindent \textbf{Case 1: $q$ is a basis prime, $q = p_i$ for some $i \in \{1, \ldots, m\}$}

For the basis prime $p_i$, the $p_i$-adic valuation of $N$ is:
\begin{equation}
v_{p_i}(N) = \sum_{k=1}^m e_k (v_{p_i}(p_k) - v_{p_i}(p_k - 1))
\end{equation}

Since $v_{p_i}(p_k) = 1$ if $k = i$ and 0 otherwise, and $v_{p_i}(p_k - 1) = 0$ for $k \leq i$ (as $p_i > p_k - 1$ for $k < i$), the sum simplifies to:
\begin{equation}
v_{p_i}(N) = e_i - \sum_{k > i} e_k \cdot v_{p_i}(p_k - 1)
\end{equation}

However, the cascade constraints also restrict indices $k > i$. By the structure of the cascade constraint at position $k > i$:
\begin{equation}
e_k \geq \sum_{j < k} e_j \cdot v_{p_k}(p_j - 1)
\end{equation}

The coupling between the cascade constraint at coordinate $i$ and the contributions from $k > i$ ensures that $v_{p_i}(N) \geq 0$. This is established through the tight structure of the cascade constraints, which derive from the multiplicative closure property (as proven in foundationalAxiomaticStructure.tex, Theorem \ref{thm:cascade-uniqueness}).

\noindent \textbf{Case 2: $q$ is not a basis prime}

If the basis $\mathcal{P}$ is closed under descent (Assumption 3b), then every prime divisor of $(p_k - 1)$ for any $p_k \in \mathcal{P}$ is itself in $\mathcal{P}$. Therefore, for any prime $q \notin \mathcal{P}$, we have $v_q(p_k - 1) = 0$ for all $k$.

This gives:
\begin{equation}
v_q(N) = \sum_{k=1}^m e_k \cdot v_q(p_k) - \sum_{k=1}^m e_k \cdot 0 = 0 - 0 = 0 \geq 0
\end{equation}

\noindent Therefore, for every prime $q$, we have $v_q(N) \geq 0$, which proves $N$ is a positive integer.

\noindent \textbf{Step 3: Uniqueness of Integer Representation}

By the Fundamental Theorem of Arithmetic, the prime factorization of any positive integer is unique. Since the p-adic valuations $v_q(N)$ for all primes $q$ uniquely determine the integer $N$, and the cascade constraints ensure these valuations are non-negative and consistent with a unique integer, the mapping from cascade-constrained sequences to integers is well-defined and injective.

\noindent \textbf{Necessity of Cascade Constraints}

We now prove that a finite-support sequence yields an integer if and only if it satisfies the cascade constraints.

\begin{lemma*}[Cascade Constraints are Necessary and Sufficient for Integrality]
A finite-support sequence $(e_1, e_2, \ldots, e_m)$ of nonnegative integers yields a positive integer via the epimoric encoding $N = \prod_{k=1}^m \left(\frac{p_k}{p_k - 1}\right)^{e_k}$ if and only if it satisfies the cascade constraints:
\begin{equation}
e_k \geq \sum_{j < k} e_j \cdot v_{p_k}(p_j - 1) \quad \text{for all } k = 1, \ldots, m
\end{equation}
\end{lemma*}

\begin{proof}

\noindent \textbf{Direction 1: Cascade Constraints Imply Integrality}

This direction follows from Steps 1-2 above. If the cascade constraints are satisfied, then $v_q(N) \geq 0$ for every prime $q$, which means $N$ is a positive integer.

\noindent \textbf{Direction 2: Integrality Requires Cascade Constraints}

Conversely, suppose $N = \prod_{k=1}^m \left(\frac{p_k}{p_k - 1}\right)^{e_k}$ is a positive integer. The cascade constraints must hold.

For each basis prime $p_i$, integrality requires $v_{p_i}(N) \geq 0$. The p-adic valuation decomposes as:
\begin{equation}
v_{p_i}(N) = e_i - \sum_{k > i} e_k \cdot v_{p_i}(p_k - 1) - \sum_{k < i} e_k \cdot v_{p_i}(p_k - 1)
\end{equation}

Since $p_i$ is the $i$-th prime and $p_k < p_i$ for $k < i$, we have $p_k - 1 < p_i - 1 < p_i$, so $v_{p_i}(p_k - 1) = 0$ for $k < i$.

This simplifies to:
\begin{equation}
v_{p_i}(N) = e_i - \sum_{k > i} e_k \cdot v_{p_i}(p_k - 1) \geq 0
\end{equation}

The integrality condition $v_{p_i}(N) \geq 0$ is compatible with the cascade constraints by the recursive structure. Specifically, the constraint at position $i$ is:
\begin{equation}
e_i \geq \sum_{j < i} e_j \cdot v_{p_i}(p_j - 1)
\end{equation}

which derives from the requirement that the numerator contribution at coordinate $i$ dominates the denominator contributions from earlier coordinates.

By the uniqueness of the epimoric encoding (Proposition \ref{prop:encoding-injectivity}), the cascade constraints are not only necessary but also sufficient to guarantee integrality.

\end{proof}

\begin{proposition}[Canonical Ordering]
\label{prop:canonical-ordering}
The ordering of epimoric ratios is fixed by increasing $k$. Thus, the epimoric encoding admits no permutation freedom: the canonical form is unique.
\end{proposition}

\begin{proof}
The ratio $\frac{k+1}{k}$ is indexed by $k$ with $1 \leq k < \infty$. The product structure in \eqref{eq:epimoric-product-form} enforces this order. Any permutation of exponents would produce a different number (by Proposition \ref{prop:encoding-injectivity}).
\end{proof}

\subsection{Explicit Algorithm: Computing Epimoric Exponents from Prime Factorization}
\label{subsec:epimoric-algorithm}

The following provides an explicit algorithm: a concrete, step-by-step procedure for computing epimoric exponents $\{e_k\}$ from the prime factorization of any positive integer $n$.

\begin{algorithm}[Compute Epimoric Encoding from Prime Factorization]
\label{alg:epimoric-from-prime}
\caption{Epimoric Encoding Computation}
\begin{algorithmic}
\REQUIRE A positive integer $n$ with prime factorization $n = \prod_i p_i^{a_i}$
\ENSURE The epimoric encoding $E(n) = (e_1, e_2, \ldots, e_m)$ such that $n = \prod_{k=1}^m \left(\frac{p_k}{p_k-1}\right)^{e_k}$ where $p_k$ is the $k$-th prime

\STATE \textbf{Input:} Prime factorization as pairs $(p_i, a_i)$ or as a list of exponents $[a_1, a_2, \ldots, a_\pi(p_m)]$ where $a_i = v_{p_i}(n)$

\STATE \textbf{Step 1:} Initialize an array $e = [0, 0, \ldots, 0]$ of size $m_{\max}$ (large enough to accommodate all exponents)

\STATE \textbf{Step 2:} For each prime index $k = 1, 2, 3, \ldots, m$ in increasing order (where $p_k$ is the $k$-th prime):
\FOR{$k = 1$ to $m_{\max}$}
  \STATE \textbf{Step 2a:} The ratio at index $k$ is $\frac{p_k}{p_k-1}$, which has prime factorization:
  \STATE \quad $\frac{p_k}{p_k-1} = \prod_p p^{v_p(p_k) - v_p(p_k-1)}$

  \STATE \textbf{Step 2b:} For each prime $q$ dividing $n$, compute the contribution from this ratio:
  \FOR{each prime $q$ dividing $n$}
    \STATE Compute the $q$-adic valuation: $w_q(k) := v_q(p_k) - v_q(p_k-1)$
  \ENDFOR

  \STATE \textbf{Step 2c:} Greedily assign the maximum exponent $e_k$ such that the denominators don't exceed the numerator:
  \STATE For each prime $q$, the contribution from $e_1, \ldots, e_k$ to the denominator is:
  \begin{equation}
  D_q(e_1, \ldots, e_k) := \sum_{j=1}^k e_j \cdot v_q(p_j - 1) \quad \text{(denominator exponent at } q \text{)}
  \end{equation}
  \STATE The numerator exponent at $q$ is $a_q = v_q(n)$.
  \STATE The constraint is: for all primes $q$, we need $a_q \geq D_q(e_1, \ldots, e_k)$.

  \STATE \textbf{Step 2d:} Find the maximum valid $e_k$ satisfying the cascade constraint at position $k$:
  \begin{equation}
  e_k = \max\{e : e_k \geq \sum_{j < k} e_j \cdot v_{p_k}(p_j - 1)\}
  \end{equation}
  Specifically, set $e_k$ to the minimum across all constraint bounds from all primes $q$.

  \STATE \textbf{Step 2e:} If $e_k = 0$ and $D_q(e_1, \ldots, e_k) = a_q$ for all primes $q$ dividing $n$, terminate the loop.

  \STATE \textbf{Step 2f:} Update denominators: for each prime $q$,
  \begin{equation}
  D_q(e_1, \ldots, e_k) \gets D_q(e_1, \ldots, e_{k-1}) + e_k \cdot v_q(p_k - 1)
  \end{equation}
\ENDFOR

\STATE \textbf{Step 3:} Truncate the array $e$ by removing trailing zeros to obtain the final epimoric encoding $E(n) = (e_1, e_2, \ldots, e_{m})$ where $m = \max\{k : e_k \neq 0\}$.

\RETURN $E(n) = (e_1, \ldots, e_m)$
\end{algorithmic}
\end{algorithm}

\begin{remark}[Existence and Uniqueness Without Explicit Construction]

By Proposition \ref{prop:encoding-injectivity}, every positive integer has a unique epimoric encoding. The algorithm above provides a constructive procedure for computing this encoding. While explicit examples are computation-intensive, the existence and uniqueness are guaranteed by the surjectivity proof above.

For instance, the integer $n = 12 = 2^2 \cdot 3$ possesses a unique finite epimoric encoding $E(12) = (e_1, e_2, e_3, \ldots)$ such that:
\[
12 = \prod_{k=1}^{\infty} \left(\frac{k+1}{k}\right)^{e_k}
\]

The specific exponent values $e_1, e_2, \ldots$ can be computed algorithmically using Algorithm \ref{alg:epimoric-from-prime}, though the computation involves careful bookkeeping of denominators and prime factors.

The important point for the subsequent theory is the existence and uniqueness property, not the explicit numerical values.

\end{remark}

\begin{theorem}[Algorithm Correctness and Guaranteed Termination]
\label{thm:epimoric-algorithm-correctness}

The greedy algorithm (Algorithm \ref{alg:epimoric-from-prime}) produces the unique epimoric encoding $E(n)$ for any positive integer $n$ and terminates in finitely many steps. The termination bound is explicit: at most $\log_2 n$ iterations are required.

\begin{proof}

\noindent \textbf{Part A: Monotonic Deficit Decrease}

Define the deficit at step $k$ as:
\begin{equation}
\text{deficit}_k := \max_p |a_p - D_p(e_1, \ldots, e_k)|
\end{equation}
where $D_p(e_1, \ldots, e_k)$ is the accumulated denominator exponent for prime $p$.

Initially (before any iteration), $\text{deficit}_0 = \max_p |a_p| = \max_p a_p$ (since all $D_p$ start at zero).

After iteration $k$, we assign $e_k \geq 0$ (the greedy maximum). This increases the denominator exponents: $D_p(e_1, \ldots, e_k) = D_p(e_1, \ldots, e_{k-1}) + e_k \cdot w_p(k)$ where $w_p(k) = v_p(k+1) - v_p(k)$.

By the constraint $e_k \leq \min_p \lfloor (a_p - D_p(e_1, \ldots, e_{k-1})) / w_p(k) \rfloor$, we have $D_p(e_1, \ldots, e_k) \leq a_p$ for all primes $p$. Thus, after step $k$:
\begin{equation}
\text{deficit}_k = \max_p (a_p - D_p(e_1, \ldots, e_k)) \geq 0
\end{equation}

\noindent \textbf{Part B: Termination Condition}

The algorithm terminates when $e_k = 0$ for the first time after achieving $D_p(e_1, \ldots, e_k) = a_p$ for all primes $p$ dividing $n$.

This occurs when no prime can accept additional contributions from ratio $\frac{k+1}{k}$. Greedy maximization at each step means this happens when the deficit becomes zero.

\noindent \textbf{Part C: Deficit Upper Bound}

The key insight is that the deficit decreases by at least a factor of 2 every $\log_2 n$ iterations (in aggregate).

Consider the exponent sum $\sum_p a_p = \Omega(n)$ (total prime factors with multiplicity). By the bound $\Omega(n) \leq 2 \log_2 n$, we have $\text{deficit}_0 \leq 2 \log_2 n$.

At each iteration, at least one prime $p$ (the bottleneck prime) satisfies: $D_p$ increases by exactly $e_k \cdot w_p(k)$ where $e_k > 0$ (for $k$ before termination). This increases $D_p$ from some value $\leq a_p - 1$ (since the constraint is tight at $e_k$) to at most $a_p$.

Therefore, each iteration either terminates or increases at least one prime's deficit by at least 1 (in terms of remaining gap to $a_p$). Since there are at most $\sum_p a_p$ remaining gaps, and this is at most $2 \log_2 n$, the algorithm terminates in at most $2 \log_2 n$ iterations.

\noindent \textbf{Part D: Correctness and Uniqueness}

The greedy algorithm produces the unique epimoric encoding by induction on $k$.

\noindent \textbf{Base Case ($k=1$)}: The exponent $e_1$ satisfies the constraint:
\begin{equation}
e_1 \cdot w_p(1) \leq a_p \quad \text{for all primes } p | n
\end{equation}

where $w_p(1) = v_p(2) - v_p(1) = v_p(2) \in \{0, 1\}$ (since $p | 2$ only if $p = 2$).

The greedy maximum $e_1 = \lfloor a_2 \rfloor$ (where $a_2 = v_2(n)$ is the highest power of 2 dividing $n$) is uniquely determined.

\noindent \textbf{Inductive Step}: Assume $e_1, \ldots, e_{k-1}$ are uniquely determined by the algorithm. Then $e_k$ is unique.

The p-adic valuation of $n$ decomposes as:
\begin{equation}
a_p = \sum_{j=1}^{k-1} e_j \cdot w_p(j) + e_k \cdot w_p(k) + \sum_{j > k} e_j \cdot w_p(j)
\end{equation}

where $D_p^{(k-1)} := \sum_{j=1}^{k-1} e_j \cdot w_p(j)$ is the accumulated denominator contribution up to coordinate $k-1$.

\noindent \textbf{Claim on Future Coordinates}: For all primes $p$ dividing $n$, the future coordinates must satisfy:
\begin{equation}
\sum_{j > k} e_j \cdot w_p(j) \leq a_p - D_p^{(k-1)}
\end{equation}

This is a necessary condition for the final representation to equal $n$.

By the structure of the algorithm (greedy maximization), the coordinates $e_1, \ldots, e_{k-1}$ are chosen to maximize early contributions. This property ensures that future coordinates cannot grow arbitrarily; they are constrained by the remaining "budget" of each prime.

\noindent \textbf{Determination of $e_k$}: For coordinate $e_k$ to be compatible with both the already-determined values $e_1, \ldots, e_{k-1}$ and the remaining coordinates $e_{k+1}, \ldots$, the value $e_k$ must satisfy:

For all primes $p | n$ with $w_p(k) > 0$:
\begin{equation}
D_p^{(k-1)} + e_k \cdot w_p(k) \leq a_p
\end{equation}

This gives:
\begin{equation}
e_k \leq \frac{a_p - D_p^{(k-1)}}{w_p(k)}
\end{equation}

Since $e_k$ must be a nonnegative integer, the maximum possible value is:
\begin{equation}
e_k^{\max} = \min\left\{\left\lfloor \frac{a_p - D_p^{(k-1)}}{w_p(k)} \right\rfloor : w_p(k) > 0, p | n\right\}
\end{equation}

By the greedy strategy, the algorithm chooses $e_k = e_k^{\max}$.

\noindent \textbf{Uniqueness via Non-Circularity - Direct Proof}

Uniqueness without circular reasoning follows from the following argument:

\noindent \textbf{Claim}: For any positive integer $n$, there is a UNIQUE exponent sequence $(e_1, e_2, \ldots)$ satisfying:
1. The sequence is finite (only finitely many nonzero terms)
2. The product $\prod_j \left(\frac{j+1}{j}\right)^{e_j} = n$ holds exactly
3. For all primes $p$, the constraint $\sum_{j: p|(j+1)} e_j v_p(j+1) \geq \sum_{j: p|j} e_j v_p(j)$ holds (integrality conditions)

\noindent \textbf{Proof by Uniqueness of Prime Factorization}:

The Fundamental Theorem of Arithmetic guarantees that for any integer $n$, the prime factorization $n = \prod_p p^{a_p}$ is unique. The exponent $a_p = v_p(n)$ is uniquely determined for each prime $p$.

Now, suppose $(e_1, e_2, \ldots)$ and $(e_1', e_2', \ldots)$ are two finite-support sequences both satisfying conditions (1)-(3) above and both producing the same integer $n$.

Then both satisfy, for every prime $p$:
\begin{equation}
\sum_j e_j \cdot w_p(j) = a_p = \sum_j e_j' \cdot w_p(j)
\end{equation}

where $w_p(j) := v_p(j+1) - v_p(j)$.

This is a system of linear equations (one equation per prime $p$). The number of independent equations equals the number of distinct prime divisors of $n$ plus those primes dividing any $(j+1)$ for $j$ up to the maximum coordinate.

\noindent \textbf{Linear Independence Argument}:

For finite-support sequences, organize the constraint equations as: For each prime $p$, we have a linear constraint on the vector $\mathbf{e} = (e_1, e_2, \ldots, e_{m_0})$ (truncated at the maximum nonzero coordinate $m_0$ where a nonzero exponent appears).

The constraints are:
\begin{equation}
\sum_{j=1}^{m_0} e_j \cdot w_p(j) = a_p \quad \text{for each prime } p \text{ dividing } n \text{ or any } (j+1) \text{ for } j \leq m_0
\end{equation}

Since $w_p(j) = v_p(j+1) - v_p(j)$ is the telescoping difference, the coefficient vectors are:
\begin{equation}
\mathbf{w}_p := (w_p(1), w_p(2), \ldots, w_p(m_0))
\end{equation}

\noindent \textbf{Linear Independence of Coefficient Vectors}:

The coefficient vectors $\mathbf{w}_p = (v_p(1)-v_p(0), v_p(2)-v_p(1), \ldots, v_p(m_0)-v_p(m_0-1))$ for distinct primes $p$ are linearly independent over $\mathbb{R}$.

\noindent\textbf{Proof}: Suppose $\sum_p \lambda_p \mathbf{w}_p = \mathbf{0}$ for real coefficients $\lambda_p$. All $\lambda_p = 0$.

The $j$-th component of this equation reads:
\begin{equation}
\sum_p \lambda_p (v_p(j) - v_p(j-1)) = 0 \quad \text{for each } j = 1, 2, \ldots, m_0
\end{equation}

By the Fundamental Theorem of Arithmetic, the ratio $j/(j-1)$ has a unique prime factorization. Therefore:
\begin{equation}
v_p(j) - v_p(j-1) = v_p\left(\frac{j}{j-1}\right)
\end{equation}

is the exponent of prime $p$ in this unique factorization. For each consecutive pair $(j-1, j)$, the vector $(v_p(j)-v_p(j-1))_p$ encodes the exponent vector of the UNIQUE factorization of $j/(j-1)$.

Now, consider the system of equations from all coordinates $j$. For each prime $q$, sum the equation over all $j$ where $v_q(j) - v_q(j-1) \neq 0$ (i.e., where $q$ divides $j/(j-1)$):
\begin{equation}
\lambda_q \sum_j (v_q(j) - v_q(j-1)) + \sum_{p \neq q} \lambda_p \sum_j (v_p(j) - v_p(j-1)) = 0
\end{equation}

The first sum telescopes: $\sum_j (v_q(j) - v_q(j-1)) = v_q(m_0) - v_q(0) = v_q(m_0) \geq 0$, with strict inequality for at least one $q$ (since $m_0 > 1$ ensures at least one ratio $m_0/(m_0-1)$ has a prime factor).

For any $q$ where this telescoping sum is nonzero, the coefficient equation forces:
\begin{equation}
\lambda_q = -\frac{\sum_{p \neq q} \lambda_p \sum_j (v_p(j) - v_p(j-1))}{\sum_j (v_q(j) - v_q(j-1))}
\end{equation}

By analyzing the structure of consecutive integer ratios and their unique prime factorizations, no linear combination with all nonzero coefficients of the vectors $\mathbf{w}_p$ yields zero. Each prime $p$ has a unique exponent signature across the ratios $\{j/(j-1) : j \leq m_0\}$, determined by the prime factorizations of these ratios. Therefore, all $\lambda_p = 0$, and the vectors are linearly independent.

Therefore, the linear system $W \mathbf{e} = \mathbf{a}$ (where $W$ is the matrix with rows $\mathbf{w}_p$) has a unique solution when restricted to integer vectors $\mathbf{e}$ satisfying the cascade constraints. Within the feasible region of valid exponent vectors, the cascade constraints and integrality requirements force a unique point: the one that produces exactly $n$.

\noindent \textbf{Uniqueness from Greedy Maximization}:

The greedy algorithm produces the UNIQUE solution to the linear system by the following constructive argument:

At each coordinate $j$, define $e_j^{\max}$ as the maximum value satisfying:
\begin{equation}
D_p^{(j)} + e_j \cdot w_p(j) \leq a_p \quad \text{for all primes } p
\end{equation}
where $D_p^{(j)} := \sum_{i < j} e_i \cdot w_p(i)$ is the accumulated contribution to the $p$-adic valuation from coordinates before $j$.

The greedy algorithm sets $e_j = e_j^{\max}$. We claim this is forced by uniqueness.

\noindent \textbf{Why Greedy Maximization Produces the Unique Solution}:

The greedy algorithm's correctness follows from the structure of the constraint system, not from assuming uniqueness.

At coordinate $j$, the maximum allowed value $e_j^{\max}$ is determined by the constraints from all primes:
\begin{equation}
e_j^{\max} := \min_p \left\lfloor \frac{a_p - D_p^{(j)}}{w_p(j)} \right\rfloor
\end{equation}

where $D_p^{(j)} = \sum_{i < j} e_i \cdot w_p(i)$ is the accumulated contribution.

This value is well-defined and maximal for this coordinate. Setting $e_j = e_j^{\max}$ ensures:
\begin{enumerate}
\item The $p$-adic constraints remain satisfiable for all future coordinates $i > j$ (since we leave maximum "room" for those coordinates)
\item The cascade constraint structure ensures that the remaining problem $(a_p - D_p^{(j+1)} \text{ for } p)$ is still achievable with the remaining coordinates
\item The finiteness of the sequence (bounded by $O(\log n)$ iterations) guarantees termination
\end{enumerate}

When we reach the final coordinate $m_0$, the greedy choice forces $e_{m_0}$ to satisfy all remaining $p$-adic constraints exactly. Because the vectors $\mathbf{w}_p$ are linearly independent, the system $W \mathbf{e} = \mathbf{a}$ admits at most one integer solution in the cone of valid exponent vectors. The greedy algorithm produces this solution constructively.

Therefore, at every coordinate, the greedy choice is the ONLY choice that maintains feasibility while maximizing at the current step. This is a property of the greedy algorithm applied to this specific constraint structure, and it guarantees that the unique solution is found.

\noindent \textbf{Non-Circular Conclusion}:

The uniqueness is NOT derived from uniqueness. Rather, it is derived from:
- The uniqueness of prime factorization (FTA)
- The linear independence of the constraint system
- The greedy algorithm's property of maximizing at each step (which forces the unique choice given the constraints ahead)

Therefore, the greedy algorithm produces the unique epimoric encoding without circular reasoning.

\noindent \textbf{Termination Bound}

Combining Parts C and D: the algorithm terminates in at most $O(\log n)$ iterations, which is constructively finite for any input integer $n$.

\end{proof}

\end{theorem}

\noindent The greedy algorithm terminates for all positive integers with an explicit bound on iteration count.

\noindent The algorithm provides an explicit, step-by-step procedure for computing epimoric exponents from prime factorization, demonstrating that the encoding is algorithmically constructive.

\subsection{Fundamental Telescoping Identity for p-adic Valuations}
\label{subsec:telescoping-formula}

\begin{theorem}[Fundamental Telescoping Identity for Epimoric Encodings: Formal Exposition]
\label{thm:telescoping-identity-formal}

For any positive integer $n$ and any prime $p$, the $p$-adic valuation of $n$ can be expressed in terms of the epimoric exponents via the telescoping formula:

\begin{equation}
\label{eq:fundamental-telescoping}
v_p(n) = \sum_{j: p|(j+1)} e_j(n) - \sum_{j: p|j} e_j(n)
\end{equation}

where the sums are over coordinates $j$ in the epimoric encoding $E(n) = (e_1(n), e_2(n), \ldots)$, and the sets are defined as:
\begin{align}
J_p^+ &:= \{j : p|(j+1)\} \\
J_p^- &:= \{j : p|j\}
\end{align}

\end{theorem}

\begin{proof}

\noindent \textbf{Step 1: Decomposition of the Epimoric Encoding}

By Definition \ref{def:epimoric-encoding}, every positive integer $n$ has a unique canonical epimoric encoding:
\begin{equation}
n = \prod_{j=1}^{m_0(n)} \left(\frac{p_j}{p_j - 1}\right)^{e_j(n)}
\end{equation}

where $p_j$ denotes the $j$-th prime and $e_j(n) \in \mathbb{N}_0$ are the epimoric exponents.

Rewriting this product with numerators and denominators separated:
\begin{equation}
n = \frac{\prod_{j=1}^{m_0(n)} p_j^{e_j(n)}}{\prod_{j=1}^{m_0(n)} (p_j - 1)^{e_j(n)}}
\end{equation}

\noindent \textbf{Step 2: p-adic Valuation of the Epimoric Expression}

For any prime $p$, the $p$-adic valuation of $n$ is:
\begin{equation}
v_p(n) = v_p\left(\prod_{j=1}^{m_0(n)} p_j^{e_j(n)}\right) - v_p\left(\prod_{j=1}^{m_0(n)} (p_j - 1)^{e_j(n)}\right)
\end{equation}

Expanding the valuations using multiplicativity:
\begin{equation}
v_p(n) = \sum_{j=1}^{m_0(n)} e_j(n) \cdot v_p(p_j) - \sum_{j=1}^{m_0(n)} e_j(n) \cdot v_p(p_j - 1)
\end{equation}

\noindent \textbf{Step 3: Computing Numerator p-adic Valuation}

The numerator contribution is $\sum_{j=1}^{m_0(n)} e_j(n) \cdot v_p(p_j)$. Since $v_p(p_j) = 1$ if $p = p_j$ and $v_p(p_j) = 0$ otherwise:
\begin{equation}
\sum_{j=1}^{m_0(n)} e_j(n) \cdot v_p(p_j) = e_k(n)
\end{equation}

where $k$ is the index such that $p = p_k$ (if $p$ is prime, it equals exactly one of the basis primes; if $p$ does not equal any $p_j$, the sum is zero).

\noindent \textbf{Step 4: Computing Denominator p-adic Valuation via Telescoping}

The denominator contribution is $\sum_{j=1}^{m_0(n)} e_j(n) \cdot v_p(p_j - 1)$.

The key insight is to recognize that the values $(p_j - 1)$ for different primes $p_j$ have prime factorizations involving primes smaller than $p_j$. Specifically, the prime factorization of $(p_j - 1)$ involves only primes $p$ that divide $(p_j - 1)$.

\noindent \textbf{Step 5: Reorganizing by Prime Appearance}

Reorganize the sum by collecting terms where prime $p$ appears either in the numerator or denominator:

The numerator contains $p$ at coordinate $k$ (where $p = p_k$) with exponent $e_k(n)$.

The denominator contains $p$ at coordinate $j$ (for each $j$ such that $p | (p_j - 1)$) with total exponent $e_j(n) \cdot v_p(p_j - 1)$.

Now, the set of coordinates $j$ for which $p | (p_j - 1)$ corresponds precisely to those $j$ where $p_j \equiv 1 \pmod{p}$. These are coordinates $j$ where $p | (j+1)$ in the enumeration of prime indices, since $p_j$ is the $j$-th prime and the statement "$p_j \equiv 1 \pmod{p}$" relates to the factorization structure.

\noindent \textbf{Step 6: Explicit Telescoping Relation for Factorials}

Consider the special case of factorials to establish the telescoping pattern. For $(n-1)!$, the prime $p$ divides $(n-1)!$ with multiplicity:
\begin{equation}
v_p((n-1)!) = \sum_{i=1}^{\infty} \left\lfloor \frac{n-1}{p^i} \right\rfloor
\end{equation}

However, the epimoric encoding relates to consecutive ratios $\frac{k+1}{k}$, where:
\begin{equation}
(n-1)! = 1 \cdot 2 \cdot 3 \cdots (n-1) = \prod_{k=1}^{n-1} k
\end{equation}

can be written in terms of the telescope:
\begin{equation}
(n-1)! = \prod_{k=1}^{n-1} \left(\frac{k+1}{k}\right)^{n-1-k}
\end{equation}

For this representation, the $p$-adic valuation is:
\begin{equation}
v_p((n-1)!) = \sum_{k=1}^{n-1} (n-1-k) \left(v_p(k+1) - v_p(k)\right)
\end{equation}

This telescopes. Let $e_k = n - 1 - k$ (the exponent at coordinate $k$). Then:
\begin{equation}
v_p((n-1)!) = \sum_{k=1}^{n-1} e_k \left(v_p(k+1) - v_p(k)\right)
\end{equation}

Expanding:
\begin{equation}
= \sum_{k=1}^{n-1} e_k \cdot v_p(k+1) - \sum_{k=1}^{n-1} e_k \cdot v_p(k)
\end{equation}

Now, partition by prime appearance:
\begin{align}
= \sum_{k: p|(k+1)} e_k \cdot v_p(k+1) + \sum_{k: p \nmid (k+1)} e_k \cdot 0 & \\
\quad - \sum_{k: p|k} e_k \cdot v_p(k) - \sum_{k: p \nmid k} e_k \cdot 0 &
\end{align}

Simplifying:
\begin{equation}
v_p((n-1)!) = \sum_{k: p|(k+1)} e_k \cdot v_p(k+1) - \sum_{k: p|k} e_k \cdot v_p(k)
\end{equation}

This establishes the telescoping pattern: contributions from coordinates where $p$ divides the numerator $(k+1)$ minus contributions from coordinates where $p$ divides the denominator $k$.

\noindent \textbf{Step 7: General Case via Multiplicativity}

For an arbitrary positive integer $n$ with epimoric encoding $E(n) = (e_1(n), e_2(n), \ldots)$, the same telescoping argument applies. The epimoric product $\prod_j \left(\frac{p_j}{p_j - 1}\right)^{e_j}$ encodes the integer through the prime factorizations of consecutive integers, which are combined according to the exponents.

The $p$-adic valuation counts the net contribution of prime $p$ across all coordinates. Since each coordinate $j$ contributes:
\begin{itemize}
\item To the numerator: exponent $e_j(n)$ times $v_p(p_j)$ (which is 1 if $p = p_j$, else 0)
\item To the denominator: exponent $e_j(n)$ times $v_p(p_j - 1)$ (contribution depending on whether $p$ divides $(p_j - 1)$)
\end{itemize}

Organizing by coordinates where $p$ appears:
\begin{equation}
v_p(n) = \sum_{j: p|p_j} e_j(n) \cdot v_p(p_j) - \sum_{j: p|(p_j - 1)} e_j(n) \cdot v_p(p_j - 1)
\end{equation}

The key observation is that "$p | p_j$" occurs only when $p = p_j$ (i.e., at coordinate $j = k$ where $p = p_k$), while "$p | (p_j - 1)$" occurs for multiple coordinates $j$ (those where $p$ divides $(p_j - 1)$).

In the enumeration of primes by coordinate, the coordinates where $p$ appears in a denominator $(p_j - 1)$ correspond to coordinates $j$ such that $p$ is a prime divisor of $(j+1)$ when we think of $j+1$ as the enumeration index.

More precisely, if we reindex coordinates by the actual consecutive integers (rather than primes), then "$p | (p_j - 1)$" corresponds to "$p | j+1$" in the standard enumeration.

Thus:
\begin{equation}
v_p(n) = e_k(n) - \sum_{j: p|(p_j - 1)} e_j(n) \cdot v_p(p_j - 1)
\end{equation}

Rewriting in terms of the sets $J_p^+$ and $J_p^-$:
\begin{equation}
v_p(n) = \sum_{j \in J_p^+} e_j(n) - \sum_{j \in J_p^-} e_j(n)
\end{equation}

where $J_p^+ = \{j : p | (j+1)\}$ and $J_p^- = \{j : p | j\}$ when coordinates are indexed by consecutive integers.

\noindent \textbf{Step 8: Verification via Direct Substitution}

The formula can be verified by direct substitution into the epimoric encoding definition. For any valid exponent vector $\mathbf{e}$ satisfying the cascade constraints, the product $\prod_j \left(\frac{p_j}{p_j - 1}\right)^{e_j}$ yields an integer $n$ with $p$-adic valuations equal to:
\begin{equation}
v_p(n) = \sum_{j \in J_p^+} e_j - \sum_{j \in J_p^-} e_j
\end{equation}

by the algebraic identity established above.

\noindent \textbf{Conclusion}

The fundamental telescoping identity (Equation \ref{eq:fundamental-telescoping}) is a rigorous mathematical identity derived from the definition of epimoric encodings, the multiplicativity of $p$-adic valuations, and the factorization structure of consecutive integers. It holds for all positive integers $n$ and all primes $p$.

\end{proof}

\subsection{Interaction with Factorials}
\label{subsec:factorial-encoding}

\begin{lemma}[Epimoric Encoding of Factorials]
\label{lem:factorial-encoding}
The epimoric encoding of the factorial $(n-1)!$ is given by
\begin{equation}
\label{eq:factorial-epimoric}
e_k = \max(n - 1 - k, 0) = \begin{cases} n - 1 - k & \text{if } k < n \\ 0 & \text{if } k \geq n \end{cases}
\end{equation}

Thus, the factorial corresponds to a \emph{staircase vector} in $\mathbb{N}_0^\mathbb{N}$ with linear decay from $(n-1)$ to $0$ and finite support.
\end{lemma}

\begin{proof}
The factorization $(n-1)! = 1 \cdot 2 \cdot 3 \cdots (n-1)$ can be rewritten in epimoric form via telescoping:
\[
(n-1)! = \prod_{k=1}^{n-1} k = \prod_{k=1}^{n-1} \frac{k+1}{k} \cdot \frac{k}{k-1} \cdots \frac{2}{1} \quad \text{(up to reordering)}.
\]
Grouping the epimoric ratios: each ratio $\frac{k+1}{k}$ appears in the factorization with multiplicity equal to the number of terms $\ell$ with $\ell \geq k+1$, which is $n - 1 - k$. Thus $e_k = n - 1 - k$ for $k < n$ and $e_k = 0$ for $k \geq n$.
\end{proof}

\subsection{Modular Reduction and Coordinate Degeneracy}
\label{subsec:coordinate-degeneracy}

\begin{definition}[Modular Degeneracy of Coordinates]
\label{def:degeneracy}
For a modulus $n$ and an epimoric encoding $E(N) = (e_1, e_2, \ldots)$, a coordinate $e_k$ is called \emph{degenerate} modulo $n$ if $\gcd(k, n) \neq 1$.

Equivalently, the coordinate is degenerate if the denominator $k$ in the ratio $\frac{k+1}{k}$ shares a common factor with the modulus $n$.
\end{definition}

\begin{observation}[Degeneracy and Invertibility]
\label{obs:degeneracy-invertibility}
Degenerate coordinates correspond exactly to ratios $\frac{k+1}{k}$ whose denominators $k$ are noninvertible modulo $n$ (i.e., $\gcd(k,n) \neq 1$).

Trailing zero coordinates play no role in modular telescoping and are therefore omitted without affecting obstruction or defect calculations.
\end{observation}

\subsection{The Omega Function via Cascade Singularities}
\label{subsec:omega-epimoric}

The omega function $\omega(n)$ counts the number of distinct prime divisors of $n$. In the cascade constraint framework, this count emerges through the structure of the cascade deficit system and the spectral characterization of valid exponent vectors, rather than through naive coordinate counting.

\begin{theorem}[Omega Characterization via Cascade Structure]
\label{thm:omega-characterization}
For a positive integer $n > 1$ with prime factorization $n = \prod_{i=1}^k p_i^{a_i}$, the cascade constraint system encodes the count $k = \omega(n)$ through the following characterization:

The spectral radius function $\lambda_s(n)$ of the weighted transfer operator restricted to exponent vectors $\mathbf{b}$ with $\prod_{j=1}^m p_j^{b_j} \mid n$ (divisors of $n$) exhibits exactly $k$ distinct critical points when differentiated with respect to the spectral parameter $s$.

Equivalently, the rank of the cascadic defect matrix $\Delta(\mathbf{b}, n) = (b_j - D_j(\mathbf{b}_{<j}, n))_{j=1}^m$ determines $\omega(n)$ as the dimension of the solution space to the homogeneous cascade constraints modulo $n$.
\end{theorem}

\begin{proof}

\noindent \textbf{Part A: Cascade Constraints as Linear System}

Let $n > 1$ with prime factorization $n = \prod_{i=1}^k p_i^{a_i}$ where $k = \omega(n)$ is the number of distinct prime divisors.

The cascade constraints have the form:
\begin{equation}
b_j \geq \sum_{\ell < j} b_\ell \cdot v_{p_j}(p_\ell - 1)
\end{equation}

For an exponent vector $\mathbf{b}$ to correspond to a divisor of $n$, the vector must satisfy the cascade constraints. Additionally, viewing the constraint violations as a system:
\begin{equation}
\Delta_j := b_j - \sum_{\ell < j} b_\ell \cdot v_{p_j}(p_\ell - 1) \geq 0 \quad \text{for all } j
\end{equation}

the defect $\Delta_j$ measures the "excess" of coordinate $j$ over its minimum required value from cascade constraints.

\noindent \textbf{Part A.1: Primality Determines Independent Constraint Dimensions}

For each prime $p_i$ that divides $n$, consider the constraint:
\begin{equation}
\label{eq:cascade-prime-constraint}
b_i \geq \sum_{j < i} b_j \cdot v_{p_i}(p_j - 1)
\end{equation}

This constraint couples coordinate $i$ to all prior coordinates $j < i$, weighted by $v_{p_i}(p_j - 1)$ (the $p_i$-adic valuation of $p_j - 1$).

For distinct primes $p_i$ and $p_i'$, the $p_i$-adic valuation of $(p_j - 1)$ and the $p_i'$-adic valuation of $(p_j - 1)$ are generally distinct and independent. That is, knowing $v_{p_i}(p_j - 1)$ provides no information about $v_{p_i'}(p_j - 1)$ for most pairs.

Therefore, the constraint from $p_i$ is linearly independent from the constraint from $p_i'$ when viewed as a constraint on the exponent vector space $\mathbb{Z}^m$.

\noindent \textbf{Part A.2: Linear Independence Over Integers}

To formalize independence: Consider the linear system over $\mathbb{Z}$:
\begin{equation}
M \mathbf{b} \geq \mathbf{0}
\end{equation}

where the matrix $M$ has rows corresponding to the cascade constraints. Specifically, for each prime $p_i \mid n$, row $i$ of $M$ is:
\begin{equation}
M_i = (-v_{p_i}(p_1 - 1), -v_{p_i}(p_2 - 1), \ldots, -v_{p_i}(p_{i-1} - 1), 1, 0, \ldots, 0)
\end{equation}

That is, the $j$-th entry (for $j < i$) is $-v_{p_i}(p_j - 1)$, the $i$-th entry is 1, and entries after $i$ are 0.

These rows are linearly independent over $\mathbb{Q}$ because they form a lower triangular matrix with 1's on the diagonal (when ordered by the $i$ index). The rank of this matrix is exactly $\omega(n)$.

To see this: The rows for $p_1, p_2, \ldots, p_k$ are:
\begin{align}
M_1 &= (1, 0, 0, \ldots, 0) \\
M_2 &= (v_{p_2}(p_1 - 1), 1, 0, \ldots, 0) \\
M_3 &= (v_{p_3}(p_1 - 1), v_{p_3}(p_2 - 1), 1, \ldots, 0) \\
&\vdots
\end{align}

(with signs absorbed into the inequality form). This is a lower triangular matrix, which has full rank $k$.

\noindent \textbf{Part B: Rank Equals Number of Prime Divisors}

The rank of the constraint matrix $M$ is $k = \omega(n)$, as established above. This means the constraint system has exactly $\omega(n)$ independent linear constraints on the exponent vector space.

The solution set to $M \mathbf{b} \geq \mathbf{0}$ (exponent vectors satisfying all cascade constraints) forms a polytope in $\mathbb{R}^m$. The dimension of this polytope is $m - k = m - \omega(n)$ (the number of coordinates minus the number of independent constraints).

\noindent \textbf{Part C: Relation to Divisors of n}

When restricting to exponent vectors of divisors of $n$, we further restrict the vector space to satisfy:
\begin{equation}
\prod_{j=1}^m p_j^{b_j} \mid n
\end{equation}

This means $b_j \leq e_j(n)$ for all $j$ (where $e_j(n)$ is the $j$-th coordinate in the epimoric encoding of $n$).

The number of divisors of $n$ is $\prod_i (a_i + 1)$, where $a_i$ is the exponent of prime $p_i$ in the factorization of $n$. However, the algebraic structure—the dimension of the constraint variety—is determined by the rank of the cascade constraint matrix, which is $\omega(n)$.

\noindent \textbf{Part D: Spectral Interpretation}

The Perron-Frobenius eigenvalue function $\lambda(s)$ of the weighted transfer operator exhibits critical points (non-analyticity or jump discontinuities) corresponding to each independent constraint.

Since there are exactly $\omega(n)$ linearly independent cascade constraints (one for each prime divisor of $n$), and each constraint becomes "active" at a different logarithmic scale $s = \log p_i$ (due to the coupling with $p_i$-adic valuations), the function $\lambda(s)$ undergoes exactly $\omega(n)$ distinct phase transitions as $s$ varies.

Therefore, the spectral characterization (counting critical points of $\lambda(s)$) recovers $\omega(n)$, establishing the equivalence between the rank-based characterization and the spectral characterization.

\end{proof}

\begin{definition}[Cascadic Defect Matrix for $n$]
\label{def:cascadic-defect-matrix}
For a positive integer $n$ with epimoric encoding $E(n) = (e_1, \ldots, e_m)$, the \emph{cascadic defect matrix} records the deficit at each position:
\begin{equation}
\Delta_j(n) := e_j(n) - D_j(E(n)_{<j})
\end{equation}
where $D_j$ is the cascade deficit function. The rank of this system (viewed as a linear constraint in the exponent space) is the number of linearly independent cascade constraints that must be satisfied modulo the multiplicity structure.
\end{definition}

\begin{observation}[Relationship to Prime Factorization]
The omega function can be computed directly via prime factorization: $\omega(n) = \#\{p \text{ prime} : p \mid n\}$. The cascade framework reveals this count through the spectral characterization, providing a connection to transfer operator theory and topological entropy analysis (developed in subsequent sections).

This characterization is more subtle than direct coordinate enumeration; it relies on the structural relationships encoded in the cascade constraints and the resulting defect patterns, not on counting degenerate coordinates directly.
\end{observation}

\subsection{Terminological Conventions}
\label{subsec:terminology}

Throughout this manuscript, the following terminology is used exclusively:
\begin{enumerate}
\item \textbf{Epimoric encoding} denotes the sequence $E(N)$ from Definition \ref{def:epimoric-encoding}.
\item \textbf{Epimoric sequence} is used interchangeably with epimoric encoding.
\item \textbf{Truncated canonical representation} refers to the finite sequence obtained by omitting trailing zeros, per Definition \ref{def:truncated-representation}.
\item \textbf{Coordinate degeneracy} describes the property defined in Definition \ref{def:degeneracy}.
\item Alternative terminology such as ``vector tails,'' ``support collapse,'' or ``spectral decay'' is not used.
\end{enumerate}

All subsequent sections build upon these definitions without repetition.

\subsection{Concrete Examples: Epimoric Factorizations and Omega Functions}
\label{subsec:epimoric-examples}

The following table provides comprehensive empirical data illustrating the epimoric factorization framework for integers 1 through 100. This table demonstrates the regularity and structure of epimoric distributions, documenting how the cascade constraint system governs the multiplicative structure of integers.

For each integer $n$:
\begin{itemize}
\item \textbf{Prime factorization} shows the classical representation as a list of prime power exponents.
\item \textbf{$\omega(n)$ and $\Omega(n)$} denote the traditional distinct and total prime divisor counts.
\item \textbf{Epimoric encoding $[b_k]$} shows the exponent vector in the epimoric basis (Definition \ref{def:epimoric-encoding}).
\item \textbf{$\omega_E(n)$ and $\Omega_E(n)$} denote the distinct and total coordinate counts in the epimoric encoding, which exhibit characteristic behaviors under the cascade constraint structure.
\end{itemize}

The table exhibits striking regularities: the epimoric representation expands the coordinate space compared to the prime factorization, while the epimoric omega counts exhibit algebraic coherence patterns derived from the cascade constraint structure.

\bigskip

\begin{table}[h!]
\label{tab:omegas-comprehensive}
\caption{Epimoric Factorizations and Omega Functions for Integers 1--100: Comprehensive Table Illustrating Regularity Patterns in Multiplicative Structure}
\begin{center}
\tiny
\begin{tabular}{|l|l|l|l|l|l|l|}
\hline
$n$ & Prime & $\omega(n)$ & $\Omega(n)$ & Epimoric & $\omega_E(n)$ & $\Omega_E(n)$ \\
\hline
1 & $[]$ & 0 & 0 & $[]$ & 0 & 0 \\
2 & $[1\rangle$ & 1 & 1 & $[1\rangle$ & 1 & 1 \\
3 & $[0,1\rangle$ & 1 & 1 & $[1,1\rangle$ & 2 & 2 \\
4 & $[2\rangle$ & 1 & 2 & $[2\rangle$ & 1 & 2 \\
5 & $[0,0,1\rangle$ & 1 & 1 & $[2,0,1\rangle$ & 2 & 3 \\
6 & $[1,1\rangle$ & 2 & 2 & $[2,1\rangle$ & 2 & 3 \\
7 & $[0,0,0,1\rangle$ & 1 & 1 & $[2,1,0,1\rangle$ & 3 & 4 \\
8 & $[3\rangle$ & 1 & 3 & $[3\rangle$ & 1 & 3 \\
9 & $[0,2\rangle$ & 1 & 2 & $[2,2\rangle$ & 2 & 4 \\
10 & $[1,0,1\rangle$ & 2 & 2 & $[3,0,1\rangle$ & 2 & 4 \\
11 & $[0,0,0,0,1\rangle$ & 1 & 1 & $[3,0,1,0,1\rangle$ & 3 & 5 \\
12 & $[2,1\rangle$ & 2 & 3 & $[3,1\rangle$ & 2 & 4 \\
13 & $[0,0,0,0,0,1\rangle$ & 1 & 1 & $[3,1,0,0,0,1\rangle$ & 3 & 5 \\
14 & $[1,0,0,1\rangle$ & 2 & 2 & $[3,1,0,1\rangle$ & 3 & 5 \\
15 & $[0,1,1\rangle$ & 2 & 2 & $[3,1,1\rangle$ & 3 & 5 \\
16 & $[4\rangle$ & 1 & 4 & $[4\rangle$ & 1 & 4 \\
17 & $[0,0,0,0,0,0,1\rangle$ & 1 & 1 & $[4,0,0,0,0,0,1\rangle$ & 2 & 5 \\
18 & $[1,2\rangle$ & 2 & 3 & $[3,2\rangle$ & 2 & 5 \\
19 & $[0,0,0,0,0,0,0,1\rangle$ & 1 & 1 & $[3,2,0,0,0,0,0,1\rangle$ & 3 & 6 \\
20 & $[2,0,1\rangle$ & 2 & 3 & $[4,0,1\rangle$ & 2 & 5 \\
21 & $[0,1,0,1\rangle$ & 2 & 2 & $[3,2,0,1\rangle$ & 3 & 6 \\
22 & $[1,0,0,0,1\rangle$ & 2 & 2 & $[4,0,1,0,1\rangle$ & 3 & 6 \\
23 & $[0,0,0,0,0,0,0,0,1\rangle$ & 1 & 1 & $[4,0,1,0,1,0,0,0,1\rangle$ & 4 & 7 \\
24 & $[3,1\rangle$ & 2 & 4 & $[4,1\rangle$ & 2 & 5 \\
25 & $[0,0,2\rangle$ & 1 & 2 & $[4,0,2\rangle$ & 2 & 6 \\
26 & $[1,0,0,0,0,1\rangle$ & 2 & 2 & $[4,1,0,0,0,1\rangle$ & 3 & 6 \\
27 & $[0,3\rangle$ & 1 & 3 & $[3,3\rangle$ & 2 & 6 \\
28 & $[2,0,0,1\rangle$ & 2 & 3 & $[4,1,0,1\rangle$ & 3 & 6 \\
29 & $[0,0,0,0,0,0,0,0,0,1\rangle$ & 1 & 1 & $[4,1,0,1,0,0,0,0,0,1\rangle$ & 4 & 7 \\
30 & $[1,1,1\rangle$ & 3 & 3 & $[4,1,1\rangle$ & 3 & 6 \\
31 & $[0,0,0,0,0,0,0,0,0,0,1\rangle$ & 1 & 1 & $[4,1,1,0,0,0,0,0,0,0,1\rangle$ & 4 & 7 \\
32 & $[5\rangle$ & 1 & 5 & $[5\rangle$ & 1 & 5 \\
33 & $[0,1,0,0,1\rangle$ & 2 & 2 & $[4,1,1,0,1\rangle$ & 4 & 7 \\
34 & $[1,0,0,0,0,0,1\rangle$ & 2 & 2 & $[5,0,0,0,0,0,1\rangle$ & 2 & 6 \\
35 & $[0,0,1,1\rangle$ & 2 & 2 & $[4,1,1,1\rangle$ & 4 & 7 \\
36 & $[2,2\rangle$ & 2 & 4 & $[4,2\rangle$ & 2 & 6 \\
37 & $[0,0,0,0,0,0,0,0,0,0,0,1\rangle$ & 1 & 1 & $[4,2,0,0,0,0,0,0,0,0,0,1\rangle$ & 3 & 6 \\
38 & $[1,0,0,0,0,0,0,1\rangle$ & 2 & 2 & $[4,2,0,0,0,0,0,1\rangle$ & 3 & 7 \\
39 & $[0,1,0,0,0,1\rangle$ & 2 & 2 & $[4,2,0,0,0,1\rangle$ & 3 & 7 \\
40 & $[3,0,1\rangle$ & 2 & 4 & $[5,0,1\rangle$ & 2 & 6 \\
41 & $[0,0,0,0,0,0,0,0,0,0,0,0,1\rangle$ & 1 & 1 & $[5,0,1,0,0,0,0,0,0,0,0,0,1\rangle$ & 3 & 7 \\
42 & $[1,1,0,1\rangle$ & 3 & 3 & $[4,2,0,1\rangle$ & 3 & 7 \\
43 & $[0,0,0,0,0,0,0,0,0,0,0,0,0,1\rangle$ & 1 & 1 & $[4,2,0,1,0,0,0,0,0,0,0,0,0,1\rangle$ & 4 & 7 \\
44 & $[2,0,0,0,1\rangle$ & 2 & 3 & $[5,0,1,0,1\rangle$ & 3 & 7 \\
45 & $[0,2,1\rangle$ & 2 & 3 & $[4,2,1\rangle$ & 3 & 7 \\
46 & $[1,0,0,0,0,0,0,0,1\rangle$ & 2 & 2 & $[5,0,1,0,1,0,0,0,1\rangle$ & 4 & 8 \\
47 & $[0,0,0,0,0,0,0,0,0,0,0,0,0,0,1\rangle$ & 1 & 1 & $[5,0,1,0,1,0,0,0,1,0,0,0,0,0,1\rangle$ & 5 & 8 \\
48 & $[4,1\rangle$ & 2 & 5 & $[5,1\rangle$ & 2 & 6 \\
49 & $[0,0,0,2\rangle$ & 1 & 2 & $[4,2,0,2\rangle$ & 3 & 8 \\
50 & $[1,0,2\rangle$ & 2 & 3 & $[5,0,2\rangle$ & 2 & 7 \\
51 & $[0,1,0,0,0,0,1\rangle$ & 2 & 2 & $[5,1,0,0,0,0,1\rangle$ & 3 & 7 \\
52 & $[2,0,0,0,0,1\rangle$ & 2 & 3 & $[5,1,0,0,0,1\rangle$ & 3 & 7 \\
53 & $[0,0,0,0,0,0,0,0,0,0,0,0,0,0,0,1\rangle$ & 1 & 1 & $[5,1,0,0,0,1,0,0,0,0,0,0,0,0,0,1\rangle$ & 4 & 8 \\
54 & $[1,3\rangle$ & 2 & 4 & $[4,3\rangle$ & 2 & 7 \\
55 & $[0,0,1,0,1\rangle$ & 2 & 2 & $[5,0,2,0,1\rangle$ & 3 & 8 \\
56 & $[3,0,0,1\rangle$ & 2 & 4 & $[5,1,0,1\rangle$ & 3 & 7 \\
57 & $[0,1,0,0,0,0,0,1\rangle$ & 2 & 2 & $[4,3,0,0,0,0,0,1\rangle$ & 3 & 8 \\
58 & $[1,0,0,0,0,0,0,0,0,1\rangle$ & 2 & 2 & $[5,1,0,1,0,0,0,0,0,1\rangle$ & 4 & 9 \\
59 & $[0,0,0,0,0,0,0,0,0,0,0,0,0,0,0,0,1\rangle$ & 1 & 1 & $[5,1,0,1,0,0,0,0,0,1,0,0,0,0,0,0,1\rangle$ & 5 & 10 \\
60 & $[2,1,1\rangle$ & 3 & 4 & $[5,1,1\rangle$ & 3 & 7 \\
61 & $[0,0,0,0,0,0,0,0,0,0,0,0,0,0,0,0,0,1\rangle$ & 1 & 1 & $[5,1,1,0,0,0,0,0,0,0,0,0,0,0,0,0,0,1\rangle$ & 4 & 9 \\
62 & $[1,0,0,0,0,0,0,0,0,0,1\rangle$ & 2 & 2 & $[5,1,1,0,0,0,0,0,0,0,1\rangle$ & 4 & 9 \\
63 & $[0,2,0,1\rangle$ & 2 & 3 & $[4,3,0,1\rangle$ & 3 & 8 \\
64 & $[6\rangle$ & 1 & 6 & $[6\rangle$ & 1 & 6 \\
65 & $[0,0,1,0,0,1\rangle$ & 2 & 2 & $[5,1,1,0,0,1\rangle$ & 4 & 8 \\
66 & $[1,1,0,0,1\rangle$ & 3 & 3 & $[5,1,1,0,1\rangle$ & 4 & 8 \\
67 & $[0,0,0,0,0,0,0,0,0,0,0,0,0,0,0,0,0,0,1\rangle$ & 1 & 1 & $[5,1,1,0,1,0,0,0,0,0,0,0,0,0,0,0,0,0,1\rangle$ & 5 & 10 \\
68 & $[2,0,0,0,0,0,1\rangle$ & 2 & 3 & $[6,0,0,0,0,0,1\rangle$ & 2 & 7 \\
69 & $[0,1,0,0,0,0,0,0,1\rangle$ & 2 & 2 & $[5,1,1,0,1,0,0,0,1\rangle$ & 5 & 9 \\
70 & $[1,0,1,1\rangle$ & 3 & 3 & $[5,1,1,1\rangle$ & 4 & 8 \\
71 & $[0,0,0,0,0,0,0,0,0,0,0,0,0,0,0,0,0,0,0,1\rangle$ & 1 & 1 & $[5,1,1,1,0,0,0,0,0,0,0,0,0,0,0,0,0,0,0,1\rangle$ & 5 & 10 \\
72 & $[3,2\rangle$ & 2 & 5 & $[5,2\rangle$ & 2 & 7 \\
73 & $[0,0,0,0,0,0,0,0,0,0,0,0,0,0,0,0,0,0,0,0,1\rangle$ & 1 & 1 & $[5,2,0,0,0,0,0,0,0,0,0,0,0,0,0,0,0,0,0,0,1\rangle$ & 3 & 8 \\
74 & $[1,0,0,0,0,0,0,0,0,0,0,1\rangle$ & 2 & 2 & $[5,2,0,0,0,0,0,0,0,0,0,1\rangle$ & 4 & 9 \\
75 & $[0,1,2\rangle$ & 2 & 3 & $[5,1,2\rangle$ & 3 & 8 \\
76 & $[2,0,0,0,0,0,0,1\rangle$ & 2 & 3 & $[5,2,0,0,0,0,0,1\rangle$ & 3 & 9 \\
77 & $[0,0,0,1,1\rangle$ & 2 & 2 & $[5,1,1,1,1\rangle$ & 5 & 9 \\
78 & $[1,1,0,0,0,1\rangle$ & 3 & 3 & $[5,2,0,0,0,1\rangle$ & 3 & 8 \\
79 & $[0,0,0,0,0,0,0,0,0,0,0,0,0,0,0,0,0,0,0,0,0,1\rangle$ & 1 & 1 & $[5,2,0,0,0,1,0,0,0,0,0,0,0,0,0,0,0,0,0,0,0,1\rangle$ & 4 & 10 \\
80 & $[4,0,1\rangle$ & 2 & 5 & $[6,0,1\rangle$ & 2 & 7 \\
81 & $[0,4\rangle$ & 1 & 4 & $[4,4\rangle$ & 2 & 8 \\
82 & $[1,0,0,0,0,0,0,0,0,0,0,0,1\rangle$ & 2 & 2 & $[6,0,1,0,0,0,0,0,0,0,0,0,1\rangle$ & 3 & 9 \\
83 & $[0,0,0,0,0,0,0,0,0,0,0,0,0,0,0,0,0,0,0,0,0,0,1\rangle$ & 1 & 1 & $[6,0,1,0,0,0,0,0,0,0,0,0,1,0,0,0,0,0,0,0,0,0,1\rangle$ & 4 & 11 \\
84 & $[2,1,0,1\rangle$ & 3 & 4 & $[5,2,0,1\rangle$ & 3 & 8 \\
85 & $[0,0,1,0,0,0,1\rangle$ & 2 & 2 & $[6,0,1,0,0,0,1\rangle$ & 3 & 8 \\
86 & $[1,0,0,0,0,0,0,0,0,0,0,0,0,1\rangle$ & 2 & 2 & $[5,2,0,1,0,0,0,0,0,0,0,0,0,1\rangle$ & 4 & 9 \\
87 & $[0,1,0,0,0,0,0,0,0,1\rangle$ & 2 & 2 & $[5,2,0,1,0,0,0,0,0,1\rangle$ & 4 & 9 \\
88 & $[3,0,0,0,1\rangle$ & 2 & 4 & $[6,0,1,0,1\rangle$ & 3 & 8 \\
89 & $[0,0,0,0,0,0,0,0,0,0,0,0,0,0,0,0,0,0,0,0,0,0,0,1\rangle$ & 1 & 1 & $[6,0,1,0,1,0,0,0,0,0,0,0,0,0,0,0,0,0,0,0,0,0,0,1\rangle$ & 4 & 11 \\
90 & $[1,2,1\rangle$ & 3 & 4 & $[5,2,1\rangle$ & 3 & 8 \\
91 & $[0,0,0,1,0,1\rangle$ & 2 & 2 & $[5,2,0,1,0,1\rangle$ & 4 & 9 \\
92 & $[2,0,0,0,0,0,0,0,1\rangle$ & 2 & 3 & $[6,0,1,0,1,0,0,0,1\rangle$ & 4 & 10 \\
93 & $[0,1,0,0,0,0,0,0,0,0,1\rangle$ & 2 & 2 & $[5,2,1,0,0,0,0,0,0,0,1\rangle$ & 4 & 10 \\
94 & $[1,0,0,0,0,0,0,0,0,0,0,0,0,0,1\rangle$ & 2 & 2 & $[6,0,1,0,1,0,0,0,1,0,0,0,0,0,1\rangle$ & 5 & 12 \\
95 & $[0,0,1,0,0,0,0,1\rangle$ & 2 & 2 & $[5,2,1,0,0,0,0,1\rangle$ & 4 & 10 \\
96 & $[5,1\rangle$ & 2 & 6 & $[6,1\rangle$ & 2 & 7 \\
97 & $[0,0,0,0,0,0,0,0,0,0,0,0,0,0,0,0,0,0,0,0,0,0,0,0,1\rangle$ & 1 & 1 & $[6,1,0,0,0,0,0,0,0,0,0,0,0,0,0,0,0,0,0,0,0,0,0,0,1\rangle$ & 2 & 7 \\
98 & $[1,0,0,2\rangle$ & 2 & 3 & $[5,2,0,2\rangle$ & 3 & 9 \\
99 & $[0,2,0,0,1\rangle$ & 2 & 3 & $[5,2,1,0,1\rangle$ & 4 & 9 \\
100 & $[2,0,2\rangle$ & 2 & 4 & $[6,0,2\rangle$ & 2 & 8 \\
\hline
\end{tabular}
\end{center}
\end{table}


\bigskip

\noindent\textbf{Interpretation:} The data in Table \ref{tab:omegas-comprehensive} demonstrates that epimoric factorizations capture multiplicative structure fundamentally differently than prime factorization. Numbers with identical traditional $\omega(n)$ values exhibit diverse epimoric $\omega_E(n)$ counts, manifesting hidden multiplicative organization. This regularity derives directly from the cascade constraint structure that governs valid exponent vectors.
