\subsection{Weighted Arithmetic and Convex Combinations}

\subsubsection{Replacement of Multiplicative Operations with Weighted Averaging}

In the normalized simplex representation, the traditional arithmetic operations of multiplication and division are replaced by operations on weighted probability distributions. For integers $n$ and $m$ with normalized exponent vectors $\mathbf{w}(n)$ and $\mathbf{w}(m)$:

\begin{equation}
n = \prod_k p_k^{a_k}, \quad m = \prod_k p_k^{b_k}
\end{equation}

The product $n \cdot m = \prod_k p_k^{a_k + b_k}$ corresponds to the vector sum:

\begin{equation}
\mathbf{a} + \mathbf{b} = (a_1 + b_1, a_2 + b_2, \ldots)
\end{equation}

However, in the \emph{normalized} domain, we operate with:

\begin{equation}
\mathbf{w}(n \cdot m) = \frac{\mathbf{a} + \mathbf{b}}{||\mathbf{a} + \mathbf{b}||_1} = \frac{\mathbf{a} + \mathbf{b}}{\Omega(n) + \Omega(m)}
\end{equation}

This is a \emph{weighted average} (convex combination) of the normalized vectors:

\begin{equation}
\mathbf{w}(n \cdot m) = \frac{\Omega(n)}{\Omega(n) + \Omega(m)} \mathbf{w}(n) + \frac{\Omega(m)}{\Omega(n) + \Omega(m)} \mathbf{w}(m)
\end{equation}

The weights are precisely the \emph{relative magnitudes} of the exponent sums. This transformation turns multiplicative closure into \emph{convex closure}.

\subsubsection{Convex Combination Framework}

A \emph{convex combination} of weight vectors is defined as:

\begin{equation}
\mathbf{w}_{\text{conv}} = \sum_{j=1}^{k} \lambda_j \mathbf{w}_j, \quad \lambda_j \geq 0, \; \sum_{j=1}^{k} \lambda_j = 1
\end{equation}

The set of all convex combinations of a finite set of vectors forms a \emph{convex polytope} (the convex hull). In our context:

\begin{itemize}
\item The \emph{extreme points} (vertices) are the normalized exponent vectors of prime numbers
\item Every composite integer corresponds to an interior point or face point of the convex hull
\item The position of $n$ in the polytope reflects its \emph{structural composition}
\end{itemize}

\subsubsection{Scaling Operations}

To ``scale'' an integer's influence or importance, we adjust its total weight (exponent sum). Define the \emph{scaled weight}:

\begin{equation}
\mathbf{w}_{\lambda}(n) = \frac{\lambda \cdot \mathbf{a}}{||\lambda \cdot \mathbf{a}||_1} = \frac{\lambda \cdot \mathbf{a}}{\lambda \cdot \Omega(n)} = \mathbf{w}(n)
\end{equation}

Scaling by a common factor $\lambda > 0$ does \emph{not change} the normalized weight vector. Only the magnitude changes, reflecting the scale-invariance of probability distributions.

However, we can define a \emph{magnitude-weighted} operation:

\begin{equation}
\mathbf{W}_{\lambda}(n) = (w_k(n), \lambda \Omega(n))
\end{equation}

which preserves the exponent sum magnitude as a secondary coordinate. This is useful for studying how integers of different sizes relate geometrically.

\subsubsection{Shifted Normalization and the Distribution of Debt}

For an integer $n$ represented in the shifted prime system with displacement $q$:

\begin{equation}
n = \prod_{k=1}^{m} (p_k + q)^{b_k^{(q)}}
\end{equation}

the normalized weights are:

\begin{equation}
w_k^{(q)} = \frac{b_k^{(q)}}{\sum_j |b_j^{(q)}|}
\end{equation}

The magnitude of $b_k^{(q)}$ (which can be negative) measures the ``debt'' or ``credit'' assigned to the $k$-th shifted prime in explaining $n$. The distribution $\mathbf{w}^{(q)}$ exhibits:

\begin{itemize}
\item \textbf{Which shifted primes provide structural support}: Large $|b_k^{(q)}|$ denotes strong dependence
\item \textbf{Direction of support}: Positive $b_k^{(q)}$ means upward shift required; negative means downward
\item \textbf{Concentration patterns}: High entropy $\mathbf{w}^{(q)}$ means support is distributed; low entropy means concentrated
\end{itemize}

For a given $q$, the map $n \mapsto \mathbf{w}^{(q)}(n)$ is a different embedding of the integer into the simplex, and varying $q$ provides \emph{multiple lenses} for examining the same integer.

\subsubsection{Barycentric Interpolation and Intermediate Integers}

Given two integers $n$ and $m$, define their \emph{barycentric interpolation}:

\begin{equation}
I_{\lambda}(n, m) = \text{the unique integer whose normalized weights are } \lambda \mathbf{w}(n) + (1-\lambda)\mathbf{w}(m)
\end{equation}

For $\lambda \in (0, 1)$, this defines a ``path'' of integers in the simplex from $n$ to $m$. However, since not all points on the simplex correspond to integers, $I_{\lambda}(n, m)$ is not always uniquely defined; instead, it represents an equivalence class of integers with similar factorization profiles.

The \emph{interpolation error} measures how far the closest integer lies from the exact barycentric position:

\begin{equation}
\epsilon_{\lambda} = \min_{n' \in \mathbb{N}} ||\lambda \mathbf{w}(n) + (1-\lambda) \mathbf{w}(m) - \mathbf{w}(n')||_2
\end{equation}

Small interpolation error indicates that integers with intermediate factorization profiles are \emph{dense} in the simplex.

\subsubsection{Closure Properties and Algebraic Structure}

The convex combinations of normalized exponent vectors form a \emph{convex cone}:

\begin{equation}
\mathcal{C} = \left\{ \sum_j \lambda_j \mathbf{w}_j : \lambda_j \geq 0, \; \mathbf{w}_j \in \Delta^{\infty} \right\}
\end{equation}

Key properties:
\begin{enumerate}
\item \textbf{Closure under convex combination}: If $\mathbf{w}, \mathbf{w}' \in \mathcal{C}$ and $\lambda \in [0,1]$, then $\lambda \mathbf{w} + (1-\lambda) \mathbf{w}' \in \mathcal{C}$
\item \textbf{Closure under positive scaling}: If $\mathbf{w} \in \mathcal{C}$ and $\alpha > 0$, then $\alpha \mathbf{w} \in \mathcal{C}$ (though scaling violates the simplex constraint)
\item \textbf{Extremal points}: The vertices of the cone are the prime weight vectors $\mathbf{w}(p_k)$
\end{enumerate}

The quotient structure $\mathcal{C} / \sim$ (where $\sim$ identifies vectors differing by a positive scalar) recovers the projective simplex, which is the natural space for comparing factorization structures without regard to scale.

\subsubsection{Coupling and Optimal Transport}

The theory of \emph{optimal transport} (Monge-Kantorovich theory) provides tools for measuring ``distances'' between factorization distributions. The \emph{Wasserstein distance} between two normalized exponent distributions is:

\begin{equation}
W_p(\mathbf{w}(n), \mathbf{w}(m)) = \left(\inf_{\gamma} \int_{\mathbb{P} \times \mathbb{P}} d(k, \ell)^p \, d\gamma(k, \ell)\right)^{1/p}
\end{equation}

where:
\begin{itemize}
\item $\gamma$ ranges over all couplings (joint distributions) with marginals $\mathbf{w}(n)$ and $\mathbf{w}(m)$
\item $d(k, \ell)$ is a ground metric on the prime indices (e.g., $d(k, \ell) = |\log p_k - \log p_{\ell}|$)
\end{itemize}

The Wasserstein distance is a metric on probability distributions that respects the geometry of the underlying space (the primes), making it ideal for studying how factorization structures vary across the integer lattice.

\subsubsection{Geometric Algebra of Simplices}

The normalized exponent vectors live in a \emph{Clifford algebra} structure when extended with a sign grading. This allows the formulation of:

\begin{itemize}
\item \textbf{Exterior products}: Representing intersection of support sets
\item \textbf{Clifford products}: Combining multiplicative and additive structures
\item \textbf{Spinor representations}: Encoding orientation and chiral factorization structure
\end{itemize}

The Clifford algebraic perspective connects the simplex structure to the broader geometric framework of the manuscript, providing a unified language for all multi-perspective analyses.
