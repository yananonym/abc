\section{Omega Counting Functions: Prime versus Prime-Numerator Epimoric}

The fundamental distinction between prime and prime-numerator epimoric systems becomes sharp when examining counting functions. We define parallel families of functions for both representations.

\subsection{Prime Counting Functions}

For a natural number $n$ with prime factorization $n = \prod_{k=1}^{\infty} p_k^{a_k}$, define:

\begin{align}
\omega(n) &= \text{number of distinct primes dividing } n = \#\{k : a_k > 0\} \\
\Omega(n) &= \text{number of prime factors counted with multiplicity} = \sum_{k=1}^{\infty} a_k
\end{align}

For example, $\omega(60) = \omega(2^2 \cdot 3 \cdot 5) = 3$ and $\Omega(60) = 2 + 1 + 1 = 4$.

\subsection{Prime-Numerator Epimoric Omega Functions}

For the prime-numerator epimoric factorization $n = \prod_{k=1}^{\infty} (p_k/(p_k-1))^{b_k}$, define:

\begin{align}
\omega_E(n) &= \text{number of nonzero exponents in prime-numerator epimoric form} = \#\{k : b_k > 0\} \\
\Omega_E(n) &= \text{sum of all exponents in prime-numerator epimoric form} = \sum_{k=1}^{\infty} b_k
\end{align}

For the prime-denominator epimoric direction $n = \prod_{k=1}^{\infty} ((p_k+1)/p_k)^{c_k}$ (where exponents may be negative), define the unsigned count:

\begin{align}
\Omega_E^{\text{unsigned}}(n) &= \sum_{k=1}^{\infty} |c_k|
\end{align}

The asymmetry between these directions provides insight into prime distribution:

\begin{equation}
A(n) = \Omega_E(n) - \Omega_E^{\text{unsigned}}(n)
\end{equation}

\subsection{Key Observation: Omega Functions Coincide for Distinct Primes}

The distinct prime count is preserved exactly:
\begin{equation}
\omega_E(n) = \omega(n)
\end{equation}

The largest prime dividing $n$ has nonzero exponent in both representations, and primes correspond bijectively between systems.

However, the total count with multiplicity differs substantially:
\begin{equation}
\Omega_E(n) \geq \Omega(n)
\end{equation}

The gap $\Omega_E(n) - \Omega(n)$ arises from the denominator contributions $(p_k - 1)$ in the prime-numerator epimoric ratios.

\subsection{Table of Decompositions: $n = 1$ to $100$}

The following table presents prime, prime-numerator epimoric, and prime-denominator epimoric decompositions with corresponding omega functions:

\vspace{0.5cm}

\begin{center}
\tiny
\begin{tabular}{|c|c|c|c|c|c|c|c|}
\hline
$n$ & Prime & $\omega(n)$ & $\Omega(n)$ & Prime-Numerator Epimoric $[b_k]$ & $\Omega_E(n)$ & Prime-Denominator Epimoric $[c_k]$ & $\Omega_E^{\text{uns}}$ \\
\hline
1 & $[]$ & 0 & 0 & $[]$ & 0 & $[]$ & 0 \\
2 & $[1]$ & 1 & 1 & $[1]$ & 1 & $[1,1]$ & 2 \\
3 & $[0,1]$ & 1 & 1 & $[1,1]$ & 2 & $[2,1]$ & 3 \\
4 & $[2]$ & 1 & 2 & $[2]$ & 2 & $[2,2]$ & 4 \\
5 & $[0,0,1]$ & 1 & 1 & $[2,0,1]$ & 3 & $[3,2,-1]$ & 6 \\
6 & $[1,1]$ & 2 & 2 & $[2,1]$ & 3 & $[3,2]$ & 5 \\
7 & $[0,0,0,1]$ & 1 & 1 & $[2,1,0,1]$ & 4 & $[3,3,0,-1]$ & 7 \\
8 & $[3]$ & 1 & 3 & $[3]$ & 3 & $[3,3]$ & 6 \\
9 & $[0,2]$ & 1 & 2 & $[2,2]$ & 4 & $[4,2]$ & 6 \\
10 & $[1,0,1]$ & 2 & 2 & $[3,0,1]$ & 4 & $[4,3,-1]$ & 8 \\
11 & $[0,0,0,0,1]$ & 1 & 1 & $[3,0,1,0,1]$ & 5 & $[4,3,0,0,-1]$ & 8 \\
12 & $[2,1]$ & 2 & 3 & $[3,1]$ & 4 & $[4,3]$ & 7 \\
13 & $[0,0,0,0,0,1]$ & 1 & 1 & $[3,1,0,0,0,1]$ & 5 & $[4,4,0,-1,0,-1]$ & 9 \\
14 & $[1,0,0,1]$ & 2 & 2 & $[3,1,0,1]$ & 5 & $[4,4,0,-1]$ & 9 \\
15 & $[0,1,1]$ & 2 & 2 & $[3,1,1]$ & 5 & $[5,3,-1]$ & 9 \\
16 & $[4]$ & 1 & 4 & $[4]$ & 4 & $[4,4]$ & 8 \\
17 & $[0,0,0,0,0,0,1]$ & 1 & 1 & $[4,0,0,0,0,0,1]$ & 5 & $[5,3,0,0,0,0,-1]$ & 9 \\
18 & $[1,2]$ & 2 & 3 & $[3,2]$ & 5 & $[5,3]$ & 8 \\
19 & $[0,0,0,0,0,0,0,1]$ & 1 & 1 & $[3,2,0,0,0,0,0,1]$ & 6 & $[5,4,-1,0,0,0,0,-1]$ & 11 \\
20 & $[2,0,1]$ & 2 & 3 & $[4,0,1]$ & 5 & $[5,4,-1]$ & 10 \\
21 & $[0,1,0,1]$ & 2 & 2 & $[3,2,0,1]$ & 6 & $[5,4,0,-1]$ & 10 \\
22 & $[1,0,0,0,1]$ & 2 & 2 & $[4,0,1,0,1]$ & 6 & $[5,4,0,0,-1]$ & 10 \\
23 & $[0,0,0,0,0,0,0,0,1]$ & 1 & 1 & $[4,0,1,0,1,0,0,0,1]$ & 7 & $[5,4,0,0,0,0,0,0,-1]$ & 10 \\
24 & $[3,1]$ & 2 & 4 & $[4,1]$ & 5 & $[5,4]$ & 9 \\
25 & $[0,0,2]$ & 1 & 2 & $[4,0,2]$ & 6 & $[6,4,-2]$ & 12 \\
26 & $[1,0,0,0,0,1]$ & 2 & 2 & $[4,1,0,0,0,1]$ & 6 & $[5,5,0,-1,0,-1]$ & 12 \\
27 & $[0,3]$ & 1 & 3 & $[3,3]$ & 6 & $[6,3]$ & 9 \\
28 & $[2,0,0,1]$ & 2 & 3 & $[4,1,0,1]$ & 6 & $[5,5,0,-1]$ & 11 \\
29 & $[0,0,0,0,0,0,0,0,0,1]$ & 1 & 1 & $[4,1,0,1,0,0,0,0,0,1]$ & 7 & $[6,4,-1,0,0,0,0,0,0,-1]$ & 12 \\
30 & $[1,1,1]$ & 3 & 3 & $[4,1,1]$ & 6 & $[6,4,-1]$ & 11 \\
31 & $[0,0,0,0,0,0,0,0,0,0,1]$ & 1 & 1 & $[4,1,1,0,0,0,0,0,0,0,1]$ & 7 & $[5,5,0,0,0,0,0,0,0,0,-1]$ & 10 \\
32 & $[5]$ & 1 & 5 & $[5]$ & 5 & $[5,5]$ & 10 \\
33 & $[0,1,0,0,1]$ & 2 & 2 & $[4,1,1,0,1]$ & 7 & $[6,4,0,0,-1]$ & 11 \\
34 & $[1,0,0,0,0,0,1]$ & 2 & 2 & $[5,0,0,0,0,0,1]$ & 6 & $[6,4,0,0,0,0,-1]$ & 11 \\
35 & $[0,0,1,1]$ & 2 & 2 & $[4,1,1,1]$ & 7 & $[6,5,-1,-1]$ & 13 \\
36 & $[2,2]$ & 2 & 4 & $[4,2]$ & 6 & $[6,4]$ & 10 \\
37 & $[0,0,0,0,0,0,0,0,0,0,0,1]$ & 1 & 1 & $[4,2,0,0,0,0,0,0,0,0,0,1]$ & 6 & $[6,5,-1,0,0,0,0,-1,0,0,0,-1]$ & 15 \\
38 & $[1,0,0,0,0,0,0,1]$ & 2 & 2 & $[4,2,0,0,0,0,0,1]$ & 7 & $[6,5,-1,0,0,0,0,-1]$ & 13 \\
39 & $[0,1,0,0,0,1]$ & 2 & 2 & $[4,2,0,0,0,1]$ & 7 & $[6,5,0,-1,0,-1]$ & 13 \\
40 & $[3,0,1]$ & 2 & 4 & $[5,0,1]$ & 6 & $[6,5,-1]$ & 12 \\
41 & $[0,0,0,0,0,0,0,0,0,0,0,0,1]$ & 1 & 1 & $[5,0,1,0,0,0,0,0,0,0,0,0,1]$ & 7 & $[6,5,0,-1,0,0,0,0,0,0,0,0,-1]$ & 13 \\
42 & $[1,1,0,1]$ & 3 & 3 & $[4,2,0,1]$ & 7 & $[6,5,0,-1]$ & 12 \\
43 & $[0,0,0,0,0,0,0,0,0,0,0,0,0,1]$ & 1 & 1 & $[4,2,0,1,0,0,0,0,0,0,0,0,0,1]$ & 7 & $[6,5,0,0,-1,0,0,0,0,0,0,0,0,-1]$ & 14 \\
44 & $[2,0,0,0,1]$ & 2 & 3 & $[5,0,1,0,1]$ & 7 & $[6,5,0,0,-1]$ & 12 \\
45 & $[0,2,1]$ & 2 & 3 & $[4,2,1]$ & 7 & $[7,4,-1]$ & 12 \\
46 & $[1,0,0,0,0,0,0,0,1]$ & 2 & 2 & $[5,0,1,0,1,0,0,0,1]$ & 8 & $[6,5,0,0,0,0,0,0,-1]$ & 12 \\
47 & $[0,0,0,0,0,0,0,0,0,0,0,0,0,0,1]$ & 1 & 1 & $[5,0,1,0,1,0,0,0,1,0,0,0,0,0,1]$ & 8 & $[6,5,0,0,0,0,0,0,0,0,0,0,0,0,-1]$ & 11 \\
48 & $[4,1]$ & 2 & 5 & $[5,1]$ & 6 & $[6,5]$ & 11 \\
49 & $[0,0,0,2]$ & 1 & 2 & $[4,2,0,2]$ & 8 & $[6,6,0,-2]$ & 14 \\
50 & $[1,0,2]$ & 2 & 3 & $[5,0,2]$ & 7 & $[7,5,-2]$ & 14 \\
51 & $[0,1,0,0,0,0,1]$ & 2 & 2 & $[5,1,0,0,0,0,1]$ & 7 & $[7,4,0,0,0,0,-1]$ & 12 \\
52 & $[2,0,0,0,0,1]$ & 2 & 3 & $[5,1,0,0,0,1]$ & 7 & $[6,6,0,-1,0,-1]$ & 14 \\
53 & $[0,0,0,0,0,0,0,0,0,0,0,0,0,0,0,1]$ & 1 & 1 & $[5,1,0,0,0,1,0,0,0,0,0,0,0,0,0,1]$ & 8 & $[7,4,0,0,0,0,0,0,0,0,0,0,0,0,0,-1]$ & 12 \\
54 & $[1,3]$ & 2 & 4 & $[4,3]$ & 7 & $[7,4]$ & 11 \\
55 & $[0,0,1,0,1]$ & 2 & 2 & $[5,0,2,0,1]$ & 8 & $[7,5,-1,0,-1]$ & 14 \\
56 & $[3,0,0,1]$ & 2 & 4 & $[5,1,0,1]$ & 7 & $[6,6,0,-1]$ & 13 \\
57 & $[0,1,0,0,0,0,0,1]$ & 2 & 2 & $[4,3,0,0,0,0,0,1]$ & 7 & $[7,5,-1,0,0,0,0,-1]$ & 14 \\
58 & $[1,0,0,0,0,0,0,0,0,1]$ & 2 & 2 & $[5,1,0,1,0,0,0,0,0,1]$ & 8 & $[7,5,-1,0,0,0,0,0,0,-1]$ & 14 \\
59 & $[0,0,0,0,0,0,0,0,0,0,0,0,0,0,0,0,1]$ & 1 & 1 & $[5,1,0,1,0,0,0,0,0,1,0,0,0,0,0,0,1]$ & 8 & $[7,5,-1,0,0,0,0,0,0,0,0,0,0,0,0,0,-1]$ & 14 \\
60 & $[2,1,1]$ & 3 & 4 & $[5,1,1]$ & 7 & $[7,5,-1]$ & 13 \\
61 & $[0,0,0,0,0,0,0,0,0,0,0,0,0,0,0,0,0,1]$ & 1 & 1 & $[5,1,1,0,0,0,0,0,0,0,0,0,0,0,0,0,0,1]$ & 8 & $[6,6,0,0,0,0,0,0,0,0,-1,0,0,0,0,0,0,-1]$ & 14 \\
62 & $[1,0,0,0,0,0,0,0,0,0,1]$ & 2 & 2 & $[5,1,1,0,0,0,0,0,0,0,1]$ & 8 & $[6,6,0,0,0,0,0,0,0,0,-1]$ & 12 \\
63 & $[0,2,0,1]$ & 2 & 3 & $[4,3,0,1]$ & 8 & $[7,5,0,-1]$ & 13 \\
64 & $[6]$ & 1 & 6 & $[6]$ & 6 & $[6,6]$ & 12 \\
65 & $[0,0,1,0,0,1]$ & 2 & 2 & $[5,1,1,0,0,1]$ & 8 & $[7,6,-1,-1,0,-1]$ & 16 \\
66 & $[1,1,0,0,1]$ & 3 & 3 & $[5,1,1,0,1]$ & 8 & $[7,5,0,0,-1]$ & 13 \\
67 & $[0,0,0,0,0,0,0,0,0,0,0,0,0,0,0,0,0,0,1]$ & 1 & 1 & $[5,1,1,0,1,0,0,0,0,0,0,0,0,0,0,0,0,0,1]$ & 8 & $[7,5,0,0,0,0,-1,0,0,0,0,0,0,0,0,0,0,0,-1]$ & 15 \\
68 & $[2,0,0,0,0,0,1]$ & 2 & 3 & $[6,0,0,0,0,0,1]$ & 7 & $[7,5,0,0,0,0,-1]$ & 13 \\
69 & $[0,1,0,0,0,0,0,0,1]$ & 2 & 2 & $[5,1,1,0,1,0,0,0,1]$ & 8 & $[7,5,0,0,0,0,0,0,-1]$ & 13 \\
70 & $[1,0,1,1]$ & 3 & 3 & $[5,1,1,1]$ & 8 & $[7,6,-1,-1]$ & 15 \\
71 & $[0,0,0,0,0,0,0,0,0,0,0,0,0,0,0,0,0,0,0,1]$ & 1 & 1 & $[5,1,1,1,0,0,0,0,0,0,0,0,0,0,0,0,0,0,0,1]$ & 8 & $[7,5,0,0,0,0,0,0,0,0,0,0,0,0,0,0,0,0,0,-1]$ & 13 \\
72 & $[3,2]$ & 2 & 5 & $[5,2]$ & 7 & $[7,5]$ & 12 \\
73 & $[0,0,0,0,0,0,0,0,0,0,0,0,0,0,0,0,0,0,0,0,1]$ & 1 & 1 & $[5,2,0,0,0,0,0,0,0,0,0,0,0,0,0,0,0,0,0,0,1]$ & 7 & $[7,6,-1,0,0,0,0,-1,0,0,0,-1,0,0,0,0,0,0,0,0,-1]$ & 20 \\
74 & $[1,0,0,0,0,0,0,0,0,0,0,1]$ & 2 & 2 & $[5,2,0,0,0,0,0,0,0,0,0,1]$ & 7 & $[7,6,-1,0,0,0,0,-1,0,0,0,-1]$ & 18 \\
75 & $[0,1,2]$ & 2 & 3 & $[5,1,2]$ & 8 & $[8,5,-2]$ & 15 \\
76 & $[2,0,0,0,0,0,0,1]$ & 2 & 3 & $[5,2,0,0,0,0,0,1]$ & 8 & $[7,6,-1,0,0,0,0,-1]$ & 15 \\
77 & $[0,0,0,1,1]$ & 2 & 2 & $[5,1,1,1,1]$ & 9 & $[7,6,0,-1,-1]$ & 15 \\
78 & $[1,1,0,0,0,1]$ & 3 & 3 & $[5,2,0,0,0,1]$ & 8 & $[7,6,0,-1,0,-1]$ & 15 \\
79 & $[0,0,0,0,0,0,0,0,0,0,0,0,0,0,0,0,0,0,0,0,0,1]$ & 1 & 1 & $[5,2,0,0,0,1,0,0,0,0,0,0,0,0,0,0,0,0,0,0,0,1]$ & 8 & $[7,6,-1,0,0,0,0,0,0,0,0,0,0,0,0,0,0,0,0,0,0,-1]$ & 15 \\
80 & $[4,0,1]$ & 2 & 5 & $[6,0,1]$ & 7 & $[7,6,-1]$ & 14 \\
81 & $[0,4]$ & 1 & 4 & $[4,4]$ & 8 & $[8,4]$ & 12 \\
82 & $[1,0,0,0,0,0,0,0,0,0,0,0,1]$ & 2 & 2 & $[6,0,1,0,0,0,0,0,0,0,0,0,1]$ & 8 & $[7,6,0,-1,0,0,0,0,0,0,0,0,-1]$ & 15 \\
83 & $[0,0,0,0,0,0,0,0,0,0,0,0,0,0,0,0,0,0,0,0,0,0,1]$ & 1 & 1 & $[6,0,1,0,0,0,0,0,0,0,0,0,1,0,0,0,0,0,0,0,0,0,1]$ & 8 & $[7,6,0,-1,0,0,0,0,0,0,0,0,0,0,0,0,0,0,0,0,0,0,-1]$ & 15 \\
84 & $[2,1,0,1]$ & 3 & 4 & $[5,2,0,1]$ & 8 & $[7,6,0,-1]$ & 14 \\
85 & $[0,0,1,0,0,0,1]$ & 2 & 2 & $[6,0,1,0,0,0,1]$ & 7 & $[8,5,-1,0,0,0,-1]$ & 15 \\
86 & $[1,0,0,0,0,0,0,0,0,0,0,0,0,1]$ & 2 & 2 & $[5,2,0,1,0,0,0,0,0,0,0,0,0,1]$ & 8 & $[7,6,0,0,-1,0,0,0,0,0,0,0,0,-1]$ & 15 \\
87 & $[0,1,0,0,0,0,0,0,0,1]$ & 2 & 2 & $[5,2,0,1,0,0,0,0,0,1]$ & 8 & $[8,5,-1,0,0,0,0,0,0,-1]$ & 15 \\
88 & $[3,0,0,0,1]$ & 2 & 4 & $[6,0,1,0,1]$ & 8 & $[7,6,0,0,-1]$ & 14 \\
89 & $[0,0,0,0,0,0,0,0,0,0,0,0,0,0,0,0,0,0,0,0,0,0,0,1]$ & 1 & 1 & $[6,0,1,0,1,0,0,0,0,0,0,0,0,0,0,0,0,0,0,0,0,0,0,1]$ & 8 & $[8,5,-1,0,0,0,0,0,0,0,0,0,0,0,0,0,0,0,0,0,0,0,0,-1]$ & 15 \\
90 & $[1,2,1]$ & 3 & 4 & $[5,2,1]$ & 8 & $[8,5,-1]$ & 14 \\
91 & $[0,0,0,1,0,1]$ & 2 & 2 & $[5,2,0,1,0,1]$ & 8 & $[7,7,0,-2,0,-1]$ & 17 \\
92 & $[2,0,0,0,0,0,0,0,1]$ & 2 & 3 & $[6,0,1,0,1,0,0,0,1]$ & 8 & $[7,6,0,0,0,0,0,0,-1]$ & 14 \\
93 & $[0,1,0,0,0,0,0,0,0,0,1]$ & 2 & 2 & $[5,2,1,0,0,0,0,0,0,0,1]$ & 8 & $[7,6,0,0,0,0,0,0,0,0,-1]$ & 14 \\
94 & $[1,0,0,0,0,0,0,0,0,0,0,0,0,0,1]$ & 2 & 2 & $[6,0,1,0,1,0,0,0,1,0,0,0,0,0,1]$ & 9 & $[7,6,0,0,0,0,0,0,0,0,0,0,0,0,-1]$ & 13 \\
95 & $[0,0,1,0,0,0,0,1]$ & 2 & 2 & $[5,2,1,0,0,0,0,1]$ & 8 & $[8,6,-2,0,0,0,0,-1]$ & 17 \\
96 & $[5,1]$ & 2 & 6 & $[6,1]$ & 7 & $[7,6]$ & 13 \\
97 & $[0,0,0,0,0,0,0,0,0,0,0,0,0,0,0,0,0,0,0,0,0,0,0,0,1]$ & 1 & 1 & $[6,1,0,0,0,0,0,0,0,0,0,0,0,0,0,0,0,0,0,0,0,0,0,0,1]$ & 7 & $[7,7,0,-2,0,0,0,0,0,0,0,0,0,0,0,0,0,0,0,0,0,0,0,0,-1]$ & 17 \\
98 & $[1,0,0,2]$ & 2 & 3 & $[5,2,0,2]$ & 9 & $[7,7,0,-2]$ & 16 \\
99 & $[0,2,0,0,1]$ & 2 & 3 & $[5,2,1,0,1]$ & 9 & $[8,5,0,0,-1]$ & 14 \\
100 & $[2,0,2]$ & 2 & 4 & $[6,0,2]$ & 8 & $[8,6,-2]$ & 16 \\
\hline
\end{tabular}
\end{center}

\vspace{0.5cm}

\subsection{Observations from the Extended Table}

Observe that $\omega(n) = \omega_E(n)$ for all entries, confirming that the count of distinct primes is identical in both systems. However, $\Omega_E(n) \geq \Omega(n)$ consistently throughout the table. The inflation of $\Omega_E$ arises because denominator prime factors in the prime-numerator epimoric ratios require absorption through the cumulative exponent structure.

In the prime-denominator epimoric direction $(p_k+1)/p_k$, exponents frequently become negative, particularly for primes near prime gaps. The unsigned count $\Omega_E^{\text{unsigned}}(n)$ grows more rapidly than $\Omega_E(n)$, reflecting the contribution of negative exponents. The asymmetry between these two directions—prime-numerator epimoric versus prime-denominator epimoric—encodes information about the relative positioning of integers and primes.

For prime $p$: $\Omega_E(p)$ shows growth depending on $\pi(p)$ (the prime counting function), while $\Omega_E^{\text{unsigned}}(p)$ exhibits larger values reflecting the necessity of negative exponents in the prime-denominator direction. This asymmetry reflects fundamental algebraic constraints arising from the multiplicative structure of integers.