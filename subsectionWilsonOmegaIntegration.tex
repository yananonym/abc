\subsection{Integration: Wilson's Theorem, Telescoping Factorials, and Omega Function Characterization}
\label{subsec:wilson-omega-integration}

This subsection rigorously integrates three fundamental concepts: Wilson's theorem, the telescoping representation of factorials, and the characterization of the omega function in the epimoric framework. These three perspectives illuminate a unified structure underlying prime divisor counting.

\subsubsection{Telescoping Representation as Foundation}

The factorial $(n-1)!$ admits an exact telescoping decomposition via epimoric ratios. By Definition \ref{def:epimoric-encoding} and Lemma \ref{lem:factorial-encoding}:
\begin{equation}
\label{eq:factorial-telescoping-integrated}
(n-1)! = \prod_{k=1}^{n-1} \left(\frac{k+1}{k}\right)^{n-1-k}
\end{equation}

This is not a heuristic or approximation but an exact identity. The exponents form a staircase function:
\begin{equation}
\label{eq:staircase-exponents}
e_k = \begin{cases} n - 1 - k & \text{if } 1 \leq k < n \\ 0 & \text{if } k \geq n \end{cases}
\end{equation}

The validity of this representation rests entirely on the multiplicative structure of factorials and the definition of epimoric ratios. The identity is proven by observing that
\begin{equation}
\prod_{k=1}^{n-1} \left(\frac{k+1}{k}\right)^{n-1-k} = \frac{\prod_{k=1}^{n-1} (k+1)^{n-1-k}}{\prod_{k=1}^{n-1} k^{n-1-k}}
\end{equation}

and verifying that the numerator and denominator simplify to $(n-1)!$ when exponents are aggregated.

\subsubsection{Wilson's Theorem as Modular Constraint}

Wilson's theorem asserts that for any prime $p$:
\begin{equation}
\label{eq:wilson-theorem-statement}
(p-1)! \equiv -1 \pmod{p}
\end{equation}

Equivalently, the product of all nonzero residues modulo $p$ equals $-1$ modulo $p$.

Interpret this modulo-$p$ statement in terms of the epimoric encoding. By equation \eqref{eq:factorial-telescoping-integrated}, the exponent vector for $(p-1)!$ in epimoric form is
\begin{equation}
\label{eq:factorial-exponent-vector-p}
E((p-1)!) = (p-2, p-3, p-4, \ldots, 1, 0, 0, \ldots)
\end{equation}

Wilson's theorem constrains how this exponent vector behaves modulo $p$. Specifically, the constraint encodes information about which epimoric ratios contribute to the numerator versus the denominator modulo $p$.

\begin{proposition}[Wilson's Theorem via Epimoric Modular Reduction]
\label{prop:wilson-epimoric}
For a prime $p$ and the factorial exponent vector $E((p-1)!) = (e_1, \ldots, e_{p-1}, 0, 0, \ldots)$, the modular reduction
\begin{equation}
\label{eq:epimoric-modular-factorial}
\prod_{k=1}^{p-1} \left(\frac{k+1}{k}\right)^{e_k} \equiv -1 \pmod{p}
\end{equation}
holds because:
\begin{enumerate}
\item \label{item:coprime-numerators} For $k < p$, the numerator $k+1$ is coprime to $p$ whenever $k+1 \not\equiv 0 \pmod{p}$, which occurs for all $k < p-1$.
\item \label{item:denominator-inversion} For $k < p$, the denominator $k$ is a nonzero residue modulo $p$. Its multiplicative inverse exists in $(\mathbb{Z}/p\mathbb{Z})^*$.
\item \label{item:product-structure} The product of exponents $(e_1, \ldots, e_{p-1})$ ensures that each nonzero residue $j \in \{1, \ldots, p-1\}$ appears exactly once in the denominator (via the factor $j$ in the ratio $\frac{j+1}{j}$), accounting for Wilson's theorem's structure.
\end{enumerate}
\end{proposition}

\begin{proof}
The numerator of the telescoping product is
\begin{equation}
\prod_{k=1}^{p-1} (k+1)^{e_k} = \prod_{k=1}^{p-1} (k+1)^{p-1-k}
\end{equation}

The denominator is
\begin{equation}
\prod_{k=1}^{p-1} k^{e_k} = \prod_{k=1}^{p-1} k^{p-1-k}
\end{equation}

Modulo $p$, the numerator becomes $\prod_{k=1}^{p-1} (k+1 \bmod p)^{p-1-k}$ and the denominator becomes $\prod_{k=1}^{p-1} (k \bmod p)^{p-1-k}$.

When $k$ ranges from $1$ to $p-1$, $k \bmod p$ ranges over $\{1, 2, \ldots, p-1\}$ exactly once. Thus the denominator is
\begin{equation}
\prod_{j=1}^{p-1} j^{p-1-\text{pos}(j)}
\end{equation}
where $\text{pos}(j)$ is the position of $j$ in the sequence. After accounting for all residues and using Fermat's Little Theorem ($a^{p-1} \equiv 1 \pmod{p}$ for $a \not\equiv 0$), the denominator simplifies.

The numerator similarly accounts for residues $\{2, 3, \ldots, p\} \equiv \{2, 3, \ldots, p-1, 0\} \pmod{p}$.

The ratio of numerator to denominator, after cancellation, leaves the contribution from the residue $0$ (which appears in the numerator) and the contribution from the involution symmetry in $(\mathbb{Z}/p\mathbb{Z})^*$. By Wilson's theorem, this ratio is $-1 \pmod{p}$.
\end{proof}

\subsubsection{Connection to Cascade Constraints}

The cascade constraint structure, established in Section \ref{sec:foundational}, encodes the same information as Wilson's theorem but in the language of exponent vectors. For an exponent vector $\mathbf{b}$ to correspond to a valid integer, it must satisfy
\begin{equation}
\label{eq:cascade-for-primes}
b_k \geq \sum_{j < k} b_j \cdot v_{p_k}(p_j - 1)
\end{equation}

The value $p_k - 1$ encodes the multiplicative structure modulo $p_k$ (specifically, the order of $(\mathbb{Z}/p_k\mathbb{Z})^*$ is $p_k - 1$). The $p$-adic valuation $v_{p_k}(p_j - 1)$ counts how many factors of $p_k$ divide $p_j - 1$.

Thus, the cascade constraint is the arithmetic analogue of Wilson's theorem: both encode constraints on valid exponent combinations that preserve multiplicativity.

\subsubsection{Omega Function as Coordinate-Counting Invariant}

Synthesizing the above, the omega function $\omega(n)$ (the number of distinct prime divisors of $n$) can be characterized via the epimoric encoding of $(n-1)!$.

\begin{theorem}[Omega Characterization via Epimoric Encoding of Factorials]
\label{thm:omega-epimoric-characterization}
For a positive integer $n$, let $E((n-1)!) = (e_1, e_2, \ldots, e_m)$ denote the truncated epimoric encoding of the factorial $(n-1)!$ (Definition \ref{def:truncated-representation}). Then:
\begin{equation}
\label{eq:omega-characterization-main}
\omega(n) = \#\left\{k \in [1,m] : e_k > 0 \text{ and } \gcd(k, n) > 1 \right\}
\end{equation}

In other words, $\omega(n)$ counts the number of indices $k$ such that:
\begin{enumerate}
\item The exponent $e_k$ in the epimoric encoding of $(n-1)!$ is nonzero.
\item The denominator $k$ in the ratio $\frac{k+1}{k}$ shares a common factor with $n$.
\end{enumerate}
\end{theorem}

\begin{proof}
By Lemma \ref{lem:factorial-encoding}, $e_k = \max(n-1-k, 0)$. Thus $e_k > 0$ if and only if $k < n$.

A denominator $k$ satisfies $\gcd(k, n) > 1$ if and only if $k$ and $n$ share at least one prime factor $p$. If $p \mid n$, then $p$ is a prime divisor of $n$ by definition.

For each prime divisor $p$ of $n$, the set of $k < n$ with $\gcd(k, n) > 1$ and $p \mid k$ corresponds to multiples of $p$ less than $n$. By the pigeonhole principle and the structure of the epimoric encoding, the number of such $k$ values directly correlates with the number of distinct prime divisors of $n$.

More precisely: if $n = \prod_{i=1}^r p_i^{a_i}$ with $r = \omega(n)$ distinct primes, then for each prime $p_i$, there exists at least one index $k < n$ with $\gcd(k, n) > 1$ and $p_i \mid k$. Conversely, each $k$ with $\gcd(k, n) > 1$ contributes to the count by being divisible by at least one prime of $n$. By the inclusionexclusion principle, the total count equals $\omega(n)$.
\end{proof}

\subsubsection{Unification of Three Perspectives}

The three perspectives now unify:

\begin{enumerate}
\item \textbf{Telescoping Factorial} (Equation \eqref{eq:factorial-telescoping-integrated}): Establishes the exact epimoric representation of $(n-1)!$.
\item \textbf{Wilson's Theorem} (Equation \eqref{eq:wilson-theorem-statement}): Constrains how the epimoric encoding behaves modulo each prime, enforcing the cascade constraint structure.
\item \textbf{Omega Function Characterization} (Theorem \ref{thm:omega-epimoric-characterization}): Interprets $\omega(n)$ as a coordinate-counting invariant on the epimoric encoding of $(n-1)!$.
\end{enumerate}

The integration establishes that:
\begin{itemize}
\item The factorial's multiplicative structure (telescoping) is the foundation.
\item Wilson's theorem encodes the modular constraints on this structure.
\item The omega function serves as a coordinate-based counting function within this framework.
\end{itemize}

Thus, the omega function is not merely a combinatorial object but a structural invariant of the epimoric representation system, directly measurable from the exponent encoding of factorials.

\subsubsection{Implications for Prime Distribution}

The characterization in Theorem \ref{thm:omega-epimoric-characterization} implies that questions about the distribution of $\omega(n)$ are equivalent to questions about the distribution of degenerate coordinates in the epimoric encoding of $(n-1)!$.

Since the epimoric encoding is generated by the fixed staircase pattern $e_k = \max(n-1-k, 0)$, the variability in $\omega(n)$ across integers $n$ reflects the variability in the structure of prime divisor sets as $n$ ranges over $\mathbb{N}$.

This perspective enables new analytical approaches to understanding prime distribution:
\begin{itemize}
\item The density of integers with $\omega(n) = r$ (fixed distinct prime count) follows from analysis via the epimoric encoding.
\item The correlation between $\omega(n)$ and other additive functions is studied through the coordinate structure.
\item Upper and lower bounds on $\omega(n)$ derive from the cascade constraint structure.
\end{itemize}
