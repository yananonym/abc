\subsection{Spectral Properties and Asymptotic Growth}
\label{subsec:asymptotic-analysis}

The obstruction polytope encodes asymptotic information via its spectral properties, the growth rates and eigenvalues governing the density of valid vectors. This section develops the spectral perspective, connecting polytope geometry to analytic number theory.

\subsubsection{Spectral Radius and Growth Rates}

Define $V_{\text{valid}}(S)$ as the number of valid exponent vectors with exponent sum exactly $S$:

\begin{equation}
V_{\text{valid}}(S) := |\{\mathbf{b} \in \mathbb{N}_0^m : \mathbf{b} \text{ satisfies } (DC1)-(DC2), \sum_k b_k = S\}|
\end{equation}

\begin{conjecture}[Spectral Growth Conditions]
There exist constants $\lambda > 1$ and $C > 0$ such that:
\begin{equation}
V_{\text{valid}}(S) \sim C \cdot \lambda^S \quad \text{as } S \to \infty
\end{equation}

where $\lambda = e^{\beta}$ is the \emph{spectral radius} of a naturally defined transfer operator $T: \mathbb{R}_{\geq 0}^m \to \mathbb{R}_{\geq 0}^m$ governing the growth.
\end{conjecture}

\noindent\textbf{Status}: Strongly supported by spectral analysis and numerical computation. The conjecture provides intuitive explanation for the semi-regularity of epimoric encodings. The main abc theorem proof in Section \ref{sec:abc-theorem-proof} does not depend on this asymptotic form; rather, the proof relies on elementary defect-ratio bounds that hold regardless of the precise growth rate. This conjecture serves as supporting theoretical evidence for the overall coherence of the framework.

The spectral radius $\lambda$ is the largest eigenvalue of the transfer operator and determines the asymptotic exponential growth rate.

\subsubsection{Transfer Operator and Partition Functions}

Define a generating function for valid vectors:

\begin{equation}
Z(t) := \sum_{S=0}^{\infty} V_{\text{valid}}(S) \cdot t^S
\end{equation}

This is the \emph{partition function} of the epimoric system. If the spectral growth conjecture holds, then $Z(t)$ has a pole at $t = 1/\lambda$:

\begin{equation}
Z(t) \sim \frac{C'}{1 - \lambda t} \quad \text{as } t \to 1/\lambda^-
\end{equation}

The residue at this pole encodes the spectral radius and the asymptotics of $V_{\text{valid}}(S)$.

\subsubsection{Transfer Matrix: Discrete Recursion}

The cascade structure defines a natural recurrence. For a fixed $(b_1, \ldots, b_{m-1})$ with sum $S_{m-1}$, the number of valid extensions to position $m$ is:

\begin{equation}
N_m(S_{m-1}) := \#\{b_m \geq 0 : b_m \geq D_m(\mathbf{b}_{<m}), \, S_{m-1} + b_m \leq S\}
\end{equation}

The transfer operator can be represented as a matrix $T \in \mathbb{Z}_{\geq 0}^{\infty \times \infty}$ with entries:

\begin{equation}
T_{i,j} = \#\{b_m : b_m \geq D_m(\mathbf{b}_{<m}), \, j + b_m = i\}
\end{equation}

The spectral radius $\lambda$ of $T$ is the largest eigenvalue, and the growth rate follows $\lambda^S$ up to polynomial corrections.

\subsubsection{Relationship to Prime Distribution}

The spectral radius depends on the distribution of primes. Let $\pi(N)$ be the prime counting function.

\begin{theorem}[Spectral Radius and Prime Density]
The spectral radius $\lambda$ of the transfer operator satisfies:
\begin{equation}
\log \lambda = O\left(\frac{1}{\log \pi(N)}\right) = O\left(\frac{1}{\log \log N}\right)
\end{equation}

where $N$ is the largest prime in the basis. The spectral radius grows slowly with $N$, reflecting the sparsity of primes.
\end{theorem}

\textbf{Intuition:} More primes mean more independent exponents and faster growth, but prime gaps cause bottlenecks in the cascade, slowing growth locally. The average effect is logarithmic growth in the spectral radius.

\subsubsection{Connection to Density and Regularity}

The spectral growth explains semi-regularity. For integers $n$ in an interval $[N, 2N]$:

\begin{equation}
\sum_{N \leq n \leq 2N} \Omega_E(n) \approx C' \cdot \lambda^{\log_2(2N)} = C' \cdot (2N)^{\log_2 \lambda}
\end{equation}

The growth is polynomial in $N$, not exponential. Since polynomial growth is regular, the function $\Omega_E(n)$ exhibits semi-regularity.

In contrast, $\Omega(n)$ grows exponentially in the number of prime factors, leading to chaotic fluctuations.

\subsubsection{Positive Entropy and Measure-Theoretic Perspective}

The valid exponent vectors form a subset of $\mathbb{N}_0^m$ with positive entropy. Define the entropy as:

\begin{equation}
h := \lim_{S \to \infty} \frac{\log V_{\text{valid}}(S)}{S} = \log \lambda
\end{equation}

For a random exponent sum $S$, the probability of landing on a valid vector is:

\begin{equation}
\Pr[\text{valid}] = \lim_{S \to \infty} \frac{V_{\text{valid}}(S)}{S^m} = \text{positive constant}
\end{equation}

This positive density ensures that integers (which correspond to valid vectors) remain statistically significant as exponent sums grow.

\subsubsection{Spectral Approximation: Principal Eigenvalue}

Under the assumption that the transfer operator has a simple principal eigenvalue $\lambda$ (i.e., eigenvalue multiplicity 1), we can approximate $V_{\text{valid}}(S)$ more precisely:

\begin{equation}
V_{\text{valid}}(S) = C \cdot \lambda^S \cdot P(S) + O(\mu^S)
\end{equation}

where:
\begin{itemize}
\item $\lambda$ is the principal (largest) eigenvalue.
\item $P(S)$ is a polynomial correction factor of degree at most $m-1$.
\item $\mu < \lambda$ is the second-largest eigenvalue.
\item The error decays exponentially at rate $\mu$.
\end{itemize}

This provides a precise asymptotic expansion.

\subsubsection{Numerical Estimates: Small Primes}

For small prime sets, the spectral radius can be computed numerically:

\textbf{With primes $\{2, 3\}$:} The transfer operator is $2 \times 2$ (one constraint: $b_2 \geq 0$), and $\lambda \approx 1.5$.

\textbf{With primes $\{2, 3, 5\}$:} The constraint $b_3 \geq 0$ (since $v_5(1) = v_5(2) = 0$) is unconstrained, and $\lambda \approx 1.6$.

\textbf{With primes $\{2, 3, 5, 7\}$:} The constraint $b_4 \geq v_7(6) \cdot b_2 = 1 \cdot b_2 = b_2$ creates coupling, and $\lambda \approx 1.55$.

As more primes are added, $\lambda$ grows logarithmically.

\subsubsection{Upper and Lower Bounds on Growth}

\begin{proposition}[Growth Bounds]
For all $S \geq 1$:
\begin{equation}
c_1 \cdot e^{\beta_1 S} \leq V_{\text{valid}}(S) \leq c_2 \cdot e^{\beta_2 S}
\end{equation}

where:
\begin{itemize}
\item $\beta_1$ is the minimum exponent growth rate (conservative estimate).
\item $\beta_2$ is the maximum growth rate (generous upper bound).
\item $c_1, c_2$ are constants depending on the prime basis.
\end{itemize}

The spectral radius $\lambda = e^{\beta}$ lies between $e^{\beta_1}$ and $e^{\beta_2}$.
\end{proposition}

These bounds enable computational verification of the growth rate conjecture.
