\subsection{Homological and Topological Structure}

The valid exponent vectors admit a homological interpretation via exact sequences and derived functors. This perspective reveals deep obstructions in the lattice structure that persist across scales.

\subsubsection{Exact Sequence of Valuation Constraints}

The integrality constraints fit into an exact sequence in the category of finitely generated abelian groups:

\begin{definition}[Valuation Exact Sequence]
\begin{equation}
0 \to \text{Valid}(\mathcal{P}) \xrightarrow{\iota} \mathbb{N}_0^m \xrightarrow{\phi} \bigoplus_q \mathbb{Z}_{\geq 0} \xrightarrow{\psi} \text{Obstr} \to 0
\end{equation}

where:
\begin{itemize}
\item $\text{Valid}(\mathcal{P})$ is the subgroup of valid exponent vectors.
\item $\iota$ is the inclusion map.
\item $\phi(\mathbf{b}) = \left(v_q\left(\prod_k p_k^{b_k}\right)\right)_q$ maps to numerator valuations.
\item $\psi(v_q) = \left(v_q - v_q\left(\prod_k (p_k-1)^{b_k}\right)\right)_q$ computes the divisibility deficit.
\item $\text{Obstr}$ is the cokernel, the group of divisibility obstructions.
\end{itemize}
\end{definition}

\textbf{Exactness Interpretation:} An exponent vector is valid (in the image of $\iota$) if and only if its numerator valuations $\phi(\mathbf{b})$ have zero deficit under $\psi$, i.e., the numerator dominates the denominator at every prime.

\subsubsection{Tor Groups and Torsion}

Applying the derived functor $\text{Tor}_*^{\mathbb{Z}}(-, \mathbb{Z})$ to the exact sequence yields a long exact sequence in Tor:

\begin{equation}
\cdots \to \text{Tor}_1(\text{Obstr}, \mathbb{Z}) \to \text{Valid}(\mathcal{P}) \otimes \mathbb{Z} \to \mathbb{N}_0^m \otimes \mathbb{Z} \to \cdots
\end{equation}

The Tor groups measure torsion in the valuation module:

\begin{definition}[Torsion Module]
$\text{Tor}_1(\text{Obstr}, \mathbb{Z})$ is the group of torsion elements in $\text{Obstr}$, corresponding to obstructions that cancel out over a multiple of each prime.
\end{definition}

For each prime $q$, the torsion at $q$ is:
\begin{equation}
\text{Tor}_q := \{x \in \text{Obstr} : m \cdot x = 0 \text{ for some } m \in \mathbb{Z}\}
\end{equation}

\subsubsection{Spectral Sequence: Computing the Homology}

Define graded modules $\mathcal{V}_p$ corresponding to valuations at each distinct prime $p$:

\begin{equation}
\mathcal{V}_p := \mathbb{Z} / \langle \gcd\{v_p(p_k - 1) : k\} \rangle
\end{equation}

This is a free module if $\gcd = 1$ (the prime $p$ divides $(p_k - 1)$ for distinct values of $k$), and is torsion otherwise.

A spectral sequence computes the homology of the valid vector complex:

\begin{definition}[Valuation Spectral Sequence]
\begin{equation}
E_1^{p,q} = \text{Tor}_q\left(\mathcal{V}_p, \mathbb{Z}\right) \Rightarrow H_{p+q}(\text{Valid}(\mathcal{P}); \mathbb{Z})
\end{equation}

The $E_1$ page consists of Tor groups of the valuation modules at each prime. The differential $d_1$ measures how valuations at different primes interact via the cascade constraints.
\end{definition}

\subsubsection{Degeneration Conjecture}

\begin{conjecture}[Spectral Sequence Degeneration]
The spectral sequence degenerates at the $E_2$ page: $E_2 = E_\infty$.
\end{conjecture}

\noindent\textbf{Status}: Unproven. Partial structural evidence from cascade cocycle theory. Key obstacle: establishing the equivalence between spectral sequences in the exponent space polytope and classical algebraic topology constructions.

If true, this implies:

\begin{equation}
H_*(\text{Valid}(\mathcal{P})) \cong \bigoplus_{p} \text{Tor}_*(\mathcal{V}_p, \mathbb{Z})
\end{equation}

\textbf{Consequence:} The topology of the valid vector space is entirely determined by the algebraic structure of valuations of $\{p_k-1\}$ at earlier primes. No additional topological complexity emerges from the cascade interactions.

\subsubsection{Homology Groups}

Under the degeneration conjecture, the homology groups are:

\begin{proposition}[Homology Structure]
\begin{equation}
H_0(\text{Valid}(\mathcal{P}); \mathbb{Z}) = \mathbb{Z}
\end{equation}
(one connected component, the identity)

\begin{equation}
H_1(\text{Valid}(\mathcal{P}); \mathbb{Z}) = \bigoplus_p \text{Tor}_1(\mathcal{V}_p, \mathbb{Z})
\end{equation}

For $n \geq 2$, $H_n(\text{Valid}(\mathcal{P})) = 0$ (no higher homology).
\end{proposition}

The first homology group is generated by 1-cycles (loops) corresponding to prime-gap anomalies.

\subsubsection{Persistent Homology}

As the exponent sum $S$ increases, the valid vector sets form a filtered complex:

\begin{equation}
\mathcal{C}_0 \subset \mathcal{C}_1 \subset \mathcal{C}_2 \subset \cdots
\end{equation}

where $\mathcal{C}_S = V_S \cap \mathbb{Z}^m$ (lattice points with exponent sum $S$).

The persistent homology of this filtration encodes topological features that persist across multiple values of $S$:

\begin{definition}[Persistent Homology Barcode]
A barcode is a multiset of intervals $[b, d)$ (birth time, death time), each corresponding to a persistent homology class. An interval $[b, d)$ records a topological feature that is born at exponent sum $b$ and dies at $S = d$.
\end{definition}

\subsubsection{Gap Detection via Persistence}

\begin{theorem}[Prime Gap-Persistence Correspondence]
A prime gap of size $\gamma_k = p_{k+1} - p_k$ produces a persistent 1-homology class with:
\begin{itemize}
\item Birth time $b \approx \sum_{j=1}^k v_{p_j}(p_k - 1)$ (when the gap first impacts the exponent system).
\item Lifespan (length of persistence) approximately $\log(\gamma_k)$.
\item Death time $d \approx b + \log(\gamma_k)$.
\end{itemize}

Larger gaps produce longer-lived persistent homology bars.
\end{theorem}

\textbf{Example:} The gap $\gamma_1 = 3 - 2 = 1$ (trivial) produces no persistent classes. The gap $\gamma_2 = 5 - 3 = 2$ produces a persistent class with lifespan $\approx \log(2)$. The gap $\gamma_3 = 7 - 5 = 2$ also produces a class with lifespan $\approx \log(2)$.

\subsubsection{Computational Persistent Homology}

The persistent homology barcode can be computed via standard algorithms:

\begin{algorithm}
\caption{Persistent Homology Computation}
\begin{algorithmic}
\FUNCTION{ComputeBarcode}{$\mathcal{C}_0, \mathcal{C}_1, \ldots, \mathcal{C}_{S_{\max}}$}
    \STATE Initialize empty barcode
    \FOR{$S = 0$ to $S_{\max}$}
        \STATE Build boundary matrix of $\mathcal{C}_S$ from valid vectors
        \STATE Reduce boundary matrix to normal form
        \STATE Extract persistent pairs (birth, death) from reduced matrix
        \STATE Add to barcode
    \ENDFOR
    \RETURN barcode
\ENDFUNCTION
\end{algorithmic}
\end{algorithm}

This computation is polynomial in the exponent sum and the number of primes, making it computationally tractable for moderate sizes.

\subsubsection{Obstruction Cycles and Non-Triviality}

A persistent 1-homology class corresponds to a \emph{loop} or \emph{cycle} in the graph of valid vectors, a closed path that cannot be contracted.

\begin{proposition}[Cycle Obstruction]
An obstruction cycle occurs when the cascade constraints create a "bottleneck": two exponent vectors $\mathbf{b}$ and $\mathbf{b}'$ are adjacent (differ by 1 in one coordinate) but the edge between them crosses a constraint boundary, forcing a detour through other vectors that increases the exponent sum.
\end{proposition}

\textbf{Example:} With primes $\{2, 3, 5\}$, the constraint $b_3 \geq v_5(6) \cdot b_2 = 0$ is vacuous. But with primes $\{2, 3, 7\}$, the constraint $b_3 \geq v_7(6) \cdot b_2 = 1 \cdot b_2 = b_2$ creates a bottleneck: to move from $(b_1, 0, b_3)$ to $(b_1, 1, b_3)$, we must increase $b_3$ to at least 1, creating a loop if we try to return while decreasing $b_3$.

\subsubsection{Implications for Prime Distribution}

The topological obstructions detected by persistent homology correspond to structural anomalies in the prime sequence:

\begin{conjecture}[Persistent Homology and Prime Gap Detection]
The barcode of persistent homology of the valid vector complex encodes the prime distribution. Specifically, the total persistence (sum of all bar lengths) is:
\begin{equation}
\text{TotalPersistence} = \sum_k \log(\gamma_k) + O(1)
\end{equation}

where the sum is over all prime gaps. Deviations from this prediction indicate rare prime gaps (anomalies in prime distribution).
\end{conjecture}

\noindent\textbf{Status}: Unproven, exploratory. Supporting evidence: computational experiments on polytope faces up to $n=100$. Key challenge: Rigorously connecting barcodes to prime gap asymptotics.

Persistent homology provides a tool for detecting and analyzing prime gap anomalies.
